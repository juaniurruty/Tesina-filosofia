\documentclass[11pt]{article}
\usepackage[authoryear]{natbib}
\bibliographystyle{apa-en.bst}

%\usepackage{times}
\usepackage{setspace}
\onehalfspacing

%\usepackage[natbibapa]{apacite}
%\bibliographystyle{apacite}
%\bibliographystyle{apa}
%\setlength{\bibsep}{10pt}

\usepackage[utf8]{inputenc}
\usepackage[spanish]{babel}
\usepackage{subfigure}
\usepackage{graphicx}
\usepackage{float}
%\usepackage[authoryear]{natbib}
%\usepackage[dvipsnames]{xcolor}
\usepackage[svgnames]{xcolor}
\usepackage[colorlinks=true,allcolors=DarkBlue]{hyperref}
\usepackage[a4paper]{geometry}
\geometry{top=2.5cm, bottom=2.5cm, left=2.5cm, right=2.5cm}
\usepackage{caption}
%\usepackage{natbib}
\usepackage{multirow}
\usepackage{booktabs,threeparttable}
%\setlength{\parindent}{0pt}
\usepackage{amsmath}
\usepackage[labelfont=bf]{caption}
\usepackage{hyperref}
\setcounter{tocdepth}{5}
\setcounter{secnumdepth}{5}
\usepackage{dirtytalk}
\usepackage{xcolor}
\usepackage[shortlabels]{enumitem}
\usepackage{times}
%\usepackage[skip=2pt plus0pt, indent=10pt]{parskip}

%\setstretch{20}

%\spacing{2}
\usepackage[symbol]{footmisc}




\begin{document}

\begin{titlepage}
\begin{center}

%\begin{figure}[H]
%\centering
   % \includegraphics[width =1\textwidth]{Fhuce.png}
   % \end{figure}

\begin{figure}[H]
\centering
    \includegraphics[width =0.81\textwidth]{fhuce2.jpg}
    \end{figure}
    
\vspace{9mm}

\huge{\textbf{Justicia e incentivos: desde la competencia hacia la cooperación}}


\vspace{18mm}

\LARGE 
\textbf{Juan Ignacio Urruty Rodríguez}

\vspace{70mm}


\large{

Licenciatura en Filosofía

Facultad de Humanidades y Ciencias de la Educación

Universidad de la República}

\vspace{10mm}

Montevideo - Uruguay

Junio de 2023



\end{center}
\end{titlepage}

\begin{center}

\thispagestyle{empty}



\huge{\textbf{Justicia e incentivos: desde la competencia hacia la cooperación}}


\vspace{18mm}

\LARGE{ 
\textbf{Juan Ignacio Urruty Rodríguez}}
\end{center}

\vspace{30mm}

\large{Tesina de Grado presentada en el marco de la Licenciatura en Filosofía de la Facultad de Humanidades y Ciencias de la Educación de la Universidad de la República, como parte de los requisitos necesarios para la obtención del título de Licenciado en Filosofía.

\vspace{30mm}

\begin{center}

Tutor: 

Profesor Agustín Reyes


\vspace{55mm}

Licenciatura en Filosofía

Facultad de Humanidades y Ciencias de la Educación

Universidad de la República

\vspace{10mm}

Montevideo - Uruguay

%Enero de 2023
    
\end{center}}






\newpage

\doublespacing
\begin{center}

\thispagestyle{empty}

% Borrar la parte que dice opcional
\normalsize Agradecimientos

\end{center}

\vspace{3mm}

% Comentar en el main en caso de no querer escribir agradecimientos. En otro 
% caso, redactar la dedicatoria luego del comando \normalsize
\normalsize Gracias a esta carrera tuve la suerte de conocer a un conjunto de grandes personas con las cuales mantengo una valiosa amistad en la que nos une este mismo interés por la reflexión filosófica.

Por eso quiero agradecer a Yamila Lara, Facundo Correa, Paula Ibiñete y Elena Gomes. Haber compartido esta carrera con ellos hizo que mi experiencia en la Facultad de Humanidades fuera muy amena y enriquecedora.

En la misma línea, quiero agradecer también a mi tutor Agustín Reyes quien fue de gran ayuda a la hora de elaborar mi tesina. Tener la posibilidad de intercambiar ideas de manera frecuente con Agustín fue muy importante y se encuentra reflejado en el trabajo que realicé.

Por otro lado, quiero agradecer a mi familia: mi madre Sofía, mi padre Fernando, mi hermana Evangelina y mi abuela Lila. El apoyo y cuidado de ellos ha sido fundamental para convertirme en la persona que soy. Sus convicciones me llevaron a perseguir este camino académico. Prometo seguir adelante con mucho trabajo y esfuerzo.

Finalmente, quiero agradecer a mi amiga y colega economista Lucía Bertoletti. Lucía se tomo la molestia de leer uno de los primeros borradores de este trabajo y sus comentarios brillantes, como siempre, me fueron de mucha utilidad.

\newpage

% Epígrafe (opcional): frase que aluda al tema de la tesis
\begin{flushright}

\thispagestyle{empty}

% Comentar en el main en caso de no querer incluir un epígrafe. En caso de 
% querer escribir una frase, reemplazar en \say{\textit{AQUI}}
\say{\textit{La mejor victoria es vencer sin combatir}}

\vspace{3mm}

% Reemplazar con la autora de la frase
\textit{Sun Tzu}

\end{flushright}

% Salto de página
\newpage

\onehalfspacing
\begin{center}
\thispagestyle{empty}
    \huge{\textbf{Justicia e incentivos: desde la competencia hacia la cooperación}}
    
    \renewcommand{\thefootnote}{\fnsymbol{footnote}}
    
    \vspace{10mm}
    
    \large Juan Ignacio Urruty\footnote[1]{Licenciado en Economía - Facultad de Ciencias Económicas y Administración, Universidad de la República. Correo de contacto: juaniurruty9@gmail.com}
    
    \vspace{20mm}
    
   \large \textbf{Resumen}
    
    \vspace{5mm}
    
\end{center}



\begin{quote}
\small

    Los incentivos económicos diferenciales, ¿pueden entrar en conflicto con la idea de justicia distributiva? El objetivo del siguiente trabajo es realizar un recorrido por una de las líneas de investigación respecto de la pregunta de los incentivos. El trabajo se divide de la siguiente manera. Como punto de partida, en la sección \ref{sec1}, se presenta una visión desde la teoría económica neoclásica. En la sección \ref{sec2}, se introduce el concepto de justicia distributiva. A continuación, se presenta una crítica normativa de la teoría neoclásica respecto de las consecuencias que se siguen de la misma. Dicha crítica busca evidenciar la necesidad de brindar una respuesta alternativa. Por lo cual, se presenta la idea del principio de la diferencia propuesto por John Rawls. Si bien este concepto implica un tratamiento normativo de mayor alcance que la teoría neoclásica, no está exento de críticas. En este sentido, en la sección \ref{sec3}, se analiza la crítica realizada por Gerald A. Cohen a ciertas implicancias del principio de la diferencia. En la misma línea, se presenta la propuesta de Cohen respecto de la igualdad de oportunidades socialista y la necesidad de un \textit{ethos} igualitarista. Dicha propuesta es recogida por John E. Roemer, por lo cual, en la sección \ref{sec4} se presenta un intento de modelización económica de la misma. Finalmente, en la sección \ref{sec5} se presentan un par de reflexiones que buscan dar cuenta de las limitaciones y virtudes de la línea de investigación que se recorre en este trabajo.
    
    \vspace{5mm}
    
    \textit{Palabras clave:} justicia distributiva, incentivos económicos, Gerald A. Cohen, John E. Roemer.
\end{quote}


\newpage 



\normalsize



\doublespacing
\tableofcontents
\thispagestyle{empty}
\newpage

\doublespacing
\renewcommand*{\thefootnote}{\arabic{footnote}}
\setcounter{footnote}{0}


\doublespacing
\section{Introducción} \label{sec1}

\setcounter{page}{1}

La disciplina económica puede ser entendida de diversas maneras. Una formulación ampliamente generalizada es que la misma se encarga de la asignación de recursos escasos. Existen distintos criterios que podrían guiar dicha asignación. En particular, en los análisis económicos, se le brinda un rol sustantivo al criterio de la eficiencia. De manera no tan rigurosa, una forma de presentar el concepto de eficiencia es en contraposición al de eficacia. La eficacia implica el cumplimiento de un objetivo concreto mientras que la eficiencia requiere que dicho cumplimiento sea empleando la menor cantidad de recursos posibles. 


Desde la óptica de la producción de bienes, se podría aceptar provisionalmente que una situación con una mayor cantidad de bienes es preferible respecto de otra en donde hay una cantidad menor. En este sentido, un objetivo por parte de las firmas podría ser el de producir la mayor cantidad de bienes con el menor costo posible. Conseguir esta asignación de recursos productivos sería una forma de garantizar la eficiencia técnica. A la hora de querer operacionalizar esta intuición, los modelos microeconómicos simples llegan a soluciones o equilibrios. En estos equilibrios se logra: por un lado, la maximización de beneficios o la minimización de costos desde la óptica de las empresas\footnote{Bajo los supuestos estándar de la teoría microeconómica básica, resolver el problema de la maximización de beneficios es equivalente a resolver la minimización de costos ya que estos problemas presentan una relación de dualidad \citep{MasColell_1995}.}; por otro lado, desde la óptica de los trabajadores, se logra la maximización de su utilidad tomando en cuenta las decisiones respecto del trabajo y del ocio.


Estos modelos emplean ciertos supuestos para tratar de reducir la complejidad del fenómeno social. En particular, uno de los supuestos más extendidos en la disciplina económica es la idea del egoísmo racional. Este supuesto plantea que los individuos buscan tomar la acción que les brinde mayor utilidad o bienestar\footnote{Si bien cuestionar este supuesto podría ser un tema de investigación en sí mismo, en este trabajo se discutirá este punto de manera secundaria.}. De la aplicación de estos modelos, se llega al resultado de que la maximización de beneficios implica que la productividad marginal de los factores productivos debe ser igual a su remuneración. 

De este resultado se sigue que los salarios son iguales a la productividad marginal del trabajo. Por lo cual, un modelo microeconómico básico plantea que las personas más productivas deben obtener una remuneración mayor. En la misma línea, es posible complejizar la situación introduciendo asimetrías de información entre los agentes. Concretamente, si las firmas desconocen cuál es la productividad de las personas que piensan contratar, alguien puede querer hacerse pasar por una persona que posee una alta productividad. Este ejemplo es denominado por la literatura como selección adversa \citep{Laffont_2002}. Una forma que tienen las firmas para lidiar con este problema es a través de la implementación de mecanismos que induzcan a las personas a revelar su información privada. Por lo general, estos mecanismos involucran el pago de incentivos económicos diferenciales a las personas de mayor productividad.



Con estos dos ejemplos se busca ilustrar el siguiente punto: en búsqueda de soluciones eficientes, la economía como disciplina propone el uso de incentivos económicos y remuneraciones diferenciales para personas con mayor productividad. Existe una corriente dentro de la economía que se deslinda de entrar en el terreno de los juicios normativos. Siguiendo con esta corriente, que se encuentra en autores como \citet{Friedman_1966}, la bondad de la teoría, es decir, de la parte \textit{positiva}, gira en torno a la capacidad de predecir fenómenos. Ocupa un lugar secundario preguntarse si los supuestos empleados resultan \textit{creíbles}. A su vez, se evita entrar en cuestionamientos normativos de los resultados que se derivan, solamente importa la faceta predictiva de la teoría.

Ahora bien, existe una corriente de pensadores dentro de la economía que defiende otro tipo de enfoque. En particular, \citet{Sen_1991} sostiene que la disciplina podría servirse de sus orígenes asociados con la política. Al hacer esto, se pondría en evidencia la naturaleza normativa de la economía y se la enriquece. A su vez, la propia división entre economía positiva y economía normativa ha sido ampliamente criticada. Siguiendo a \citet{Putnam_2004}, es posible argumentar que la distinción entre economía normativa y positiva se sustenta a partir de la dicotomía entre los conceptos de hecho y valor. El argumento central del autor es que, si bien es posible evidenciar una distinción entre los juicios normativos que pueden ser de carácter ético y otros tipos de juicios, esto no implica que exista una dicotomía a nivel metafísico entre los conceptos de hecho y valor. 

En contraposición a una dicotomía, \citet{Putnam_2004} plantea que existe un enredo (\textit{entanglement}) entre dichos conceptos. En el ámbito de la ciencia, es posible evidenciar este enredo si se reconoce que no todos los valores son valores éticos. En particular, existen valores epistémicos (coherencia, plausibilidad, simplicidad, entre otros) que son empleados a la hora de elegir entre hipótesis y teorías que compiten por ser las explicaciones predilectas. 

En la misma linea, \citet{Davis_2022} argumenta que la disciplina económica debe entenderse como normativa. A diferencia de lo que algunas corrientes de economistas creen, dentro de la teoría económica \textit{mainstream} se encuentra un ideal normativo que es central. Concretamente, dicho ideal consiste en la realización del individuo. A su vez, \citet{Davis_2022} sostiene que la teoría económica \textit{mainstream} entiende que la manera de realizar este ideal es mediante la satisfacción de las preferencias de los individuos. 

Habiendo evidenciado la faceta normativa de la economía como disciplina, es posible preguntarse: ¿cómo pueden ser vistas estas propuestas de remuneraciones diferenciales según la productividad de las personas desde una teoría de la justicia? ¿Existe una tensión entre la idea de justicia distributiva y los incentivos económicos diferenciales? A la hora de brindar una respuesta a estas preguntas, es necesario problematizar cómo es que las personas llegan a desarrollar estas diferentes productividades. En dicha discusión entran en juego: talentos, suerte, esfuerzo, entre otras variables. A su vez, se vuelve necesario cuestionar un supuesto empleado en algunas situaciones dentro de la ciencia económica, a saber, los salarios son iguales a la productividad. Una visión más realista sugeriría que la determinación de los salarios gira en torno a variables más allá de la productividad como puede ser: suerte, \textit{networking}, sexo, educación, antigüedad, entre otras variables. En la misma línea, existe numerable evidencia de que las instituciones que rigen el mercado laboral influyen sobre las remuneraciones \citep{Blau_1999}.


Respecto de la pregunta sobre la relación entre los incentivos diferenciales y la justicia distributiva, es posible encontrar una respuesta en el trabajo de John Rawls (\citeyear{Rawls_1971}, \citeyear{Rawls_2002}). En particular, el autor argumenta que algunos factores que influyen sobre los talentos o habilidades de los individuos están fuera de su control. Bajo el concepto de lotería natural, \citet{Rawls_1971} alude a cómo algunas dotaciones de talentos o de recursos de los individuos les son dadas por el contexto en el cual nacen. En este sentido, emplear un sistema que remunera de manera diferente a los más talentosos iría en contra de un ideal de justicia distributiva.

De todas formas, \citet{Rawls_1971} sostiene que existe una manera en la cual se pueden compatibilizar a los incentivos diferenciales con un ideal de justicia distributiva. Concretamente, en un contexto en el que se garanticen un conjunto de libertades y una igualdad de oportunidades para todos los individuos, \citet{Rawls_1971} argumenta que las desigualdades que terminen mejorando la situación de los menos aventajados pueden ser permitidas bajo el principio de la diferencia.

Sin embargo, el filósofo Gerald Cohen (\citeyear{Cohen_2001}) ha elaborado una crítica al argumento presentado por Rawls. En particular, \citet{Cohen_2001} sostiene que se evidencia una contradicción dentro de la teoría rawlsiana ya que existen individuos que dicen estar motivados por una preocupación igualitaria pero a la vez exigen una compensación diferencial para emplear sus talentos de manera productiva. En este sentido, el autor argumenta que, además de fijarse en el diseño de las instituciones formales para que plasmen los ideales de justicia distributiva, resulta necesaria la existencia de un \textit{ethos} igualitarista que fomente ciertas conductas que tienden hacia la equidad. En particular, este \textit{ethos} lograría influir las conductas de los individuos que no son capturadas por las instituciones formales. En la misma línea, a la hora de presentar su preocupación por cuestiones de justicia distributiva, \citeauthor{Cohen_2014b} (\citeyear{Cohen_2014b}, \citeyear{Cohen_2014c}) ha propuesto una vuelta a los ideales socialistas de igualdad y comunidad. 

Un punto del planteo de \citet{Cohen_2014c} que cabe destacar es que el principal problema al que se enfrentan los teóricos del socialismo es un problema de diseño. Desde la teoría económica se puede extraer información sobre cómo diseñar unos arreglos institucionales que funcionen en base a la competencia. Sin embargo, para desarrollar una economía que funcione bajo otros valores, no existe en términos relativos la misma cantidad de conocimiento teórico. En este sentido, la propuesta de \citet{Cohen_2014c} puede ser vista como un desafío de diseño institucional. 

Ante este problema de diseño, el economista John Roemer (\citeyear{Roemer_2019}, \citeyear{Roemer_2021a}) propone que una posible solución es trabajar sobre el concepto de cooperación. El autor comparte la intuición de \citeauthor{Cohen_2001} (\citeyear{Cohen_2001}, \citeyear{Cohen_2014c}) respecto de los ideales socialistas y la necesidad de un \textit{ethos} igualitarista. Con este objetivo en mente, \citet{Roemer_2019} desarrolla el concepto de \textit{optimización kantiana}. A través de este concepto, el autor busca modelizar de manera alternativa el comportamiento cooperativo y en base al mismo, construir modelos económicos denominados como socialismo de mercado.

Teniendo en cuenta los puntos esbozados anteriormente, el objetivo del siguiente trabajo es realizar un recorrido a través de esta línea de investigación respecto de la pregunta de los incentivos. Como forma de acotar el problema, la discusión se centra en los incentivos económicos. A su vez, se omite la discusión de los casos en donde existen asimetrías de información. La idea es hacer énfasis en cómo los talentos y habilidades de las personas llevan a distintos resultados en la esfera económica. 

El trabajo se divide de la siguiente manera. En la sección \ref{sec2} se introduce el concepto de justicia distributiva. Este concepto normativo es la pieza fundamental de este trabajo. A continuación, se presenta una evaluación o crítica normativa de la teoría neoclásica respecto de las consecuencias que se siguen de la misma. Concretamente, una crítica a las desigualdades que se podrían general al emplear los incentivos económicos. Dicha crítica busca evidenciar la necesidad de brindar una respuesta alternativa. Por lo cual, se presenta la idea del principio de la diferencia propuesto por el filósofo John Rawls. Si bien este concepto implica un tratamiento normativo de mayor alcance que la teoría neoclásica, no está exento de críticas. En este sentido, en la sección \ref{sec3}, se analiza la crítica realizada por el filósofo Gerald A. Cohen a ciertas implicancias del principio de la diferencia. En la misma línea, se presenta la propuesta de Cohen respecto de la igualdad de oportunidades socialista y la necesidad de un \textit{ethos} igualitarista. Dicha propuesta es recogida por el economista John E. Roemer, por lo cual, en la sección \ref{sec4} se presenta un intento de modelización económica de la misma. Finalmente, en la sección \ref{sec5} se presentan un par de reflexiones que buscan dar cuenta de las limitaciones y virtudes de la línea de investigación que se recorre en este trabajo.


\section{Justicia distributiva} \label{sec2}

El concepto de justicia distributiva es central para el desarrollo de este trabajo ya que es la idea normativa sobre la cual se evaluan las diversas propuestas que se presentan como respuestas al problema de si es posible conciliar incentivos e igualdad. La discusión sobre qué implica la justicia se encuentra presente a lo largo de toda la historia de la filosofía. Sin embargo, en el presente trabajo, se discute a partir de los desarrollos contemporáneos.


\subsection{Nociones básicas} \label{sec2.1}

A la hora de abordar cuestiones de justicia distributiva, existe una concepción que se encuentra ampliamente generalizada. La misma sostiene que los arreglos institucionales que conforman las sociedades implican una distribución de cargas y beneficios entre los miembros de dicha sociedad \citep{Rawls_1971}. La forma que tenga esta distribución es relevante ya que tiene un impacto sobre la vida de las personas. En este sentido, indagar sobre cuestiones de justicia distributiva puede implicar la búsqueda de los principios más adecuados para regular las instituciones encargadas de la distribución de las cargas y beneficios. 

Existen numerables visiones sobre cuáles han de ser los principios rectores. Es posible evidenciar principios que surgen de las siguientes corrientes filosóficas: un igualitarismo estricto, el principio de la diferencia (y el resto de los principios de la teoría rawlsiana), la igualdad de oportunidades, el igualitarismo de la suerte, principios basados en el bienestar de las personas, principios basados en el mérito, principios libertarios, entre otros. A su vez, las cuestiones de justicia distributiva pueden plantearse en distintos ámbitos: a nivel de sociedades individuales o a nivel de relaciones entre diversos países, a nivel de las relación entre las distintas generaciones, a nivel de aspectos asociados con las conductas individuales y entender a la justicia como una virtud, entre otros enfoques.

Este trabajo se centra en la discusión respecto de los principios de justicia que han de regir a una sociedad sin entrar en cómo las distintas sociedades interactúan entre sí. Como punto de partida, \citet[p. 81]{Rawls_2002} argumenta que el problema de la justicia distributiva consiste en responder a la pregunta de: \say{¿cómo han de regularse las instituciones de la estructura básica como un esquema unificado de instituciones para que pueda mantenerse a lo largo del tiempo, de una generación a otra, un sistema equitativo, eficiente y productivo de cooperación social?} Rawls enfatiza en la distinción entre los conceptos de \textit{justicia distributiva} y \textit{justicia asignativa}. La justicia asignativa se centra en la repartición de un conjunto de bienes entre diversas personas; mientras que la justicia distributiva implica más que la asignación debido a que tiene en cuenta, entre otras cosas, al proceso de cooperación social que produce los bienes materiales. 

Esta forma de concebir a la justicia ha sido denominada como el \say{paradigma distributivo de la justicia} \citep{Gosepath_2013}. Según Gosepath, desde el paradigma distributivo se entiende que:

\vspace{3mm}
\begin{quote}
    [...] toda concepción de la justicia debe ser capaz de dar una respuesta fundada a la pregunta acerca de quién debe qué a quién, bajo qué circunstancias, de qué manera, por qué, desde qué perspectiva, según qué principio y con qué modo de aplicación. [...] Según el paradigma distributivo, la tarea material de la justicia es asegurar una distribución equitativa de aquellos bienes que proveen a todos los miembros de la sociedad del estatus de personas libres e iguales, y que dentro de este marco les permiten perseguir sus proyectos elegidos de manera autónoma, siempre y cuando estos proyectos no interfieran con la misma libertad de las demás personas \citep[p. 48]{Gosepath_2013}.
\end{quote}
\vspace{3mm}

Sin embargo, existen diversas críticas a esta visión de la justicia distributiva ya que se entiende que este enfoque deja por fuera cuestiones relevantes. En particular, \citet{Young_2000} argumenta que este paradigma distributivo ignora cómo los bienes son producidos y cómo se determinan las estructuras políticas que influyen sobre su distribución. Adicionalmente, \citet{Forst_2013} sostiene que existen dos imágenes de la justicia y que el paradigma distributivo se ha centrado principalmente en una de ellas. Según el autor, esta imagen no permite evidenciar que los reclamos sobre la posesión de bienes se producen discursivamente en un contexto adecuado de justificiación. En la misma línea, esta visión ignora cuestiones asociadas a la injusticia ya que no permite distinguir entre distintas situaciones de carencias de bienes y los fundamentos morales detrás de las razones por las cuales es relevante hacer algo al respecto de dicha privaciones. Por lo cual, se plantea que:

\vspace{3mm}
\begin{quote}
    Por estas razones, al tratar con cuestiones de justicia distributiva resulta especialmente necesario tener en cuenta el aspecto \textit{político} de la justicia y librarnos de la imagen falsa, centrada únicamente en cantidades de bienes por más importantes que estos sean. En contraste, en una segunda imagen, a mi entender más apropiada, la justicia debe orientarse a \textit{relaciones y estructuras intersubjetivas}, no a \textit{estados subjetivos o putativamente objetivos} de la provisión de bienes \citep[p. 31]{Forst_2013}.
\end{quote}
\vspace{3mm}

Según \citet[p. 49]{Gosepath_2013}, es posible resumir las críticas a esta forma de concebir a la justicia de la siguiente manera: 1) el paradigma distributivo deriva en una reificación de las relaciones sociales y de las normas sociales; 2) el paradigma distributivo concibe a la justicia exclusivamente en pautas de distribución no considerando seriamente a los procesos sociales subyacentes y 3) como consecuencia de lo anterior, esta visión terminaría ignorando los procesos de producción de los bienes a distribuir.

Sin embargo, para el autor es posible defender esta visión de las críticas presentadas. Para responder a estas críticas, \citet{Gosepath_2013} argumenta que es necesario evidenciar tres polémicas sobre: la concepción de los bienes, la concepción de las relaciones sociales y el estatus de la justicia política. En primer lugar, respecto de la concepción de los bienes, \citet{Gosepath_2013} sostiene que es posible distinguir entre bienes estructurales e individuales. Los bienes estructurales son los prerrequisitos institucionales de la vida humana, tales como un marco constitucional en el cual desarrollar los individuos pueden desarrollar sus planes vitales. Mientras que los bienes individuales pueden ser internos (talentos, aptitudes y salud) o externos (alimento, vestimenta y vivienda). Esta conceptualización de los bienes permite evidenciar que los derechos, libertades y oportunidades pueden ser concebidos como bienes en un sentido amplio. Por lo cual,

\vspace{3mm}
\begin{quote}
    Es verdad que uno no \say{posee} derechos y oportunidades en exactamente el mismo sentido en que uno es propietario de bienes materiales; sin embargo, incluso la propiedad, o mejor dicho la posesión, debería ser (también) entendida como un paquete de derechos morales. Esto significa que tanto en el caso de bienes materiales como cuando se trata de derechos y oportunidades, regulamos las pretensiones morales, es decir, hablamos de asignaciones que detallan qué bienes, derechos y oportunidades podemos esperar legítimamente y cómo podemos hacer uso de ellos \citep[p. 51]{Gosepath_2013}.
\end{quote}
\vspace{3mm}

En segundo lugar, respecto de la polémica asociada a la forma de concebir a las relacionales sociales, se argumenta que al expandir el concepto de justicia distributiva hacia bienes no materiales, estos objetos serían tratados como objetos estáticos más que relaciones y procesos sociales. Ante esto, \citet{Gosepath_2013} plantea que dicha objeción sería plausible solamente cuando las concepciones de justicia se centren exclusivamente en el resultado final de la distribución. El autor sostiene que existen razones morales para rechazar dicho tipo de enfoque. En particular, si se consideran las acciones individuales, además del resultado final, también es relevante tener en cuenta el proceso según el cual se logran dichos resultados.

Finalmente, respecto del status de la justicia, \citet{Gosepath_2013} propone que es necesario preguntarse por la justicia de los procesos sociales en sí mismos. Este tipo de cuestionamiento se encuentra vinculado con la tradición del contrato social, en especial, en la propuesta de \citeauthor{Rawls_1971} (\citeyear{Rawls_1971}, \citeyear{Rawls_2002}). Bajo la teoría rawlsiana, ser miembro de una sociedad consiste en un conjunto de derechos y deberes que los ciudadanos libremente se conceden unos a otros. En este sentido, \citet{Gosepath_2013} argumenta que el establecimiento de una sociedad justa es una cuestión de justicia distributiva. Cuando los individuos libres e iguales reflexionan sobre cuál es la mejor forma que debe poseer la estructura básica de la sociedad, se negocian sus cargas y beneficios de la cooperación social. Esta negociación alude a una \say{distribución original} o \say{estado original}.

Como propuesta alternativa a este tipo de teoría procedimentalista, es posible presentar el planteo de \citet{Honneth_2013}. Según este autor, una propuesta reconstructivista de la justicia captura de mejor manera las relaciones intersubjetivas y las relaciones de reconocimiento que se desarrollan a lo largo de la historia. Sin embargo, \citet{Gosepath_2013} sostiene que una teoría reconstructivista no puede establecer un punto de vista imparcial sobre el cual fundamentar los principios de justicia. En particular,

\vspace{3mm}
\begin{quote}
    [...] una teoría \say{reconstructiva} de la justicia de este tipo carece de cualquier tipo de criterio independiente que nos permita evaluar los estándares normativos inmanentes de una práctica histórica contingente. Para establecer tales criterios, necesariamente debemos recurrir a una teoría abstracta e ideal de justicia que funcione con supuestos hipotéticos, incluso cuando una teoría política y crítica de justicia comprensiblemente considere tales abstracciones, idealizaciones e hipostatizaciones como insatisfactorias \citep[p. 55]{Gosepath_2013}.
\end{quote}
\vspace{3mm}

A modo de síntesis, el punto central de \citet{Gosepath_2013} es que la distribución justa no se limita a los bienes sino que puede incluir cosas de vital importancia como pueden ser los derechos, deberes y oportunidades. En esencia, la intuición básica de la idea de justicia es asegurar que ningún individuo se vea excesivamente desfavorecido.

Habiendo definido algunas cuestiones básicas acerca del concepto de justicia distributiva, a continuación se desarrolla una forma de concebir el vínculo entre la filosofía normativa y la teoría económica.

\subsection{Justicia y economía} \label{sec2.2}

Siguiendo a \citet{Sen_1991}, desde sus orígenes la economía se ha vinculado con cuestiones normativas. Concretamente, el autor evidencia dos orígenes de la disciplina con enfoques diferenciados. En primer lugar, el autor argumenta que en los trabajos de \citeauthor{Aristoteles_2005} (\citeyear{Aristoteles_2005}, \citeyear{Aristoteles_2014}) se puede evidenciar una relación estrecha entre la política y la economía. En esta relación se destaca la faceta normativa de la economía. Por otro lado, \citet{Sen_1991} plantea que existe otro enfoque que denomina que asocia a la economía con la ingeniería. En este caso, la economía toma a los fines como algo dado y se cuestiona respecto de la adecuación de los medios materiales para obtener dichos fines. Según \citet{Sen_1991}, si bien conceptualizar a la economía de esta manera ha sido muy productivo en términos de resultados, este enfoque ha sido predominante en detrimento del enfoque normativo. En este sentido, el autor argumenta que la disciplina económica se enriquecería si su faceta normativa es puesta en un primer plano.

Existe un vínculo especial entre los economistas y filósofos a la hora de abordar cuestiones sobre justicia distributiva. Concretamente, los practicantes de las disciplinas mencionadas, proceden de distintas maneras. Según \citet{Roemer_1996a}, muchos economistas abandonan incursionar en cuestiones de filosofía política al encontrarla muy informal o carente de rigor para poder ser manejada. La práctica de la disciplina ha llevado a los economistas a tomar preguntas interesantes y buscar su formalización a través de modelos. Por otro lado, es cierto que los filósofos apreciarían la contribución de los modelos formales. Sin embargo, su interés giraría en torno al proceso intelectual que ha llevado a formularlo. Roemer sostiene que un filósofo se enfocaría en cuestionar los supuestos empleados cuando se encuentra ante algún modelo económico.  

Ahora bien, cuando estos dos enfoques se encuentran al abordar la temática de la justicia distributiva, Roemer argumenta que la relación y el aporte que puede hacer el economista al filósofo tiene la particularidad de que:
\vspace{3mm}

\begin{quote}
    [...] la economía es la sirvienta en esta relación. La forma de pensar del economista puede comprobar la consistencia de una teoría filosófica o proporcionar una formulación concreta (un modelo) para hacer precisa alguna de sus todavía vagas aseveraciones. A menudo puede traducir una visión filosófica sobre la justicia distributiva en una política social concreta [...] Sin embargo, no creo que la forma de pensar del economista haya producido, o producirá alguna vez, nuevos conocimientos importantes sobre lo que es la justicia distributiva\footnote{A lo largo de este trabajo, las traducciones de citas en inglés son propias: \say{[...] economics is the handmaiden in this relationship. The economist's way of thinking can check the consistency of a philosophical theory or provide a concrete formulation (a model) to make precise some of its still vague assertions. It can often translate a philosophical view about distributive justice into a concrete social policy [...] I do not, however, believe that the economist's way of thinking has produced or will ever produce, important new insights into what distributive justice is} \citep[p. 3]{Roemer_1996a}.} \citep[p. 3]{Roemer_1996a}.
\end{quote}
\vspace{3mm}


A la hora de evaluar los postulados de la economía, en este trabajo se hace énfasis en la teoría neoclásica. Cabe destacar que dicha teoría está compuesta por un conjunto de modelos. Existe un aforismo dentro de la disciplina económica y la estadística que dice que \say{todos los modelos están equivocados pero algunos son útiles}\footnote{Aforismo atribuido al estadístico Box, es posible encontrarlo en varios de sus trabajos. En particular, en \citet[p. 202]{Box_1979} una sección de este trabajo se titula: \say{All models are wrong, but some are useful.}}, es decir, los modelos siempre implican una simplificación de la realidad, pero los mismos pueden ser útiles para predecir algunos fenómenos. Esta aclaración sugiere que esta teoría no pretende determinar exhaustivamente los factores que explican las remuneraciones sino establecer un esquema general sobre el cual hacer predicciones. 

Un posible argumento que podría esbozar un partidario de la teoría neoclásica es que la misma no discute la distribución de las remuneraciones ya que, en la situación ideal que plantea el modelo, todos los factores reciben una remuneración en función de su aporte a la producción total. Esto es consecuencia de un postulado principal de dicha teoría. Concretamente, en una situación ideal en donde hay competencia e información perfecta, las remuneraciones de las personas se explican a través de la productividad de las mismas. En particular, \citet{Arrow_1954} emplean una demostración matemática de la existencia de una asignación de recursos que conforma un equilibrio general. En dicha asignación se cumple que: 1) las firmas maximizan sus beneficios, 2) los consumidores maximizan su utilidad y 3) todos los mercados se vacían. A su vez, en este equilibrio cada factor productivo es remunerado en función de su producción marginal.

Ahora bien, este argumento no cierra completamente la posibilidad de analizar las implicancias de esta teoría en términos de justicia distributiva. Es posible presentar dos críticas a este tipo de defensa. En primer lugar, \citet{Rawls_1971} plantea que una sociedad organizada de esta forma puede ser enmarcada en un sistema de libertad natural. Concretamente, bajo este esquema, se plantea que \say{una estructura básica que satisfaga el principio de eficiencia y en la que los puestos estén abiertos a quienes puedan y estén dispuestos a disputar por ellos conducirá a una distribución justa}\footnote{\say{[...] a basic structure satisfying the principle of efficiency and in which positions are open to those able and willing to strive for them will lead to a just distribution} \citep[p. 66]{Rawls_1971}.} \citep[p. 66]{Rawls_1971}.

De todas formas, el sistema de libertad natural posibilita que la distribución de ingreso y riqueza sea influenciada en buena medida por los resultados de la lotería natural. En particular, cuestiones que escapan al control de los individuos como pueden ser su dotación de talentos y el contexto social de nacimiento. En este sentido, \say{la injusticia más evidente del sistema de libertad natural es que permite que las partes que se distribuyen se vean indebidamente influidas por estos factores tan arbitrarios desde el punto de vista moral}\footnote{\say{[...] the most obvious injustice of the system of natural liberty is that it permits distributive shares to be improperly influenced by these factors so arbitrary from a moral point of view} \citep[p. 72]{Rawls_1971}.} \citep[p. 72]{Rawls_1971}. 

De manera similar, \citet{Roemer_1998} argumenta desde la perspectiva de la igualdad de oportunidades que existen factores que influyen sobre las capacidades de las personas que se encuentran por fuera del control de los individuos. Si no se tienen en cuenta las circunstancias en las que se encuentran situados los individuos, resulta evidente que si se aplicara el criterio distributivo que propone la teoría neoclásica se ignorarían cuestiones relevantes para la justicia distributiva. Si los salarios son iguales a la productividad, pero no se compensa a las personas por sus circunstancias arbitrarias, lo que ocurre es que las personas que salieron más beneficiadas de la lotería natural terminarían con remuneraciones elevadas por cuestiones que estaban fuera de su control, es decir, dichos niveles no tendrían una justificación en el esfuerzo o merecimiento.


En segundo lugar, aunque en el modelo de neoclásico se sostenga que cada factor productivo recibe una remuneración acorde a su contribución en el proceso productivo, por lo general existe un excedente que es apropiado por los dueños de las empresas. \citet{Roemer_2021a} argumenta que este ingreso no se explica a través aporte de un factor productivo, sino que se debe a que los empresarios poseen los derechos de propiedad sobre lo que se produce. Por lo cual, es posible establecer que:

\vspace{3mm}
\begin{quote}
    La equidad de esta asignación es cuestionable. ¿No es discutible que los trabajadores y los inversores deberían compartir el excedente que emerge en la producción? La estructura legal del capitalismo asigna los beneficios a los propietarios, pero esto no es necesariamente justo o ético. Es una tradición en la teoría neoclásica decir que los trabajadores no son explotados si reciben salarios iguales a sus productos (valores) marginales. Los marxistas, sin embargo, dicen que los trabajadores que reciben salarios iguales al producto marginal \textit{son} explotados porque no comparten el excedente de la producción\footnote{\say{The fairness of this allocation is questionable. For it is not arguable that workers and investors should share in the surplus that emerges in production? The legal structure of capitalism allocates profits to owners, but that is not necessarily fair or ethical. It is a tradition in neoclassical theory to say that workers are not exploited if they receive wages equal to their marginal (value) products. Marxists, however, say that workers who receive marginal-product wages \textit{are} exploited because they do not share in the surplus from production} \citep[p. 266]{Roemer_2021a}.} \citep[p. 266]{Roemer_2021a}.
\end{quote}
\vspace{3mm}


El esquema desarrollado por la teoría neoclásica se centra primordialmente en el aspecto de la competencia entre las personas y predice que las personas más productivas o talentosas serán las que recibirán las mayores recompensas. El hecho de que algunas formulaciones de la teoría neoclásica no problematizan cómo los individuos llegan a desarrollar sus capacidades y cómo se distribuye el excedente del proceso productivo, es la razón por la cual se vuelve necesario presentar un esquema conceptual alternativo. A continuación, se presenta un esbozo de la teoría rawlsiana como alternativa que busca articular las nociones de eficiencia, incentivos y justicia distributiva. En particular, se esboza un concepción en donde la idea de cooperación social tiene un papel más relevante que la idea de competencia o de simple coordinación entre agentes.


\subsection{Una teoría de la justicia} \label{sec2.3}

El trabajo de \citeauthor{Rawls_1971} (\citeyear{Rawls_1971}, \citeyear{Rawls_2002}) ha sido influyente tanto en el campo de la filosofía normativa como en el campo de las ciencias sociales y dicho impacto ha continuado hasta la actualidad\footnote{Para desarrollar la propuesta del autor se sigue la obra de \textit{A Theory of Justice} (\citeyear{Rawls_1971})  y \textit{La justicia como equidad} (\citeyear{Rawls_2002}).}. Siguiendo a \citeauthor{Rawls_1971} (\citeyear{Rawls_1971}, \citeyear{Rawls_2002}), la justicia es la virtud principal de las instituciones sociales. Para el autor, una sociedad es justa cuando los derechos de las personas no se encuentran sujetos a una negociación política o sujetos a una maximización de utilidad social. En este sentido, el propósito de Rawls es elaborar una teoría de la justicia que logre incorporar las convicciones de las personas respecto de la primacía de la justicia. Para este objetivo, se requieren de principios que determinen cuáles instituciones sociales se prefieren por sobre otras a la hora de determinar la distribución de los recursos. Concretamente, \say{estos principios son los principios de la justicia social: proporcionan una forma de asignar derechos y deberes en las instituciones básicas de la sociedad y definen la distribución adecuada de los beneficios y cargas de la cooperación social}\footnote{\say{These principles are the principles of social justice: they provide a way of assigning rights and duties in the basic institutions of society and they define the appropiate distribution of the benefits and burdens of social cooperation} \citep[p. 4]{Rawls_1971}.} \citep[p. 4]{Rawls_1971}.

Estos principios se enmarcan en sociedades que son concebidas como \say{un sistema equitativo de cooperación social a lo largo del tiempo de una generación a la siguiente} \citep[p. 28]{Rawls_2002}. Esta idea de cooperación social tiene tres rasgos esenciales: a) está guiada por reglas y procedimientos públicamente reconocidos, b) incluye términos equitativos de la cooperación asociados a una idea de reciprocidad y c) los participantes que cooperan promueven su propio bien. Este esquema de cooperación se conforma por participantes que son considerados razonables y racionales. Por un lado, según \cite{Rawls_2002}, las personas razonables están dispuestas a cumplir con los principios necesarios para la cooperación equitativa siempre que los demás estén igualmente dispuestos a honrarlos. Por otro lado, las personas racionales podrían tener motivos basados en el interés personal para no cumplir con dichos principios y obtener una ventaja. 

Una de las razones por las cuales se conciben a las sociedades ideales de esta forma se debe a la pluralidad de concepciones del bien que poseen los ciudadanos. Dado que no es posible llegar a un acuerdo sobre una única visión comprehensiva del mundo, es decir, un acuerdo sobre una idea del bien, los ciudadanos deben llegar a un acuerdo sobre una concepción política de la justicia que regirá la interacción social (\citeauthor{Rawls_1971}, \citeyear{Rawls_1971}, \citeyear{Rawls_2002}). En este acuerdo, se llega a un pluralismo razonable en donde las concepciones del bien que se mantienen en la sociedad no vulneran los derechos de ninguna persona. 

Al concebir a la sociedad como un sistema de cooperación, se presenta la idea de las sociedades bien ordenadas, es decir, sociedades que se encuentran efectivamente reguladas por una concepción pública de la justicia. Este concepto implica una idealización que presenta tres grandes rasgos \citep{Rawls_2002}. En primer lugar, cada persona acepta y sabe que los demás aceptan la misma concepción política de la justicia. En segundo lugar, los ciudadanos tienen buenas razones para creer que las principales instituciones políticas y sociales satisfacen dicha concepción de justicia. Finalmente, en una sociedad bien ordenada, los ciudadanos poseen un sentido de la justicia que \say{los capacita para entender y aplicar los principios públicamente reconocidos de justicia y, en su mayor parte, para actuar según lo exige su posición en la sociedad, con sus deberes y obligaciones} \citep[p. 32]{Rawls_2002}.


Ahora bien, cabe precisar el ámbito para el cual estan diseñados los principios de justicia. En particular, \citeauthor{Rawls_1971} (\citeyear{Rawls_1971}, \citeyear{Rawls_2002}) argumenta que el sujeto primario de la justicia es la estructura básica de la sociedad, es decir, aquellas instituciones de mayor relevancia social que definen la manera en que se distribuyen las cargas y beneficios de la cooperación social. La teoría rawlsiana se enfoca principalmente en este ámbito porque la estructura básica tiene efectos de vital importancia sobre los objetivos, aspiraciones y oportunidades de los ciudadanos. En la misma línea, \citet{Rawls_1971} sostiene que esta discusión sobre los principios que han de regir sobre la estructura básica se vincula con la tradición política del contrato social que se evidencia en autores como Locke, Rousseau y Kant\footnote{En particular, es posible destacar el trabajo de \citet{Locke_1960}, \citet{Rousseau_2007} y \citet{Kant_2012}.}. En este contexto, no se trata de definir un estado de naturaleza sobre el cual se genera un acuerdo original para establecer un gobierno, sino que:

\vspace{3mm}
\begin{quote}
    [...] los principios de justicia para la estructura básica de la sociedad son el objeto del acuerdo original. Son los principios que personas libres y racionales preocupadas por sus propios intereses aceptarían en una posición inicial de igualdad como definidores de los términos fundamentales de su asociación\footnote{\say{[...] the principles of justice for the basic structure of society are the object of the original agreement. They are the principles that free and rational persons concerned to further their own interests would accept in an initial position of equality as defining the fundamental terms of their association} \citep[p. 11]{Rawls_1971}.} \citep[p. 11]{Rawls_1971}.
\end{quote}
\vspace{3mm}

Para dar cuenta de este acuerdo original, \citeauthor{Rawls_1971} (\citeyear{Rawls_1971}, \citeyear{Rawls_2002}) desarrolla el concepto de la posición original. Bajo este concepto se busca construir una forma de deliberación para acordar los principios de justicia que van a regir a la estructura básica de la sociedad. Esta deliberación se realiza bajo condiciones que  \say{deben situar equitativamente a las personas libres e iguales y no deben permitir que algunos puedan negociar con los demás desde posiciones no equitativas de ventaja} \citep[p. 39]{Rawls_2002}. La posición original es una experimento mental, es decir, una situación imaginaria en donde todos los ciudadanos reales poseen un representante ideal y estos representantes llegan a un acuerdo sobre los principios de justicia que han de regir a los ciudadanos reales. Para mantener las condiciones equitativas, los representantes se encuentran bajo un velo de ignorancia que no les permite conocer características arbitrarias de las persona a las cuales se encuentran representando.


La teoría rawlsiana concibe a los ciudadanos como personas que poseen dos \say{facultades morales}:

\vspace{3mm}
\begin{quote}
    i) Una de esas facultades es la capacidad de poseer un sentido de la justicia: es la capacidad de entender, aplicar y obrar según (y no sólo de conformidad con) los principios de la justicia política que definen los términos equitativos de la cooperación social.
    
    ii) La otra facultad moral es la capacidad de poseer una concepción del bien: es la capacidad de poseer, revisar y perseguir racionalmente una concepción del bien \citep[p. 43]{Rawls_2002}.
\end{quote}
\vspace{3mm}

A través de estas facultades morales es posible establecer cómo es que los ciudadanos se conciben como personas libres e iguales. Por un lado, los ciudadanos se conciben como iguales debido a que todos se entienden poseedores de las facultades morales necesarias para participar en la cooperación social durante toda una vida. Por otro lado, los ciudadanos se conciben como libres por dos razones: 1) ya que reconocen que los demás tienen la facultad moral de poseer una concepción del bien y 2) se conciben como fuentes autoautentificatorias de exigencias válidas, es decir, \say{se ven a sí mismos con derecho a presentar exigencias a sus instituciones con ánimo de promover sus concepciones del bien} \citep[p. 48]{Rawls_2002}.

Mediante la deliberación en la posición original por representantes de ciudadanos libres e iguales, \citeauthor{Rawls_1971} (\citeyear{Rawls_1971}, \citeyear{Rawls_2002}) presenta dos principios de justicia que serían acordados como resultado del proceso. Se define a los dos principios de la siguiente manera:

\vspace{3mm}
\begin{quote}
   [...] a) cada persona tiene el mismo derecho irrevocable a un esquema plenamente adecuado de libertades básicas iguales que sea compatible con un esquema similar de libertades para todos; y
   
   b) las desigualdades sociales y económicas tienen que satisfacer dos condiciones: en primer lugar, tienen que estar vinculadas a cargos y posiciones abiertos a todos en condiciones de igualdad equitativa de oportunidades; y, en segundo lugar, las desigualdades deben redundar en un mayor beneficio de los miembros menos aventajados de la sociedad (el principio de diferencia) \citep[p. 73]{Rawls_2002}.
\end{quote}
\vspace{3mm}

Respecto de esta forma de concebir a los principios de justicia, en \textit{La justicia como equidad}, \citet{Rawls_2002} sostiene que se trata de una formulación revisada a la que se encuentra en \textit{Teoría de la justicia} §11. De todas formas, esta formulación revisada es similar a la que se encuentra en \textit{Teoría} §46. Asimismo, \citeauthor{Rawls_1971} (\citeyear{Rawls_1971}, \citeyear{Rawls_2002}) argumenta que el primer principio tiene una prioridad sobre el segundo principio. A su vez, dentro de las partes que componen al segundo principio de justicia, la igualdad equitativa de oportunidades (primera parte) tiene prioridad sobre el principio de la diferencia (segunda parte).

Estos principios tienen como objetivo distintas instancias de la estructura básica de la sociedad. Por un lado, el primer principio se enfoca en determinar la constitución política de las sociedades. Específicamente, este principio plantea que los ciudadanos tienen derecho a un conjunto de libertades básicas. En particular: 

\vspace{3mm}
\begin{quote}
[...] libertad de pensamiento y libertad de conciencia; libertades políticas (por ejemplo, el derecho de voto y el derecho a participar en política) y libertad de asociación, así como los derechos y libertades determinados por la libertad y la integridad (física y psicológica) de la persona; y finalmente, los derechos y libertades amparados por el imperio de la ley \citep[p. 75]{Rawls_2002}.
\end{quote}
\vspace{3mm}

Por otro lado, el ámbito del segundo principio es el diseño de las leyes y el diseño de las instituciones asociadas a la economía. Este principio de dos partes establece que, una vez que se haya cumplido con la igualdad equitativa de oportunidades para los ciudadanos, es posible establecer desigualdades económicas siempre y cuando las mismas favorezcan al grupo menos aventajado. A la hora de definir a este grupo, es necesario introducir el concepto de bienes primarios. Concretamente, 

\vspace{3mm}
\begin{quote}
    Estos bienes son las diversas condiciones sociales y los medios de uso universal que son por lo general necesarios para que los ciudadanos puedan desarrollarse adecuadamente y ejercer plenamente sus dos facultades morales, y para que puedan promover sus concepciones específicas del bien \citep[p. 90]{Rawls_2002}.
\end{quote}
\vspace{3mm}

Específicamente, el autor distingue cinco clases de bienes primarios: a) los derechos y libertades básicas, b) la libertad de movimiento y la libre elección de empleo, c) los poderes y prerrogativas que acompañan a cargos y posiciones, d) ingresos y riquezas y e) las bases sociales del autorrespeto. Bajo este último ítem, se enmarcan las instituciones básicas \say{esenciales si los ciudadanos han de tener clara conciencia de su valor como personas y han de ser capaces de promover sus fines con autoconfianza} \citep[p. 92]{Rawls_2002}. Según \citeauthor{Rawls_1971} (\citeyear{Rawls_1971}, \citeyear{Rawls_2002}), si nos situamos en una sociedad bien ordenada en donde se garantizan derechos y libertades a los ciudadanos, los menos aventajados son aquellos que integran el grupo de ingresos con las expectativas más bajas.

Volviendo sobre el principio de la diferencia, cabe recordar que el mismo debe ser entendido como un principio de justicia distributiva. En particular, este principio regula cómo han de diseñarse instituciones económicas que permiten la cooperación social para la producción de bienes materiales. La idea de permitir ciertas desigualdades es que las personas más productivas o talentosas consideran estas desigualdades como:

\vspace{3mm}
\begin{quote}
   [...] incentivos para que el proceso económico sea más eficiente, la innovación avance a un ritmo más rápido, etc. Eventualmente, los beneficios materiales resultantes se extienden por todo el sistema y hacia los menos aventajados. No consideraré hasta qué punto estas cosas son ciertas. El punto es que se debe argumentar algo de este tipo si estas desigualdades van a ser solo por el principio de diferencia\footnote{\say{[...] incentives so that the economic process is more efficient, innovation proceeds at a faster pace, and so on. Eventually the resulting material benefits spread throughout the system and to the least advantaged. I shall not consider how far these things are true. The point is that something of this kind must be argued if these inequalities are to be just by the difference principle} \citep[p. 71]{Rawls_1971}.} \citep[p. 71]{Rawls_1971}.
\end{quote}
\vspace{3mm}

En este sentido, \citet{Rawls_2002} argumenta que el principio de la diferencia invoca una idea de reciprocidad. Concretamente, a las personas más aventajadas se las incentiva a buscar beneficios mayores a través del cultivo de sus talentos y este desarrollo termina favoreciendo a los menos aventajados. La idea de reciprocidad que se encuentra en el principio de la diferencia brinda un argumento para alejarse de la distribución igualitaria estricta ya que \say{selecciona un punto focal natural entre las demandas de la eficiencia y la igualdad} \citep[p. 169]{Rawls_2002} En la misma línea, la noción de reciprocidad \say{está implícita en la idea de que ha de considerarse como un activo común la distribución de las dotaciones innatas} \citep[p. 170]{Rawls_2002}. 

A la hora de evidenciar arreglos institucionales que podrían cumplir con los dos principios de justicia, \citet{Rawls_2002} presenta la idea de la democracia de propietarios. Estos arreglos institucionales son contrastados con un capitalismo de Estado de bienestar, destacando que:

\vspace{3mm}
\begin{quote}
    [...] las instituciones de trasfondo de la democracia de propietarios contribuyen a dispersar la propiedad de la riqueza y el capital, con lo que impiden que una pequeña parte de la sociedad controle la economía y asimismo, indirectamente, la vida política. Por el contrario, el capitalismo del Estado de bienestar permite que una pequeña clase tenga un cuasimonopolio de los medios de producción \citep[p. 189]{Rawls_2002}.
\end{quote}
\vspace{3mm}

Por esta razón, un capitalismo de Estado de bienestar no logra satisfacer los principios de justicia. Específicamente, \citet{Rawls_2002} sostiene que en un capitalismo de Estado de bienestar se busca que ningún ciudadano caiga por debajo de un mínimo nivel de vida decente. Para lograr este objetivo, se establecen protecciones al empleo y a la salud. Adicionalmente, se emplea la redistribución de los ingresos a través de esquemas impositivos que pueden ser progresivos. En contraposición, en una democracia de propietarios el objetivo es que las instituciones logren operacionalizar la idea de la sociedad como un sistema equitativo de cooperación. Con este objetivo en mente, los ciudadanos deben ser provistos de los medios necesarios, tanto el capital humano y capital productivo, para ser miembros cooperantes de la sociedad durante toda una vida. En este contexto, los menos aventajados:

\vspace{3mm}
\begin{quote}
    [...] no son los desafortunados y desventurados —objetos de nuestra caridad y compasión, aunque mucho menos de nuestra piedad— sino aquellos a los que se debe reciprocidad en nombre de la justicia política, justicia política para una totalidad de ciudadanos libres e iguales. Aunque controlen menos recursos, cumplen su parte plenamente de un modo que es reconocido por todos como mutuamente ventajoso y consistente con el autorrespeto de cada cual \citep[p. 190]{Rawls_2002}.
\end{quote}
\vspace{3mm}

En síntesis, el principio de la diferencia brinda una respuesta al problema de los incentivos diferenciales: estos incentivos pueden ser compatibles con un esquema de justicia distributiva siempre y cuando se enmarquen en un contexto de igualdad de oportunidades y terminen siendo favorables para los menos aventajados. A su vez, la democracia de propietarios involucra unos arreglos institucionales diseñados para distribuir los bienes y mantener la igualdad entre los ciudadanos.

Ahora bien, aunque Rawls tiene presente la idea de cooperación tanto a la hora de conceptualizar a las sociedades como a la hora conceptualizar a los ciudadanos como racionales y razonables, parece persistir un cierto punto de la teoría neoclásica. En particular, si bien el principio de la diferencia se encuentra limitado por los demás principios de justicia, el mismo parece permitir ciertos aspectos de la competencia o características de las personas concebidas como egoístas racionales. Este punto será retomado \citet{Cohen_2001} ya que cuando se consideran las acciones de los ciudadanos, los principios de justicia no rigen en dicho ámbito generando una tensión en esta idea de cooperación.
 



\section{La crítica de Cohen} \label{sec3}

La idea de que los incentivos diferenciales son justos cuando terminan beneficiando a los peores situados, no está exenta de polémica. Un autor que discrepa con el argumento de Rawls analizado anteriormente es el filósofo Gerald Cohen. 

\subsection{¿Por qué la igualdad?}

Para enmarcar el pensamiento de Cohen, se distinguen dos etapas. En primer lugar, una etapa en donde su interés radica en analizar la teoría marxista desde una perspectiva diferente. Al emplear su formación en la tradición analítica, Cohen perteneció a una corriente denominada como \say{marxismo analítico}; la obra fundamental de esta etapa es \textit{Karl Marx's Theory of History: A Defence} (\citeyear{Cohen_1978}). En un segundo momento, el autor se embarca en el estudio de la filosofía política normativa haciendo énfasis en cuestiones de justicia distributiva. Este movimiento en el campo de interés podría ser considerado poco fructícero desde la corriente marxista tradicional.

Ahora bien, \citet{Cohen_2001} argumenta que una corriente socialista debería tener un interés sobre la filosofía normativa ya que resulta difícil sostener dos hipótesis defendidas desde el marxismo tradicional respecto del desarrollo histórico, a saber: 1) el ascenso de la clase trabajadora organizada, que garantizaría la igualdad y 2) el desarrollo de las fuerzas productivas que generaría una abundancia de bienes materiales. Estos postulados eran presentados como tendencias irreversibles del devenir histórico, por lo cual carecía de sentido un argumento desde el punto de vista de la igualdad en la distribución. 

Para el autor, la primera hipótesis resulta falsa ya que el proletariado se encuentra en un proceso de desintegración. Con el desarrollo del capitalismo, no se generó una situación en la que todos los trabajadores se encasillaron en la pobreza sino que la fuerza de trabajo se ha ido complejizando volviendo poco realista la idea de que todos los trabajadores se enfrentan a situaciones de escasez similares. Por otro lado, respecto de la segunda hipótesis, existe un límite impuesto por los recursos naturales existentes en el mundo. Este límite no permitiría un nivel de consumo de bienes para todas las personas similar al consumo de los países desarrollados.

En la misma línea, \citet{Cohen_2001} sostiene que para los marxistas ahondar en los fundamentos normativos detrás de su propuesta política no era relevante. En particular, ya que las personas que conformaban el proletariado contaban con las siguientes características: 1) constituían la mayoría de  la sociedad, 2) producían la riqueza de la sociedad, 3) eran los explotados de la sociedad y 4) eran los necesitados de la sociedad; dadas estas características, es posible establecer que 5) no tendrían nada que perder con la revolución y 6) podrían transformar y transformarían la sociedad. Teniendo en cuenta esta descripción, muchos principios normativos podrían alinearse con la lucha por mejores condiciones de vida para el proletariado, por lo cual no sería relevante indagar la temática.

Partiendo de la base de las dos hipótesis sobre las tendencias del devenir histórico y de la caracterización del proletariado, no se exigía la igualdad sino que se pensaba que era algo inevitable. Según Cohen, al abandonar este terreno del marxismo tradicional, surge un nuevo fundamento para exigir la igualdad vinculado con la crisis ecológica:

\vspace{3mm}
\begin{quote}
    [...] nuestro medio ambiente ya está severamente degradado y que, si hay alguna forma de salir de la crisis, esa forma ha de pasar por un menor consumo material del que ahora existe y, como resultado de ello, ha de pasar por cambios no deseados en el estilo de vida de cientos de millones de personas. [...] Es indudable que el consumo que realiza Occidente, \textit{medido en términos de uso de la energía combustible fósil y de recursos naturales}, en porcentaje debe reducirse drásticamente y que el consumo que realizan los países no occidentales, considerado en conjunto, nunca alcanzará los niveles actuales de Occidente, \textit{medidos de esta forma} \citep[p. 152-153]{Cohen_2001}.
\end{quote}
\vspace{3mm}

Al verse frustrada la posibilidad de una mejora ilimitada en la calidad de vida de las personas debido a las restricciones que imponen los recursos naturales, Cohen afirma que los niveles de desigualdad en términos de consumo y/o ingreso se vuelven mucho más intolerables desde un punto de vista moral. Ahora bien, cabe destacar la importancia del postulado de la abundancia que surgiría con el desarrollo de las fuerzas productivas. Según \citet{Cohen_2001}, respecto de esta proposición, existía un importante optimismo desde el marxismo tradicional. En particular, ya que al mismo tiempo se tenía una postura pesimista respecto del ordenamiento social en condiciones que no fueran de abundancia material. Concretamente, bajo las condiciones de escasez, se creía que la sociedad de clases era inevitable.

Para el autor, es necesario abandonar tanto el optimismo sobre la posibilidad de la abundancia material como también el pesimismo sobre el cambio social en condiciones de escasez. Esta doble renuncia implica que al aceptar la premisa de la escasez material hay pensar claramente respecto de \say{aquello que estamos buscando, qué razones tenemos para buscarlo y por qué medios institucionales puede realizarse} \citep[p. 156]{Cohen_2001}.

Pese al aparente desinterés del marxismo tradicional respecto de las cuestiones normativas, \citet{Cohen_2014a} cree que la justicia ocupa un lugar central en la creencia marxista revolucionaria. Para dar cuenta de esto, el autor establece una comparación entre la izquierda marxista y los socialdemócratas\footnote{Sobre este punto, en la Sección \ref{sec2.3}, Rawls desarrolla un contraste similar entre la democracia de pequeños propietarios y el capitalismo de Estado de bienestar. Los socialdemócratas que Cohen tiene en mente en este caso favorecían la consecución de un capitalismo de Estado de bienestar.}. A la hora de plantear objeciones morales a la economía de mercado, los socialdemócratas argumentan que en este tipo de sociedades, los más débiles padecen niveles de privación elevados que podrían ser mitigados mediante, por ejemplo, el establecimiento de un sistema de previsión social. Este tipo de crítica respecto de algunas consecuencias no deseadas de las economías de mercado no llega a ser una crítica respecto de la injusticia de las instituciones implicadas. Cohen argumenta que para que se invoquen cuestiones asociadas a la justicia, la crítica debería plantear que es por la operativa del mercado que algunas personas se ven privadas de \say{sus derechos sobre aquello que moralmente debería considerarse un bien común} (\citeyear[p. 40]{Cohen_2014a}). 

En la misma línea, para que los socialdemócratas esbocen una crítica a las consecuencias de la economía de mercado en el bienestar de las personas, se debería poner en tela de juicio a la causa de dichas consecuencias, es decir, a la economía de mercado en sí misma. Lo que sucede es que desde la perspectiva de la socialdemocracia, no se cuestiona dicha estructura institucional y se podría decir que se toman sus consecuencias como algo dado. Según Cohen, los defensores de esta corriente se abstienen de profundizar en este debate debido a que se entraría en un terreno más radical, el cual no quieren recorrer. El autor sugiere, de manera exagerada, que los socialdemócratas \say{parecen sensibles a los efectos de la explotación sobre las personas, pero no al fenómeno de la explotación en sí. Quieren socorrer a los que explotados y minimizar cualquier posible confrontación con aquellos que los explotan} (\citeyear[p. 43-44]{Cohen_2014a}). En contraposición a esta postura, Cohen sostiene que la crítica socialista al capitalismo se basa en la justicia de sus instituciones. En particular, dicha objeción plantea que la economía de mercado \say{permite la propiedad privada de medios de existencia que nadie tiene el derecho a poseer de manera privada, basándose por ende sobre un fundamento injusto} \citep[p. 41]{Cohen_2014a}.


Habiendo presentado el giro normativo de Cohen, a continuación se desarrolla la crítica realizada por el autor al principio de la diferencia desarrollado por Rawls. Esta crítica se centra en analizar el argumento por el cual se considerarían justos los incentivos diferenciales para motivar la contribución de las personas.


\subsection{La crítica al argumento de los incentivos}

\begin{quote}
-\textit{Mejor situado:} \say{Mire, conciudadano, trabajaré duro y mejoraré tanto a usted como a mí, siempre que obtenga una parte mayor que usted.}

-\textit{Peor situado:} \say{Pues eso es bastante bueno; pero ¿pensé que estabas de acuerdo en que la justicia requiere igualdad?}

-\textit{Mejor situado:} \say{Sí, pero eso es solo como punto de referencia, ya ves. Para que estemos aún mejor, los dos, usted entiende, se pueden requerir pagos de incentivos diferenciales a personas como yo.}

-\textit{Peor situado:} \say{Vaya. Bueno, ¿qué los hace necesarios?}

-\textit{Mejor situado:} \say{Lo que los hace necesarios es que no trabajaré tan duro si no consigo más que usted.}

-\textit{Peor situado:} \say{Bueno, ¿por qué no?}

-\textit{Mejor situado:} \say{No lo sé... Supongo que esa es la forma en la que estoy constituido.}

-\textit{Peor situado:} \say{Lo que significa que realmente no te importa mucho la justicia, ¿eh?}

-\textit{Mejor situado:} \say{Em, no, supongo que no.}\footnote{\textit{Rawls on Equal Distribution of Wealth} \citep[p. 287-288]{Narveson_1978}: «\textit{Well-off:} \say{Look here, fellow citizen, I'll work hard and make both you and me better off, provided I get a bigger share than you.} \textit{Worse-off:} \say{Well, that's rather good; but I thought you were agreeing that justice requires equality?} \textit{Well-off:} \say{Yes, but that's only as a benchmark, you see. To do still better, both of us, you understand, may require differential incentive payments to people like me.} \textit{Worse-off:} \say{Oh. Well, what makes them necessary?} \textit{Well-off:} \say{What makes them necessary is that I won't work as hard if I don't get more than you.} \textit{Worse-off:} \say{Well, why not?} \textit{Well-off:} \say{I dunno... I guess that's just the way I'm built.} \textit{Worse-off:} \say{Meaning, you don't really care all that much about justice, eh?} \textit{Well-off:} \say{Er, no, I guess not.}»}
\end{quote}
\vspace{3mm}

La cita con la que comienza esta sección\footnote{Esta misma cita se encuentra también en \citet{Cohen_2008}.} puede resumir en buena medida la crítica de Cohen al \textit{argumento de los incentivos} presentado por diversos autores dentro de los cuales cabe destacar a Rawls (\citeyear{Rawls_1971}, \citeyear{Rawls_2002}). La crítica de Cohen sostiene que resulta contradictorio que personas que dicen estar motivadas por una idea de justicia al mismo tiempo exijan el pago de incentivos diferenciales para esforzarse bajo el argumento de que, si se brindan incentivos, todas las personas se verían beneficiadas en mayor o menos medida. La contradicción se presenta cuando una persona está motivada por un egoísmo contrario a la idea de igualdad como también por una idea de justicia que involucra la igualdad.  

Antes de entrar en el detalle de la crítica de Cohen, es necesario esclarecer algunas cuestiones. El argumento de los incentivos puede presentarse de diversas formas, por ejemplo, como defensa del capitalismo. Cuando la lógica mercantil basada en la propiedad privada es preponderante, los empresarios pueden disponer de sus bienes de la manera que mejor les parezca. Esto genera condiciones productivas favorables, en contraposición a situaciones donde existen restricciones a la iniciativa lucrativa. A su vez, se sostiene que hasta los desposeídos de bienes se ven beneficiados por el funcionamiento del libre mercado. En este contexto, Cohen plantea que aparece la noción de los incentivos ya que \say{interferir con la tendencia natural a que las ganancias se acumulen en manos de aquellos que disfrutan de la riqueza y de posiciones elevadas frena su creatividad como inversores, empresarios y generentes, lo que perjudica a todos} (\citeyear[p. 29]{Cohen_2014a}).

Esta formulación del argumento, no involucra cuestiones de justicia sino que enuncia las bondades de ciertos arreglos institucionales de las economías de mercado que terminarían beneficiando a todas las personas. Otra forma de presentar dicho argumento se logra mediante el postulado de que, en algunas situaciones, los incentivos diferenciales son justos. En este caso, \citet[p. 19]{Cohen_2008} plantea que el argumento puede tomar la siguiente forma:

\vspace{3mm}
\begin{enumerate}
    \item Las desigualdades son injustas a no ser que sean necesarias para hacer que las personas peores situadas se encuentren mejor. En tal caso son justas.
    \item Brindar incentivos desiguales a las personas productivas es necesario para hacer que las personas peores situadas se encuentren mejor.
    \item Por lo tanto, el pago de incentivos desiguales es justo\footnote{\say{1. Inequalities are unjust unless they are necessary to make the worst off people better off, in which case they are just. 2. Unequalizing incentive payments to productive people \textit{are} necessay to make the worst off people better off. 3. Therefore, unequalizing incentive payments are just} \citep[p. 19]{Cohen_2008}.}.
\end{enumerate}
\vspace{3mm}

Es contra esta versión del argumento que Cohen desarrolla la mayor parte de sus críticas. Según el autor, el argumento de los incentivos puede emplearse a la hora de presentar algún tipo de defensa de las desigualdades materiales existentes entre las personas. A la hora de elaborar una defensa de la desigualdad, \citet{Cohen_2001} sostiene que existen dos tipos de argumentos: normativos o fácticos. Un argumento del tipo fáctico plantea que las desigualdades materiales entre las personas resultan inevitables ya que los individuos se comportan mayoritariamente de manera egoísta. Este tipo de comportamiento se vincula con una explicación que puede apelar a una naturaleza humana o a una explicación sociológica. 

\citet{Cohen_2001} rechaza la explicación basada en una naturaleza humana egoísta ya que comparte la creencia marxista de que son las estructuras sociales las que determinan la conciencia de las personas y no en el sentido contrario. A su vez, el autor rechaza la explicación sociológica ya que, aunque las personas sean egoístas, se podrían concebir otros arreglos institucionales que no tuvieran como resultado la desigualdad. De todas formas, Cohen reconoce que antes su postura respecto del rechazo de la explicación sociológica era más contundente; ya no se encuentra tan convencido de que solamente modificando los arreglos institucionales se pueda lograr la igualdad. El autor muestra una mayor simpatía respecto de la noción de que para superar la desigualdad es necesario \say{que haya una revolución en el sentimiento o en la motivación, en oposición a una (mera) revolución en la estructura económica}  \citep[p. 163]{Cohen_2001}.

Por otro lado, un argumento de tipo normativo sostiene que las desigualdades son justas. En este caso, el argumento de los incentivos presentado anteriormente podría ser considerado como un ejemplo de una defensa normativa de la desigualdad. En particular, a través de la idea del principio de la diferencia, Rawls (\citeyear{Rawls_1971}, \citeyear{Rawls_2002}) plantea que las desigualdades que mejoran las condiciones de vida de los peores situados, son justas.

A modo ilustrativo, supongamos que tenemos una sociedad conformada por dos personas en la cual se busca igualar respecto a único bien en particular, como puede ser el salario percibido, entonces podríamos establecer el siguiente escenario:


\vspace{3mm}
\begin{table}[H]
\caption{Ejemplo del argumento de los incentivos}
\centering
\begin{tabular}{ccc|c}
                     & \textbf{A} & \textbf{} & \textbf{B} \\ \hline
\textbf{Situación 1} & 50         &           & 50         \\
\textbf{}            &            &           &            \\
\textbf{Situación 2} & 50 + $j$     &           & 50 + $i$    
\end{tabular}
\end{table}
\vspace{3mm}





Supongamos que la única diferencia relevante entre A y B es que A es más talentosa que B en el sentido de que puede conseguir los mismos resultados que B empleando un menor esfuerzo. En la Situación 1, tanto A como B reciben 50 unidades monetarias; si en la Situación 2 tenemos que A busca realizar un esfuerzo mayor, el argumento de los incentivos nos diría que dicho esfuerzo se realizaría solamente cuando\footnote{En este caso se podría cuestionar que a la persona le interese meramente sus ingresos y que no importe la comparación con los demás; de todas formas, lo que motivaría la acción en este caso es la idea de recibir mayores ingresos.} $50 + j > 50$. Por otro lado, el principio de la diferencia nos diría que el escenario donde $50 + j \neq 50 + i$ y $j > i$, es justo sí y sólo sí $i > 0$. Es decir, la persona peor situada, que en este ejemplo es B, se beneficia de los pagos diferenciales que se le proporcionan a A. 

Un problema que puede surgir de este sencillo ejemplo es que la única condición que se explicita es que $i >0$, por lo que el valor de $j$ podría tomar valores elevados en comparación al valor de $i$. Este tipo de crítica puede encontrarse en el planteo de Cohen. De todas formas, este tipo de argumento, al menos a la hora de ser una crítica al planteo de Rawls, no considera que cuando el principio de la diferencia interactua con el principio de igualdad de oportunidades se conforma una especie de conjunto de situaciones en donde estos principios no entrarían en conflicto. En otras palabras, la situación en donde $j$ toma valores muy elevados en comparación con $i$ no sería posible; podríamos decir que existe un valor $k > 0$ tal que si $j - i \leq k$ no se violaría el principio de igualdad de oportunidades y se estaría cumpliendo con el principio de la diferencia.

Ahora bien, Cohen cuestionaría que, incluso en el intervalo delimitado por $k$, existe una contradicción de la teoría ya que sigue existiendo la conjunción de que las personas que dicen estar motivadas por una idea de igualdad exigen pagos diferenciales dando cuenta de motivaciones egoístas contrarias a la igualdad. Es por esta razón que el autor sostiene que el argumento de Rawls detrás del principio de la diferencia no debe ser concebido como una defensa normativa de la desigualdad sino como un argumento fáctico.



El punto de Cohen es que no solamente la estructura legal en la cual los individuos interactúan es lo que importa a la hora de buscar la igualdad, sino que hay que tener en cuenta las elecciones que las personas realizan dentro de dicha estructura. En este sentido es que el autor adhiere al \say{slogan} de \say{lo personal es político}. Según \citet{Cohen_2001}, es posible vincular este punto con la crítica que se realiza desde el feminismo a las posturas liberales a las que pertenece Rawls. Dicha crítica plantea que, desde una perspectiva liberal, podría darse el caso de que exista una división del trabajo y relaciones de poder que sean sexistas e injustas, incluso cuando el sistema legal que regula a la sociedad no muestre dicho sesgo sexista. Al abstraer las cuestiones asociadas al género de la crítica feminista, se obtiene la idea de que \say{las opciones no reguladas por la ley caen dentro de los límites básicos de la justicia} \citep[p. 167]{Cohen_2001}.

Cabe destacar que la crítica de Cohen se centra en la forma en que se aplica el principio de la diferencia ya que, para el autor, se estarían justificando desigualdades que perjudicarían a las personas peores situadas. Existe una discrepancia con Rawls \say{sobre el asunto de \textit{qué} desigualdades pasan el test que justifica la desigualdad según el principio y, por tanto, \textit{cuánta} desigualdad admite ese test} \citep[p. 169]{Cohen_2001}. Retomando la formulación del principio de la diferencia, dado que las remuneraciones diferenciales son necesarias como motivación para las personas más talentosas, es que Cohen plantea que dicha necesidad gira en torno a una \textit{opción} que las personas talentosas pueden tomar. 

Como en el ejemplo analizado, es posible concebir a una persona talentosa como una que cuente con la capacidad de conseguir ganancias significativas en el mercado y que pueda variar su productividad en función de la remuneración que percibe. Para el autor, en múltiples ocasiones el hecho de que una persona se encuentre en una situación tan favorable puede deberse a circunstancias que están fuera del control del individuo. Por lo tanto, Cohen argumenta que:

\vspace{3mm}
\begin{quote}
    Si una persona logra producir más que otras, esto se debe a que es más talentosa, hizo un mayor esfuerzo o tuvo suerte en las circunstancias de producción, lo que equivale a decir que tuvo suerte respecto de aquellos y aquello \textit{con} que produce. Esta última razón para una mayor productividad, la concurrencia de circunstancias afortunadas resulta moralmente (y no económicamente) ininteligible como para alzarse con una recompensa mayor. Y si bien recompensar una productividad ligada a un mayor despliegue de talento resulta de hecho moralmente inteligible desde cierta perspectiva ética, aún así implica una idea profundamente antisocialista, condenada con razón por J. S. Mill como una instancia de \say{dar a los que ya tienen} (Mill, 1848, p. 210) \footnote{Mill, J. S. (1848). \say{Principles of Political Economy} en J. M. Robson (ed.), \textit{The Collected Works of John Stuart Mill}, vols. II y III, Toronto, University of Toronto Press, Londres, Routledge \& Kegan Paul, 1965 [ed. cast.: \textit{Principios de Economía Política con algunas de sus aplicaciones a la Filosofía Social}, México, FCE, 1951].}, en la medida en que un talento mayor en sí mismo es una fortuna que no exige mayor recompensa \citep[p. 65]{Cohen_2014b}.
\end{quote}
\vspace{3mm}

A su vez, el autor reconoce que el argumento de los incentivos, puede presentar cierto atractivo cuando se desarrolla de manera impersonal. Ahora bien, \citet{Cohen_2008}\footnote{En esta parte sigo el trabajo de Cohen: \textit{Incentives, Inequality, and Community} (\citeyear{Cohen_1992}) que se encuentra en \textit{Rescuing Justice and Equality} (\citeyear{Cohen_2008})} argumenta que si nos situamos en un caso en donde una persona talentosa tiene que defender sus incentivos diferenciales ante una persona peor situada, resulta extraño para este último encontrar una razón satisfactoria del comportamiento del primero\footnote{Una instancia de esto es la cita con la que comienza esta subsección.}. Antes de examinar el argumento enunciado en primera persona podemos preguntar lo siguiente: ¿qué es lo que hace que los incentivos sean necesarios? El autor maneja dos posibilidades: o las personas talentosas no quieren trabajar de la misma manera si se le retiran los incentivos diferenciales o no podrían trabajar de la misma manera por más de que tengan intenciones de hacerlo. Una explicación podría ser que estas personas necesitan de bienes caros para lograr altos niveles de desempeño o quizás necesitan las recompensas elevadas como forma de motivarse. Aún así, Cohen sostiene que este tipo de argumento presenta a las personas talentosas como más débiles de lo que realmente podrían ser. En particular, ya que parecería ser el caso que no pueden modificar su comportamiento.  

Respecto de la idea de que las recompensas son necesarias como motivación, podría pensarse que es una razón en tanto que, si no existieran dichos incentivos, las personas talentosas sentirían que ciertas expectativas que tenían no se estarían cumpliendo. En este sentido, Cohen plantea que en las economías de mercado, puede ser que las personas talentosas, a través de su experiencia y mediante ejemplos de otras personas talentosas, hayan desarrollado la creencia de que los talentos deben ser altamente recompensados. Nuevamente, todo gira en torno a la elección que estas personas deciden realizar ante estas expectativas lo que hacen que los incentivos sean necesarios. Por lo tanto, es que \citet{Cohen_2008} concluye que el argumento de la inhabilidad de los más talentosos depende solamente de sus hábitos y creencias normativas; las cuales pueden modificarse.

Habiendo dicho esto, \citet{Cohen_2008} plantea que el argumento de los incentivos no logra conseguir una justificación comprehensiva. Según el autor, un argumento para apoyar una política cuenta con una justificación comprehensiva cuando logra superar un \textit{test interpersonal}. Mediante este test, se pone a prueba la robustez del argumento en cuestión cambiando a las personas que lo enuncian como también a las personas a las cuales se les enuncia dicho argumento. Lo que puede ocurrir es que:

\vspace{3mm}
\begin{quote}
    Si, \textit{debido} a quién lo está presentando, y/o a quién se lo presenta, el argumento no puede servir como una justificación de la política, entonces pase o no como tal bajo otras condiciones dialógicas, no logra (\textit{tout court}) proporcionar una justificación comprehensiva de la política\footnote{\say{If, \textit{because} of who is presenting it, and/or to whom it is presented, the argument, cannot serve as a justification of the policy, then whether or not it passes as such under other dialogical conditions, it fails (\textit{tout court}) to provide a comprehensive justification of the policy} \citep[p. 42]{Cohen_2008}.} \citep[p. 42]{Cohen_2008}.
\end{quote}
\vspace{3mm}

Detrás de la idea de justificación comprehensiva, subyace la idea de una comunidad justificatoria: \say{una comunidad justificatoria es un conjunto de personas entre las cuales prevalece una norma (que no siempre puede ser satisfecha) de justificación comprehensiva}\footnote{\say{A justificatory community is a set of people among whom there prevails a norm (which need not always be satisfied) of comprehensive justification} \citep[p. 43]{Cohen_2008}.} \citep[p. 43]{Cohen_2008}. El punto del autor es que el argumento de los incentivos no logra superar el test interpersonal cuando se enuncia desde las personas más talentosas hacia las personas peores situadas. Específicamente, esto se debe a que los primeros no pueden justificar razonablemente el motivo por el cual modifican su comportamiento. Según Cohen, cuando se emplea el argumento de los incentivos para defender ciertas desigualdades, se tiene en mente un modelo de sociedad que carece de los elementos que conforman una comunidad justificatoria. 

Para ejemplificar esto, Cohen propone que imaginemos una situación en la que ejecutivos y profesionales de alta jerarquía se encuentran dialogando con trabajadores que reciben un salario bajo y/o personas que, por una razón u otra, dependen de las ayudas del estado de bienestar. En este caso, es posible tomar un ejemplo que presenta \citet{Cohen_2008} respecto de una modificación al esquema impositivo. Para presentar una defensa a la reforma impositiva ante las personas desfavorecidas, los ejecutivos podrían argumentar de la siguiente manera:

\vspace{3mm}
\begin{quote}
    Las políticas públicas deberían hacer que las personas peores situadas (en este caso, como sucede, ustedes) se encuentren mejor.
        
    Si el impuesto máximo sube al 60 porciento, nosotros trabajaremos menos, y, como resultado, la posición de los pobres (su posición) será peor.
         
    Por lo tanto, el impuesto máximo sobre nuestros ingresos no debería aumentarse al 60 porciento\footnote{\say{Public policy should make the worst off people (in this case, as it happens, you) better off. If the top tax goes up to 60 percent, we shall work less hard, and, as a result, the position of the poor (your position) will be worse. So the top tax on our income should not be raised to 60 percent} \citep[p. 59]{Cohen_2008}.} \citep[p. 59]{Cohen_2008}.
    
\end{quote}
\vspace{3mm}

Ahora bien, ante esta presentación en primera persona del argumento, una persona pobre podría preguntarle al talentoso rico respecto de cuál sería su justificación para trabajar menos cuando el impuesto sube. Ante esta exigencia de justificación, las personas talentosas no pueden invocar la idea de que las desigualdades son necesarias para que los pobres estén mejor porque son ellos mismos los que las hacen necesarias. Al querer argumentar de esta manera, Cohen plantea que podría darse un caso de alienación respecto de los talentosos ricos sobre su capacidad de actuar de otra manera. Una posible respuesta del talentoso rico podría ser algo como: \say{Mira, simplemente no valdría la pena trabajar tan duro si la tasa de impuestos fuera más alta, y si estuvieran en nuestro lugar se sentirían de la misma manera}\footnote{\say{Look, it simply would not be worth our while to work that hard if the tax rate were any higher, and if you were in our shoes you would feel the same way} \citep[p. 60]{Cohen_2008}.} \citep[p. 60]{Cohen_2008}.

Por su parte, dentro de las personas peores situadas, alguien podría contestar que, si estuvieran en la misma situación que las personas talentosas, no necesariamente se comportarían de la misma manera. Al verse frustrado el argumento por parte de la persona talentosa, ésta podría alegar que todas las personas poseen un derecho a perseguir ciertos intereses, que pueden ser denominados como egoístas, dentro de un nivel razonable. Según Cohen, es cierto que las personas tienen derecho a perseguir sus intereses pero \say{un modesto derecho al interés propio parece insuficiente para justificar el rango de desigualdad, los extremos de riqueza y pobreza, que realmente prevalecen en la sociedad en cuestión}\footnote{\say{[...] a modest right of self-interest seems insufficient to justify the range of inequality, the extremes of wealth and poverty, that actually obtain in the society under discussion} \citep[p. 61]{Cohen_2008}.} \citep[p. 61]{Cohen_2008}. Entonces parece ser el caso de que no existe un argumento satisfactorio que pueda presentarse desde la perspectiva de las personas talentosas y que la \textit{necesidad} de los incentivos radica exclusivamente en la voluntad de estas personas de no esforzarse de la misma manera.

En la misma linea, \citet{Cohen_2001} sostiene que el argumento de los incentivos implica una aplicación distorsionada del principio de la diferencia. Esto se debe a que o bien las personas más talentosas aceptan el principio de la diferencia o no lo aceptan. En el caso que las personas más talentosas no aceptaran el principio de la diferencia, tendríamos que dicha sociedad no sería considerada justa desde la perspectiva rawlsiana ya que resulta necesario que las personas acepten los principios de justicia que rigen a la misma. Por otro lado, en el caso alternativo, a las personas más talentosas se les podría formular la pregunta de: por qué \say{exigen un pago mayor del que obtienen aquellos menos dotados por un trabajo que, de hecho puede requerir un talento especial, pero que no es especialmente desagradable} \citep[p. 172]{Cohen_2001}. Cohen argumenta que resultaría difícil para estas personas contestar dicha pregunta y que no podría invocarse al principio de la diferencia. En particular, ya que es la actitud de los más talentosos, de no esforzarse o trabajar de la misma manera, lo que hace que dichos incentivos sean necesarios. Por lo tanto, Cohen concluye que las remuneraciones diferenciales para los más talentosos son \say{necesarias sólo porque las opciones de los más dotados no están debidamente ajustadas al principio de la diferencia} \citep[p. 173]{Cohen_2001}.


Es entonces que Cohen afirma que, debido a los problemas que las motivaciones egoístas generan respecto de las posibilidades de alcanzar un ideal igualitario, una sociedad justa requiere además de las reglas formales un \textit{ethos} que fomente la igualdad. A su vez, Cohen argumenta que de no existir dicho \textit{ethos}, se producirían desigualdades que no ayudarían a los que están peor situados. Este \textit{ethos} \say{fomenta una distribución más justa de lo que las reglas del juego económico pueden asegurar por sí mismas} \citep[p. 174]{Cohen_2001}.



Sin embargo, es posible presentar una objeción al argumento de Cohen. Dicha objeción ha sido denominada por el autor como \say{la objeción de la estructura básica}. La misma plantea que los principios de justicia no deberían aplicarse a las elecciones personales, sino que su ámbito de aplicación es la estructura básica de la sociedad. Para responder a esta crítica, el autor elabora dos posibles respuestas. Respecto de la primera respuesta, es posible encontrar fragmentos del trabajo de Rawls en donde se pueden evidenciar ciertas contradicciones con la idea de que los principios de justicia se aplican exclusivamente en la estructura básica de la sociedad. Concretamente, \citet{Cohen_2001} presenta tres instancias donde se presenta una contradicción: 

\vspace{3mm}
\begin{enumerate}
    \item Cuando se cumple con el principio de la diferencia, la sociedad demuestra fraternidad (en un sentido fuerte: las personas no quieren tener grandes ventajas entre sí).
    \item Cuando rige el principio de la diferencia, los peores situados llevan su situación con dignidad, puesto que no sería posible una mejora material para ellos.
    \item Cuando en una sociedad justa se actúa con un sentido de justicia, las personas aplican en sus vidas propias los principios de justicia.
\end{enumerate}
\vspace{3mm}

Ante la situación que se presenta en 1, Cohen argumenta que la fraternidad que se enuncia no sería posible si los principios se centraran exclusivamente en la estructura básica. En especial, ya que no se censurarían las conductas egoístas que terminarían creando desigualdades significativas. Respecto de 2, Cohen menciona que resulta falsa ya que podría lograrse una mejora de la situación de los peores situados si las personas talentosas no fueran egoístas. Finalmente, sobre 3, el autor plantea la pregunta sobre por qué sería necesario que las personas aplicaran los principios de la justicia a sus elecciones diarias si lo que importa, desde la propuesta de Rawls, es que las mismas se encuentren dentro del marco que establece la estructura básica.

De todas formas, esta primera respuesta de Cohen no resulta tan contundente ya que en cada uno de estos casos, existe una salida para Rawls: o elimina la restricción respecto del ámbito de aplicación de los principios de justicia o elimina los comentarios 1, 2 y 3. Es en esta última dirección que procede Rawls respecto del comentario 1, ya que para éste el comentario implicaría ir en la dirección de postular una concepción comprehensiva de la justicia, es decir, una concepción moral completa no meramente política. No obstante, esta salida tiene un costo ya que \say{no se puede decir que los ideales de dignidad, fraternidad y la total realización de las naturalezas morales de la gente se expresan a través de la justicia rawlsiana} \citep[p. 183]{Cohen_2001}.

Debido a esto es que Cohen desarrolla una segunda respuesta a la objeción de la estructura básica. Para el autor esta respuesta lograría mostrar que la justicia requiere \say{un \textit{ethos} que gobierne las elecciones diarias, un \textit{ethos} que va más allá de la obediencia a las reglas justas} \citep[p. 184]{Cohen_2001}. Esta respuesta gira en torno a indagar respecto de qué es lo que conforma la estructura básica, es decir, cuáles instituciones son las que definen dicha estructura. Una primera posibilidad es que la estructura básica consista en las instituciones de carácter coercitivo-legal, es decir, aquellas que delimitan el comportamiento desde el punto de vista legal. 

Sin embargo, Cohen argumenta que esta no parece ser la caracterización correcta. En particular, ya que en el planteo de Rawls se dice que la estructura básica \say{consiste en las instituciones sociales \textit{más importantes} y no pone énfasis particular en la coerción cuando anuncia \textit{esa} especificación de la estructura básica} \citep[p. 185]{Cohen_2001}. Ahora bien, si se define a la estructura básica de esta manera, podrían incluirse instituciones \textit{importantes} que se apartan de las dependen exclusivamente de la ley. En especial, un caso paradigmático es el de la familia. Según Cohen, en el planteo rawlsiano a veces se incluye a la familia dentro de la estructura básica y a veces no. En el caso de incluir a instituciones que dependen más de la convención y la expectativa, como la familia, no se podría argumentar que hay que excluir del ámbito de la justicia a las elecciones personales no restringidas desde el punto de vista legal. Concretamente, ya que dichas instituciones se rigen a partir de las elecciones que las personas realizan en su vida cotidiana y algunas elecciones son censuradas dentro de las convenciones sociales de carácter informal.

Una posible salida desde la perspectiva rawlsiana, sería nuevamente volver a una caracterización meramente coercitiva. Sin embargo, cabe recordar que Rawls pone el foco en la estructura básica de la sociedad como el asunto principal de la justicia debido a sus efectos profundos sobre la configuración de las instituciones sociales. Por lo cual, Cohen argumenta que \say{es falso que sólo la estructura \textit{coercitiva} cause efectos profundos, como una vez más nos recuerda el ejemplo de la familia} (\citeyear[p. 187]{Cohen_2001}). Los efectos de la estructura informal respecto de las cuestiones de justicia se puede evidenciar a través de dos ejemplos: las elecciones dentro del mercado y la familia.

Por un lado, respecto de la familia, se tiene un ámbito en donde podría decirse que se distribuyen las cargas y beneficios de la asociación de las personas que conforman un hogar. Dentro de la familia existen prácticas que no están definidas por la ley que tienen un impacto importante sobre las posibilidades vitales de las personas involucradas. Sin embargo, si definimos a la estructura básica como exclusivamente coercitiva-legal, tendríamos que las prácticas que se reproducen dentro de la familia, que pueden tener un sesgo sexista, quedan por fuera del ámbito de la justicia. Por otro lado, en el caso de las elecciones dentro del mercado, podría concebirse una legislación que maximice la cuantía de los bienes primarios de la sociedad y lograra cumplir con el principio de la diferencia. Según Cohen, dicha situación es compatible con un \textit{ethos} maximizador que puede producir \say{grandes desigualdades y un escaso nivel de abastecimiento para los que peor están; no obstante Rawls tiene que declarar que esas dos cosas son justas si mantiene una concepción coercitiva de lo que la justicia juzga} \citep[p. 189]{Cohen_2001}. Una manera concisa de formular el punto de Cohen se evidencia en la siguiente pregunta: 

\vspace{3mm}
\begin{quote}
   ¿Por qué nos preocupa de manera tan desproporcionada la estructura básica coercitiva, cuando la principal razón para que nos preocupe, su impacto sobre las vidas de las personas, es también una razón para preocuparnos por la estructura informal y los criterios de elección personal? \citep[p. 190]{Cohen_2001}. 
\end{quote}
\vspace{3mm}

Entonces Cohen concluye que, debido al impacto que tiene la estructura informal sobre las posibilidades de realización del ideal de igualdad, resulta necesario un \textit{ethos} que influya sobre las elecciones personales además de una correcta formulación de la estructura formal de la sociedad. Resulta apropiado preguntarse en qué consiste este \textit{ethos}. El concepto de \textit{ethos} de una sociedad consiste en \say{un grupo de sentimientos y actitudes en virtud del cual su práctica normal y sus presiones informales son lo que son} \citep[p. 197]{Cohen_2001}. El autor reconoce que las presiones informales carecen de fuerza cuando no hay una práctica normal establecida a través de reglas que dichas presiones buscan hacer cumplir. 

Finalmente, Cohen presenta el ejemplo de las desigualdades económicas entre el Reino Unido y Estados Unidos como una instancia de cómo diferentes \textit{ethos} pueden generar ciertos comportamientos. En los primeros años posteriores a la Segunda Guerra Mundial, en los dos países considerados existía una economía de mercado. Además, se daba el caso de que la diferencia de los ingresos entre un ejecutivo de alta jerarquía y un obrero era mayor en Estados Unidos que en el Reino Unido. De todas formas, el autor estipula que:

\vspace{3mm}
\begin{quote}
    [...] muchos de los ejecutivos británicos \textit{no} habrían sentido la tentación de decir: nosotros deberíamos presionar para ganar más, puesto que había un \textit{ethos} de reconstrucción después de la guerra, un \textit{ethos} de proyecto común, que moderó el deseo de ganancia personal \citep[p. 195]{Cohen_2001}
\end{quote}
\vspace{3mm}

Habiendo presentado la faceta crítica de Cohen respecto de la posibilidad de conciliar incentivos y justicia, a continuación, se presenta el desarrollo de la propuesta del autor.


\subsection{La propuesta de Cohen}

A la hora de formular la propuesta de Cohen, se consideran dos trabajos en particular. En primer lugar, en el ensayo \textit{¿Por qué no el socialismo?} \citet{Cohen_2014c} presenta un modelo básico de un campamento para establecer un caso en donde se preferiría un tipo de organización socialista. A su vez, el autor desarrolla los principios normativos que se encuentran en el trasfondo de dicho modelo. En segundo lugar, en \textit{Un retorno a los fundamentos del socialismo} \citep{Cohen_2014b}, el autor analiza un documento elaborado en 1993 por un \textit{think-tank} cercano al Partido Laborista de Inglaterra. En dicho documento, se planteaba la necesidad de emular el ascenso que había experimentado la derecha en la década anterior. A su vez, se argumenta que para lograr dicho objetivo, no resultaba necesario entrar en una discusión filosófica respecto de los principios de la izquierda. Cohen discrepa con esta idea y sostiene que resulta de vital importancia que la izquierda se reapropie de sus valores fundamentales.

En el contexto del modelo del campamento \citep{Cohen_2014c}, los participantes no tendrían razones para organizarse según jerarquías. Todos comparten un objetivo común que consiste en pasarla bien cada uno haciendo las actividades que más les guste hacer ya sea individualmente o en grupo. Existen instrumentos para llevar a cabo dichas actividades y éstos son aprovechados colectivamente. A su vez, se distribuyen las tareas según los intereses de las personas:

\vspace{3mm}
\begin{quote}
    Uno pesca, otro prepara la comida y otro cocina. Aquellas personas que odian cocinar pero disfrutan lavar, lavan, y así sucesivamente. Somos muy diferentes, pero nuestros acuerdos mutuos y espíritu de nuestro emprendimiento aseguran que no haya desigualdades en las cuales alguien pudiera fundar una queja \citep[p. 180]{Cohen_2014c}.
\end{quote}
\vspace{3mm}

A continuación Cohen plantea la posibilidad de que sea otro el modo en que se organice el campamento. Por ejemplo, empleando una lógica más individualista. En esta nueva forma de organizar las tareas, Cohen presenta algunos escenarios de los cuales se destaca el siguiente: 

\vspace{3mm}
\begin{quote}
    A Harry le encanta pescar, y Harry es muy buen pescador. Por consiguiente, él aporta más pescado que los demás. Harry dice: \say{El modo en que estamos manejando las cosas es injusto. Yo debería comer el mejor pescado. Yo debería comer trucha\footnote{En esta traducción realizada por Luciana Sanchez, Roberto Gargarella, Félix Ovejero y Verónica Lifrieri, con el objetivo de facilitar la comprensión del ejemplo, se cambia la palabra \say{perch}, que es un tipo de comida regional estadounidense, por pescado.}, no la mezcla de trucha y bagre que todos comimos hasta ahora}. Pero sus compañeros le dirían: \say{Oh, Dios, Harry, no seas tan cretino. Te esfuerzas y transpiras tanto como nosotros. Claro que eres muy buen pescador. Nosotros no despreciamos este don especial que tienes, que en realidad constituye una fuente de satisfacción para ti; pero ¿por qué deberíamos recompensarte por esta habilidad preexistente?} (\citeyear[p. 181-182]{Cohen_2014c}).
\end{quote}
\vspace{3mm}

Este ejemplo busca mostrar que, en un campamento, nuestras intuiciones parecen entrar en conflicto con la forma de organizar las cosas de manera más individualista. A su vez, esta idea se vincula con el punto anterior de que los incentivos diferenciales van en contra de la idea de la igualdad.  

Luego de presentar este modelo del campamento, Cohen procede a presentar dos principios que se manifiestan en dicho modelo: un principio de igualdad y un principio comunitario. Según el autor, el principio asociado a una idea de comunidad \say{restringe la operatividad del principio de igualdad, al prohibir determinadas inequidades en los resultados que el principio de igualdad permite} \citep[p. 183]{Cohen_2014c}. En la misma línea, volviendo sobre los principios socialistas, se puede decir que \say{el principio de igualdad sostiene que en términos generales la cantidad de cargas y beneficios que tiene una persona en su vida debería ser comparable a la de cualquier otra} \citep[p. 62]{Cohen_2014b}.

En el contexto del modelo del campamento, Cohen tiene en mente un principio de igualdad de oportunidades. Para presentar dicho principio, el autor desarrolla unos pasos previos. En primer lugar existe una \say{igualdad de oportunidades burguesa}. Bajo este tipo de ideal, se quitarían las restricciones socialmente construidas respecto de las oportunidades de vida \say{causadas por una asignación específica de derechos y por una percepción social intolerante y perjudicial} \citep[p. 184]{Cohen_2014c}. En segundo lugar, tendríamos una \say{igualdad de oportunidades de la izquierda liberal}. En este caso se va más allá de la igualdad de oportunidades burguesa y se propone quitar desigualdades que corresponden a las circunstancias de \say{nacimiento  y crianza, cuya restricción obra no por medio de la asignación de sus víctimas de un estatus inferior, sino por su sometimiento a la pobreza y otros medios de privación} \citep[p. 185]{Cohen_2014c}. Finalmente tendríamos la \say{igualdad de oportunidades socialista}, la cual: 

\vspace{3mm}
\begin{quote}
    [...] busca corregir todas las desventajas no elegidas, desventajas que son tales porque el agente no puede ser considerado racionalmente responsable de ellas, ya sea que reflejen desgracias sociales o desgracias naturales. Cuando prevalece la igualdad de oportunidades socialista, las diferencias en el resultado no reflejan más que diferencias de gusto o elección, en vez de reflejar las debidas a capacidades y poderes naturales o sociales \citep[p. 186]{Cohen_2014c}.
\end{quote}
\vspace{3mm}

Cohen sostiene que el principio de igualdad de oportunidades socialista es compatible con tres tipos de desigualdades, aunque en algunos casos existe una tensión problemática. En primer lugar, se encuentran las desigualdades en el acceso a algunos tipos de bienes por partes de las personas. Estas diferencias se explican por la elección de distintos estilos de vida por sobre otros que realizan las personas. Este tipo de desigualdad no representa un problema ya que las personas realizarían estas elecciones en un contexto favorable a la elección de diversos caminos de acción. Los tipos de desigualdad restantes consisten en: 2) desigualdades debidas a diferencias en el esfuerzo que se escoge realizar y 3) desigualdades debidas a las diferencias en la fortuna en la elección.

El segundo tipo de desigualdad se justifica \say{en razón del esfuerzo y/o la preocupación diferente de las personas que se encuentran, al inicio, en perfecta igualdad de condiciones y que son iguales hasta en sus capacidades para emplear su esfuerzo y en su preocupación} \citep[p. 189]{Cohen_2014c}. Finalmente, el último tipo de desigualdad, la basada en la suerte de opción, resulta la más problemática\footnote{Debido a este énfasis en cómo la suerte puede afectar las perspectivas igualitaristas, es que a autores como Cohen y Roemer se los considera pertenecientes a una corriente denominada, por primera vez en \citet{Anderson_1999}, como \textit{igualitarismo de la suerte} o \textit{luck egalitarianism}.}. El ejemplo tradicional es el de una apuesta entre individuos. Según Cohen, este tipo de desigualdad puede ser similar a algunas situaciones que ocurren dentro de los mercados por el accionar de los individuos.

Ahora bien, aunque las desigualdades de tipo 2 y 3 no sean condenadas bajo el principio de igualdad de oportunidades, eso no las vuelve menos reprochables desde una visión socialista. Por lo tanto, Cohen propone un principio comunitario como forma de contrarrestar la magnitud que podrían tomar estos tipos de desigualdades. El principio de comunidad implica que a las personas \say{les importe y, cuando sea necesario y posible, se preocupen por la suerte de los demás. Y también que les importe preocuparse los unos por los otros} \citep[p. 191]{Cohen_2014c}. 

En la misma linea, este principio comunitario era concebido por \citet{Cohen_2014b} como un principio \say{antimercado}. La razón de esto es que las personas que se rigen por este principio cooperan entre sí debido a que reconocen que los demás necesitan de sus servicios y no lo hacen en búsqueda la recompensa que podrían obtener. Este principio resulta antimercado en la medida que el mercado \say{estimula la contribución productiva, no en función del compromiso con los demás y el deseo de servirles y de ser servido \textit{por} ellos a la vez, sino en función de una recompensa monetaria impersonal} \citep[p. 59]{Cohen_2014b}. Si bien es posible reconocer que algunas personas están motivadas por esta idea de servir a los demás, Cohen plantea que dicho ideal no es lo que hace funcionar al mercado como mecanismo de asignación.

Volviendo a su posterior ensayo, \citet{Cohen_2014c} profundiza sobre su forma de entender al principio de comunidad. En particular, el autor presenta dos formas de concebir este principio comunitario. En primer lugar, si las personas no enfrentan adversidades similares, debido a que sus experiencias vitales son muy distintas, es difícil que las personas puedan entender las problemáticas de los otros. Algo de esta índole sucede cuando las personas perciben ingresos muy dispares. Concretamente, esto ocurre cuando las personas peores situadas se enfrentan a ciertas vulnerabilidades sociales que podrían ser en parte remediadas con la ayuda de los mejores situados. En este ejemplo, el principio de comunidad podría ser satisfecho si las personas de mayores ingresos destinaran una porción de los mismos, dentro de unos niveles razonables que no impliquen un sacrificio elevado, para ayudar a las personas pobres.

La segunda forma de concebir al principio comunitario consiste en una forma comunitaria de reciprocidad. Para el autor, esta reciprocidad, en un contexto de igualdad de oportunidades, no se considera como un requisito de la igualdad sino que es un principio \say{para que las relaciones humanas adquieran una forma deseable} \citep[p. 193]{Cohen_2014c}. En la misma línea, Cohen plantea que:

\vspace{3mm}
\begin{quote}
    La reciprocidad comunitaria es un principio negatorio del mercado, un principio conforme al cual yo le sirvo a usted no por lo que pueda llegar a obtener a cambio, sino porque usted necesita mis servicios; por ese mismo motivo, usted me sirve a mí. La reciprocidad comunitaria no es lo mismo que la reciprocidad de mercado, en tanto el mercado incentiva las contribuciones productivas no sobre la base del compromiso con nuestros congéneres, y nuestro deseo de servirles en tanto somos servidos por ellos, sino sobre la base de una recompensa económica \citep[p. 193]{Cohen_2014c}.
\end{quote}
\vspace{3mm}

En términos generales, \citet{Cohen_2014c} argumenta que las motivaciones principales que guían las actividades productivas en una economía de mercado son una combinación de miedo y codicia. Desde la óptica de la codicia, las otras personas son vistas como posibles fuentes de enriquecimiento; mientras que desde el miedo, los demás son vistos como posibles amenazas. Por esta razón es que Cohen afirma que el principio de reciprocidad comunitario se encuentra en las antípodas de una lógica mercantilista.

Ahora bien, habiendo desarrollado estos principios de igualdad y comunidad, Cohen presenta dos preguntas respecto del modelo del campamento: 1) ¿es deseable? y 2) ¿es factible? Respecto de la primera pregunta, Cohen maneja una posible crítica en la cual este modelo sería indeseable. En particular, este modelo no sería deseable debido a que no cuenta con un ámbito en donde las personas puedan elegir ciertas acciones que tengan resultados desiguales o impliquen la instrumentalización de otra persona. Esta crítica no sería tan importante si se tiene en cuenta que siempre va a existir un lugar para la elección de las personas y que incluso en las sociedades de mercado, las elecciones de un individuo se ven limitadas por las elecciones de los demás.

Respecto de la factibilidad, Cohen se pregunta si el egoísmo humano o la tecnología social disponible pueden ser obstáculos a la realización de los ideales considerados. Ante esta pregunta, es posible distinguir dos obstáculos \citep{Cohen_2014c}. En primer lugar, las personas pueden no ser lo suficientemente generosas y cooperativas cuando nos alejamos del ámbito reducido del modelo del campamento. En segundo lugar, aunque las personas mostraran los niveles necesarios de generosidad y cooperación, podría ser el caso de que no conozcamos la forma de hacer que dichos valores fueran el \textit{motor} de la economía. El tipo de factibilidad que le interesa a Cohen no es meramente la de poder implementar un sistema inspirado en los ideales del campamento, sino también cómo lograr que dicho sistema pueda mantenerse en el tiempo. Respecto a este objetivo, el autor sostiene que el principal problema al que se enfrentan los partidarios del modelo del campamento es un problema de diseño institucional. En la misma línea, Cohen argumenta que:  

\vspace{3mm}
\begin{quote}
    Egoísmo y generosidad existen, a fin de cuentas, en (¿casi?) todos. Nuestro problema es que, aunque sabemos cómo hacer funcionar un sistema económico sobre la base del egoísmo, no sabemos cómo hacerlo funcionar sobre la base de la generosidad. Incluso si en el mundo real, en nuestra propia sociedad, muchas cosas dependen de la generosidad, o, para expresarlo de manera más general y más negativo, no dependen de incentivos de mercado \citep[p. 199]{Cohen_2014c}.
\end{quote}
\vspace{3mm}

Ante este tipo de dificultad es que surgen modelos de socialismo de mercado. Estos modelos buscan combinar ciertos aspectos positivos de las economías de mercado, como la eficiencia productiva, con criterios organizacionales o regulatorios del socialismo para poder mitigar los efectos no deseados de los dos sistemas. Si nos situamos dentro de los proyectos socialistas del siglo XIX, una idea primordial era erradicar la organización mercantilista de la economía. Para lograr esto, se proponía implementar una planificación centralizada de la vida económica. Ahora bien, Cohen propone que la experiencia que surgió de la implementación de regímenes socialistas en el siglo XX nos dice que la centralización, al menos en la manera en que fue llevada a cabo, no es capaz lograr el éxito económico una vez que se llega a ciertos niveles de avance en la estructura productiva. Es por esto que se buscó ir más allá de la centralización y se desarrolló un modelo de socialismo de mercado que:

\vspace{3mm}
\begin{quote}
    [...] es socialista porque vence la división entre capital y trabajo: en el socialismo de mercado no existe una clase capitalista que enfrenta a los trabajadores desprovistos de capital, ya que quienes poseen las empresas son los trabajadores. Sin embargo, el socialismo de mercado se diferencia del socialismo tradicional en que estas empresas en propiedad de los trabajadores compiten entre sí y por los consumidores, al estilo de la competencia de mercado \citep[p. 204]{Cohen_2014c}.
\end{quote}
\vspace{3mm}

Aunque estos modelos puedan resultar en una mejora con respecto de los arreglos institucionales vigentes, \citet{Cohen_2014c} plantea que los modelos de socialismo de mercado deben ser vistos como un \say{segundo mejor}. Esto se debe a que: en primer lugar, el ideal de igualdad se ve afectado por la competencia de mercado creando ganadores y perdedores; y en segundo lugar, el ideal de comunidad se ve perjudicado por la lógica mercantil que reduce el ámbito de la reciprocidad.

En este contexto, Cohen realiza una crítica de algunos fervientes defensores de los modelos de socialismo de mercado. El autor sugiere que diversos intelectuales socialistas desarrollaron \say{preferencias adaptativas}, es decir, sus preferencias se vieron distorsionadas por una concepción de lo que consideran factible. Debido a esto: \say{Muchos socialistas llegaron a la conclusión de que el socialismo de mercado es maravilloso simplemente porque creen que no pueden diseñar nada mejor} \citep[p. 205]{Cohen_2014c}.

Queda entonces desarrollada la propuesta de Cohen y su \say{desafío} de pensar formas superiores al socialismo de mercado para poder implementar los ideales socialistas de igualdad y comunidad. A continuación, se desarrolla la propuesta de Roemer en tanto implica un intento de operacionalizar las ideas desarrolladas por Cohen. Este desarrollo lleva a Roemer a presentar una superación del concepto de socialismo de mercado estándar.






\section{La modelización de Roemer} \label{sec4}
Un autor que recoge el guante ante el desafío de Cohen es el economista John E. Roemer, especialmente si consideramos sus más recientes trabajos. Antes de entrar en el desarrollo de las propuestas más recientes, es necesario presentar algunos desarrollos teóricos previos para dar cuenta de la evolución en su pensamiento.

\subsection{Socialismo de mercado}

A lo largo de su trabajo, Roemer se ha presentado como un académico que logra vincular los aspectos normativos propios de la filosofía con aspectos operativos propios de la economía, es decir, el autor logra operacionalizar ciertas propuestas filosóficas a través de modelos económicos. Este autor comparte una trayectoria similar a la de Cohen, en el sentido de que ambos comenzaron analizando problemas desde una perspectiva marxista y luego, a través de varias influencias, se embarcan en cuestiones de justicia distributiva.

El trabajo de \citet{Roemer_1982} respecto de cuestiones normativas comienza con un análisis de la explotación en la tradición marxista aplicando los desarrollos de la teoría de juegos\footnote{En esta parte, para presentar un recorrido temporal en el trabajo del autor, sigo la introducción de \citet{Roemer_1996b}.}. Según el autor, la filosofía política o los conceptos normativos que estaban detrás del concepto marxista de explotación no eran claros. En este sentido, Roemer argumenta que lo que motivaba el diagnóstico de la presencia de explotación es una filosofía política igualitaria. En particular, se entendía que la justicia implica la distribución igualitaria de ciertos tipos de propiedad entre las personas. Habiendo llegado a esta noción es que el autor plantea que el enfoque de la explotación basado en la plusvalía estaba mal direccionado. Adicionalmente, Roemer sostiene que lo que Marx realmente quiso decir se captura mejor bajo un enfoque de la explotación basado en las relaciones de propiedad. 



Del análisis del concepto de explotación, Roemer concluye que la filosofía política que justificaba la crítica marxista del capitalismo es una especie de igualitarismo de recursos. Esto llevó al autor a indagar los trabajos de Dworkin (\citeyear{Dworkin_1981a}, \citeyear{Dworkin_1981b}) respecto de cuestiones igualitaristas. En la discusión del trabajo de Dworkin, Roemer comparte con \citet{Cohen_1989} que el \say{punto de corte} respecto de qué cosas son consideradas como responsabilidad del individuo no deberían ser sus preferencias. La propuesta tanto de Cohen como Roemer es considerar un conjunto de características llamadas \say{circunstancias}, las cuales están fuera del control del individuo y compensar a los individuos en función de cuan favorables o desfavorables resultan dichas circunstancias. Finalmente, según el autor, un igualitarista que defendiera estas ideas debería: \say{en el mundo actual, abogar por un mecanismo económico que puede describirse como socialismo de mercado}\footnote{\say{[...] in our present world, advocate an economic mechanism that can be described as market-socialism} \citep[p. 4]{Roemer_1996b}.} \citep[p. 4]{Roemer_1996b}. Por lo tanto, a continuación, se analiza una propuesta de Roemer: su modelo de socialismo de mercado.


Respecto del modelo de socialismo de mercado, es posible tomar como referencia el trabajo de \textit{A Future for Socialism} \citep{Roemer_1994}. En el contexto de la caída de la Unión Soviética, se había reforzado la creencia de que el socialismo no podría existir tanto en el presente como en una situación ideal. Ante esta situación, Roemer argumenta que es necesario revisar las experiencias del socialismo real y sugiere que se debe defender un nuevo modelo denominado como socialismo de mercado. En dicho modelo se busca combinar las fortalezas del sistema del mercado con las fortalezas del socialismo. A través de esta combinación, el modelo resultante estaría preocupado por lograr tanto la eficiencia productiva como también cumplir con ciertos ideales igualitarios. 

Según Roemer, esta conjunción se sustenta sobre la evidencia que muestra que los mercados no logran por si solos todos sus buenos resultados sino que necesitan ayuda. Los mercados se encuentran apoyados por muchas instituciones que han ido evolucionando a lo largo del tiempo para dar respuesta a diversos problemas. El autor plantea que: 

\vspace{3mm}
\begin{quote}
En contraste con la visión neoclásica \say{delgada}, que ve a los mercados como una estructura mínima que organiza la competencia entre individuos talentosos, la visión \say{gruesa} moderna ve a los mercados como una parte de una red compleja de instituciones creadas por el hombre, a través de las cuales todas las contribuciones de los individuos se pasteurizan y refinan\footnote{\say{In contrast to the «thin» neoclassical view, which sees markets as a minimal structure organizing competition among talented individuals, the modern \say{thick} view sees markets as part of a complex network of man-made institutions, through which all individuals contributions become pasteurized and refined} \citep[p. 5-6]{Roemer_1994}.} \citep[p. 5-6]{Roemer_1994}. 
\end{quote}
\vspace{3mm}

La visión \say{gruesa} de los mercados sería la apropiada y la más dócil para la coexistencia de los mercados y el socialismo. Ahora bien, ¿cuál sería la razón para proponer este tipo de modelo? Para dar cuenta de esto, \citet{Roemer_1994} presenta su visión de los objetivos que persiguen los socialistas. Dichos objetivos consisten en alcanzar una igualdad de oportunidades para: 1) la autorrealización y el bienestar, 2) la influencia política, y 3) el estatus social\footnote{En este punto se destaca una similitud con la propuesta rawlsiana. En particular, \citet[p. 177-178]{Rawls_2002} distingue tres razones para regular las desigualdades económicas: a) resulta erróneo que unos pocos tengan mucho y otros sufran privaciones, b) impedir que una parte de la sociedad domine al resto y c) impedir desigualdades de estatus social.}. Bajo la idea de autorrealización se entiende un proceso de desarrollo y aplicación de los talentos de un individuo de manera tal que le de sentido a su vida. Para Roemer, lo que distingue a las propuestas socialistas es que para ellos no basta con eliminar la discriminación en la contratación y garantizar el acceso a la educación. El ideal de igualdad de oportunidades requiere que las personas sean compensadas por desventajas que son inducidas por factores sobres los cuales no tienen ningún control\footnote{Esta visión se encuentra desarrollada extensamente en \citet{Roemer_1998}, en donde se presenta un algoritmo capaz de traducir cualquier visión, respecto de la proporción de las circunstancias y esfuerzo de las personas, en una política social que garantice el cumplimiento de un ideal de igualdad de oportunidades.}. La situación ideal sería encontrar unos arreglos institucionales o derechos de propiedad que logren obtener lo máximo posible de los tres objetivos. Sin embargo, el autor reconoce que no es posible maximizar los tres objetivos a la vez, lo que se debe hacer es buscar un \textit{trade-off} entre los mismos. 

Ahora bien, \citet{Roemer_1994} plantea que sería posible argumentar, por parte de algunos socialistas, que los objetivos defendidos desde su posición no son más que los postulados por el credo del liberalismo igualitario. En particular, debido a que los objetivos perseguidos por el socialismo es la eliminación del sistema capitalista en el cual una pequeña clase vive de la plusvalía que extraen de los trabajadores a través de la explotación. Ante esta crítica, el autor argumenta que es necesario desarrollar la crítica al capitalismo desde la distribución desigual de la propiedad de algunos bienes en contraposición al enfoque de la plusvalía. Al tomar esta nueva perspectiva, es necesario esclarecer cuestiones de filosofía política normativa, cuestiones que no eran tomadas en cuenta desde el marxismo clásico. En este sentido, pueden existir ciertas coincidencias con los postulados del liberalismo igualitario pero eso no implica que la corriente socialista colapse en un liberalismo igualitario. 

Partiendo desde los objetivos y el énfasis en los derechos de propiedad, es necesario mencionar a la noción de propiedad pública como principal alternativa a la propiedad privada. Según Roemer, como resultado de las experiencias del socialismo real se ha tendido a asociar a la propiedad pública con la propiedad estatal cuando, en rigor, estos conceptos no son lo mismo. En la misma linea, el autor sostiene que \say{los socialistas han convertido a la propiedad pública en un fetiche: la propiedad pública ha sido vista como la condición \textit{sine qua non} del socialismo, pero este juicio se basa en una inferencia falsa}\footnote{\say{[...] socialists have made a fetish of public ownership: public ownership has been viewed as the \textit{sine qua non} of socialism, but this judgement is based on a false inference} \citep[p. 20]{Roemer_1994}.} \citep[p. 20]{Roemer_1994}. El punto de Roemer es que la forma que tengan los derechos de propiedad va a depender exclusivamente de su idoneidad a la hora de cumplir con los objetivos establecidos anteriormente. Concretamente, el autor sostiene que: 

\vspace{3mm}
\begin{quote}
[...] considero que la elección de los derechos de propiedad sobre las empresas y otros recursos es un asunto enteramente instrumental; los socialistas deben evaluar las posibilidades de organizar tales derechos de acuerdo con la probabilidad de que induzcan las tres igualdades con las cuales los socialistas están preocupados\footnote{\say{[...] I view the choice of property rights over firms and other resources to be an entirely instrumental matter; possibilities for organizing such rights should be evaluated by socialists according to the likelihood that they will induce the three equalities with which socialists are concerned} \citep[p. 23]{Roemer_1994}.} \citep[p. 23]{Roemer_1994}.    
\end{quote}
\vspace{3mm}

Por otro lado, dentro de la discusión de la idea de socialismo de mercado, Roemer distingue cinco etapas. Dentro de este debate, es posible distinguir a \citet{Lange_1956} como uno de los principales defensores de la idea del socialismo de mercado y a \citeauthor{Hayek_1935} (\citeyear{Hayek_1935}, \citeyear{Hayek_1940}) como uno de los principales críticos. En primer lugar, el debate se centró en la necesidad de usar un sistemas de precios para realizar el cálculo económico dentro de las economías socialistas, ya que utilizar una unidad natural como la cantidad de energía o trabajo que se requiere para producir un bien tendría serios problemas. En segundo lugar, la discusión derivó en la visión de que dentro de una economía socialista sería posible calcular unos precios en donde se lograría alcanzar el equilibrio general resolviendo un sistema complejo de ecuaciones simultáneas. La tercera etapa se caracterizó por la idea de que resultarían necesarios los mercados en una economía socialista para poder hallar los equilibrios. En particular, debido a que los planificadores centrales nunca podrían obtener toda la inmensa información necesaria para realizar los cálculos sobre qué y cuánto producir.

En cuarto lugar, los aportes provenientes desde la teoría de la compatibilidad de incentivos, la cual no estaba interesada directamente en este debate, contribuyó de manera importante. Esto se asocia con el funcionamiento de las economías socialistas ya que en dicho contexto se genera un problema de principal-agente entre el planificador central y los administradores de las empresas. La teoría de la compatibilidad de incentivos se centra en el estudio de mecanismos que logren que los agentes (en este caso los administradores) encuentren que actuar de acuerdo a la regla propuesta por el principal (en este caso los planificadores) es su mejor curso de acción. Finalmente, la quinta etapa se caracterizó por reconocer que no era necesario insistir exclusivamente en la propiedad pública, entendida como propiedad estatal, de las empresas o, de manera más general, de los medios de producción.

Las fortalezas de los modelos resultantes de este debate residen en que incorporan las lecciones del socialismo de la Unión Soviética. Según Roemer, la caida del modelo soviético se debió a una conjunción de tres de sus características: 


\vspace{3mm}
\begin{quote}
(1) la asignación de la mayoría de los bienes por un aparato administrativo bajo el cual los productores no estaban obligados a competir entre sí, (2) el control directo de las empresas por parte de unidades políticas, y (3) la política no democrática y no competitiva\footnote{\say{(1) the allocation of most goods by an administrative apparatus under which producers were not forced to compete with each other, (2) direct control of firms by political units, and (3) noncompetitive, nondemocratic politics} \citep[p. 37]{Roemer_1994}.} \citep[p. 37]{Roemer_1994}.
    
\end{quote}
\vspace{3mm}


A la par de esto, el autor plantea que existieron tres instancias de problemas de principal-agente: a) entre los administradores de las empresas y los trabajadores, b) entre los planificadores del gobierno y los administradores de las empresas, y c) entre la población en general y los planificadores. En conjunto, estas características y problemas, ocasionaron que no existiera un proceso de innovación que impulsara el crecimiento económico luego de que la Unión Soviética alcanzara cierto nivel de desarrollo productivo.



A continuación se presenta un resumen de las ideas generales del modelo de socialismo de mercado propuesto por \citet[§8]{Roemer_1994}. En esta exposición, el autor propone comparar los posibles resultados, en términos de bienestar de la población, de su modelo con los resultados de una economía capitalista. Para dar cuenta de esto, Roemer propone una descripción general que es igual tanto para el caso capitalista como para el caso del socialismo de mercado. Lo que pretende hacer es analizar cómo impacta el cambio del mecanismo económico sobre los resultados.

Como es habitual en los modelos económicos, primero es necesario explicitar los supuestos simplificadores subyacentes. En el modelo se tiene una economía que produce un solo bien que todas las personas desean consumir. Existe un \say{mal público}\footnote{Este concepto se encuentra estrechamente asociado a la idea de bien público. Un bien público es aquel cuyo consumo es indivisible y no excluible a ningún miembro de la comunidad. El concepto de \say{mal público} habla de algún fenómeno que nadie desea experimentar o consumir pero resulta inevitable.}, que puede ser visto como la contaminación, que se produce conjuntamente con el bien. En esta economía, hay un número de ciudadanos de los cuales una pequeña fracción son inicialmente ricos mientras que una gran proporción son pobres. Todos los ciudadanos poseen las mismas preferencias a lo largo del tiempo respecto del bien de consumo y del mal público. Se dice que el bienestar de los ciudadanos es creciente con el nivel de consumo del bien y decreciente en el nivel de consumo del mal público. El bienestar de los ciudadanos considerados de manera individual, se captura a través de una función de utilidad $u(x_0,x_2,z)$, en donde $x_0$ es el consumo en $t=0$, $x_2$ es el consumo en $t=2$ y $z$ es el \say{consumo} del mal público en $t=2$.

De manera muy esquemática y resumida, dado que el objetivo es captar las intuiciones generales, la puesta en marcha del modelo funcionaría de la siguiente manera\footnote{En el siguiente párrafo sigo a \citet[p. 62]{Roemer_1994}.}. Existen tres momentos temporales, $t=\{0,1,2\}$. En el momento $t=0$, se supone que los ciudadanos conocen la probabilidad de los posibles estados de la naturaleza que pueden ocurrir en $t=2$. Esta incertidumbre afecta a la producción del bien por parte de las empresas. A su vez, en $t=0$, los ciudadanos poseen una cantidad del bien (razón por la cual algunos son ricos y otros pobres) y una parte o participación per cápita igual de cada empresa. Esto quiere decir que los ciudadanos son \say{dueños} de una parte, que es igual para todos, de todas las empresas de la economía. En el momento inicial, es decir, en $t=0$, las personas realizan sus decisiones de consumo e inversión. Luego, en $t=1$, los ciudadanos votan para determinar el nivel del mal público que será permitido (por ejemplo, el nivel de contaminación permitido para las empresas). Finalmente, en $t=2$, ocurre uno de los estados de la naturaleza, se realiza la producción y el producto de las empresas es consumido por los ciudadanos en función de las decisiones de inversión tomadas en $t=0$. En este modelo se supone que cada agente realiza sus planes con el objetivo de maximizar su utilidad esperada.

Como se menciona anteriormente, estos supuestos y la puesta en marcha descrita son iguales en el caso capitalista como en el socialismo de mercado. El siguiente paso es analizar lo que sucede en los distintos escenarios. En primer lugar, en el caso capitalista, se supone la existencia de un mercado de las acciones de las empresas que están en mano de los ciudadanos. Los precios de las acciones se expresan en términos del bien de consumo. En este contexto, una persona puede elaborar un portafolio de acciones utilizando las acciones que posee al comenzar como también los bienes que tiene al principio, es decir, uno puede comerciar acciones por otras acciones o bienes por acciones. Ante esta situación, lo que se espera es que: 

\vspace{3mm}
\begin{quote}
   [...] los pobres venderán buena parte de su dotación inicial de acciones de las empresas a los ricos, quienes las pagarán con el bien, que los pobres consumirán en el momento 0. Esto concentrará la propiedad de las acciones en manos de los ricos, con dos efectos: los ricos serán el grupo que tiene el control en la mayoría de las empresas y, por lo tanto, las elecciones de inversión de las empresas serán de su interés, y los ricos tendrán un mayor interés que los pobres en permitir un alto nivel del mal público\footnote{\say{[...] the poor will sell a good deal of their initial endowment of firm shares to the rich, who shall pay for them with the good, which the poor shall consume at date 0. This will concentrate the ownership of stock in the hands of the rich, with two effects: the rich will be the controlling group in most firms, and hence the firms' investment choices will be in their interest, and the rich will have a greater interest than the poor in permitting a high level of the public bad} \citep[p. 68]{Roemer_1994}.} \citep[p. 68]{Roemer_1994}.
\end{quote}
\vspace{3mm}

Por otro lado, en el caso del socialismo de mercado existe una diferencia en el mercado de las acciones: no es posible comprar acciones con el bien de consumo, solamente es posible comprar acciones utilizando cupones: \say{Cada ciudadano comienza con una dotación del bien, como antes, y, digamos, 1000 cupones, impresos por el gobierno}\footnote{\say{Each citizen begins with an endowment of the good, as before, and, say, 1000 coupons, printed by the goverment} \citep[p. 66]{Roemer_1994}.} \citep[p. 66]{Roemer_1994}. En este mercado de acciones modificado, los precios de las acciones se expresan exclusivamente en términos de los cupones y es ilegal comerciar cupones por bienes. En este escenario, se espera que los pobres controlen las empresas ya que poseen la mayoría de los cupones. Debido a esto, las decisiones de inversión van a estar en función de los intereses de los pobres y el nivel del mal público producido va a ser menor que en el caso anterior. Esto ocurre debido a que, como la propiedad de las empresas se encuentra distribuida entre muchas personas, no hay tantos incentivos a permitir altos niveles del mal público. En el caso de la economía capitalista, como una pequeña fracción de la población controlaba a las empresas, estos se veían beneficiados en mayor medida por aumentar la producción y permitir altos niveles de contaminación.

En conclusión, bajo el esquema del socialismo de mercado, se previene \say{el problema de \textit{free-rider} que afecta a los pobres bajo el capitalismo}\footnote{\say{[...] prevents the free rider problem that afflicts the poor under capitalism} \citep[p. 73]{Roemer_1994}.} \citep[p. 73]{Roemer_1994}. En la versión capitalista del modelo, los pobres encontraban que lo mejor para ellos, individualmente, consistía en vender sus acciones, lo cual concentraba el control de las empresas en los ciudadanos ricos. Por lo tanto, dado un conjunto de parametrizaciones del modelo, existen casos en donde el bienestar de los pobres es mayor y mantienen el control de la mayoría de las empresas. 

En este modelo, lo importante resulta la asignación de los derechos de propiedad sobre algunos recursos importantes para el desarrollo de las personas. Este tipo de modelo es lo que \citet{Cohen_2014c} tenía en mente a la hora de presentar su crítica. Se puede decir que son modelos \textit{segundos-mejores} ya que logran una mejora en comparación con los arreglos institucionales de las economías capitalistas. Sin embargo, nada se dice en los supuestos del modelo sobre el tipo de comportamiento que deban realizar las personas. En efecto, todo parece indicar que nos encontramos en una situación tradicional de los modelos económicos: personas egoístas racionales. Por esta razón, se podría argumentar que este modelo se queda corto a la hora de plasmar los ideales de comunidad e igualdad defendidos desde la visión de Cohen (\citeyear{Cohen_2014b}, \citeyear{Cohen_2014c}). Por lo tanto, a continuación se presenta un refinamiento a la hora de formular este tipo de modelos. En una palabra, se trata de operacionalizar el concepto de \textit{ethos} propuesto por Cohen.


\subsection{La noción de Optimización \textit{Kantiana}}

En sus trabajos más recientes, por ejemplo en \citet{Roemer_2017}, se sugiere que en sus desarrollos teóricos previos se carece de un enfoque basado en el \textit{ethos}. Tal es el caso del modelo de socialismo de mercado analizado anteriormente. En dicho modelo se hace especial énfasis en las cuestiones de los derechos de propiedad y se asumía que las personas se comportaban de manera egoísta. 

El autor reconoce el punto de \citet{Cohen_2014c} de que el problema principal de llevar adelante los ideales socialistas, es un problema de diseño. En este sentido, Roemer considera otros tipos de conductas que pueden tener los agentes como forma de idear nuevos modelos de socialismo de mercado. Con este objetivo en mente, es que se logra incoporar el concepto de \textit{ethos}. Los trabajos de Roemer más recientes giran en torno a la idea de modelizar la manera en que las personas cooperan en contextos económicos. La idea del \textit{ethos} es vinculada estrechamente con el concepto de cooperación. 

La forma de modelizar la cooperación que propone Roemer (\citeyear{Roemer_2019}, \citeyear{Roemer_2021a}), se presenta como una alternativa a la idea de equilibrio de Nash en la teoría de juegos. En particular, \citet{Roemer_2019} sostiene que la teoría económica se ha centrado en analizar cómo los agentes compiten entre si en los mercados y en los juegos. Sin embargo, la competencia no es el único comportamiento que las personas exhiben en estas situaciones. En diversos contextos económicos, las personas logran cooperar. 

Desde la teoría de juegos se ha intentado dar una explicación al comportamiento cooperativo a través de \textit{multistage games}\footnote{Estos juegos consisten en varias etapas. Supongamos que tenemos dos períodos de tiempo: $t \in \{1,2\}$, en $t=1$ se juega el juego $G_1$ mientras que en $t=2$ se juega $G_2$. En este contexto pueden surgir estrategias que sean cooperativas ya que los agentes pueden jugar de una manera que se puede resumir de la siguiente forma: \say{En el primer juego, juego la acción que implica cooperar. Si los demás jugadores cooperan, entonces coopero en el segundo juego, en otro caso juego la acción de no cooperar}. Este tipo de estrategias pueden llegar a ser equilibrios de Nash perfectos por subjuegos (Véase \citet[cap. 9]{Tadelis_2013}).} o juegos repetidos infinitas veces\footnote{Estos juegos consisten en jugar de manera infinita el mismo juego. En este caso, estrategias que impliquen cooperar siempre y cuando la historia del juego sea exclusivamente de cooperación, pueden llegar a ser equilibrios de Nash perfectos por subjuegos (Véase \citet[cap. 10]{Tadelis_2013}).}. Desde la economía comportamental se ha rechazado este tipo de explicación y se propone que la manera de explicar el comportamiento cooperativo es abandonando la idea de que las personas tienen preferencias exclusivamente auto-interesadas. Ahora bien, Roemer plantea que ambas explicaciones resultan insatisfactorias ya que las mismas comparten el uso de la noción de equilibrio de Nash. Para alcanzar un equilibrio de Nash, un jugador razona considerando a todas las acciones que pueden tomar los demás jugadores como parte de su entorno. Cuando un agente optimiza utilizando este protocolo se pregunta: si considero a las acciones de los demás como algo fijo, ¿cuál es la acción o estrategia que maximiza mi utilidad? 

El autor sostiene que un modelo de cooperación debería mostrar la forma en que cada individuo contempla cómo va a coordinar con los demás, es decir, los demás jugadores no deben ser vistos como parte de su entorno sino como \textit{parte de su acción}. En este sentido, Roemer plantea que se podría explicar el comportamiento cooperativo de una manera más intuitiva si se asume que las personas optimizan de otra forma en algunos contextos: \say{En juegos simples (simétricos), un jugador en una situación cooperativa se pregunta, «¿Cuál es la estrategia que me gustaría que todos juguemos?»}\footnote{\say{In simple games (symmetric ones), a player in a cooperative situation asks himself, «What is the strategy I would like all of us to play?»} \citep[p. viii]{Roemer_2019}} \citep[p. viii]{Roemer_2019}. Por este motivo es que el autor denomina a este proceso como \textit{Optimización Kantiana} ya que el tipo de razonamiento que se emplea puede considerarse como una interpretación del imperativo categórico o hipotético de Kant: \say{realiza aquellas acciones que querrías que fueran universalizables}\footnote{Esta es la interpretación que realiza Roemer de los imperativos kantianos. Sin embargo, en el caso de Kant, estos imperativos se encuentran dirigidos hacia máximas que puedan conformar leyes universalizables. En este sentido, es posible argumentar que Roemer se suma a la corriente de economistas que extraen las nociones kantianas hacia el campo de la economía de una manera no tan rigurosa.}. 

Con esto en mente, es posible elaborar una tipología de los modelos a tener en cuenta:

%\vspace{3mm}


\begin{table}[H]

\caption{Una tipología de los modelos}
%\resizebox{14cm}{!}{
\begin{tabular}{cccccc}
                      &                      & \multicolumn{4}{c}{\textbf{Preferencias}}                                                                                                                                                                                                                        \\ \cline{3-6} 
                      &                      &                      & \textbf{Auto-interesadas}                                                                                        &                      & \textbf{Altruistas - Complejas}                                                                 \\ \cline{3-6} 
                      &                      & \textbf{Nash}        & Modelo clásico                                                                                                   &                      & \begin{tabular}[c]{@{}c@{}}Economía \\ comportamental\end{tabular}                              \\
\textbf{Optimización} &                      &                      &                                                                                                                  &                      &                                                                                                 \\
                      &                      & \textbf{Kantiana}    & \begin{tabular}[c]{@{}c@{}}En la mayoría de\\ los capítulos de\\ How We Cooperate \\ (Roemer, 2019)\end{tabular} &                      & \begin{tabular}[c]{@{}c@{}}En el capítulo 5 de\\ How We Cooperate\\ (Roemer, 2019)\end{tabular} \\ \cline{3-6} 
\multicolumn{1}{l}{}  & \multicolumn{1}{l}{} & \multicolumn{1}{l}{} & \multicolumn{1}{l}{}                                                                                             & \multicolumn{1}{l}{} & \multicolumn{1}{l}{}                                                                            \\
\multicolumn{6}{l}{Fuente: adaptado de \citet[p.39]{Roemer_2019}}                                                                                                                                                                                                                       
\end{tabular}
\end{table}


Como punto de partida, \citet{Roemer_2019} alude a trabajos del campo de la psicología evolutiva, como los trabajos de \citeauthor{Tomasello_2014a} (\citeyear{Tomasello_2014a}, \citeyear{Tomasello_2014b}, \citeyear{Tomasello_2016}). Desde estos trabajos de la psicología evolutiva se argumenta que los seres humanos son una especie cooperativa ya que poseen tres características distintivas respecto de otros primates: a) los humanos son los únicos con esclera, b) los humanos pueden señalar e imitar y c) los humanos cuentan con el lenguaje.

Ahora bien, resulta necesario esclarecer el concepto de cooperación que se maneja en este contexto. \citet{Roemer_2019} entiende a la cooperación como el trabajo conjunto de personas para conseguir un fin en común. Bajo esta definición, las personas pueden no tener un interés por los demás, los une el fin en común. Estrechamente relacionado al concepto de cooperación, se encuentra el de solidaridad. La solidaridad puede ser entendida como la unión de propósitos, simpatías o intereses entre miembros de un grupo. Este concepto es una caracterización de la situación objetiva de las personas, todas se encuentran en una situación similar. El punto del autor es que \say{la capacidad de cooperar por motivos de interés personal es menos exigente que la prescripción de preocuparse por los demás}\footnote{\say{[...] is that the ability to cooperate for reason of self-interest is less demanding than the prescription to care about others} \citep[p. 5]{Roemer_2019}.} \citep[p. 5]{Roemer_2019}. De todas formas, las condiciones necesarias para la cooperación son la solidaridad y la confianza entre las personas involucradas.

Dicho esto, para dar cuenta del concepto de Optimización Kantiana partimos del análisis de un ejemplo sencillo, un juego simétrico, un ejemplo del dilema del prisionero:


\vspace{3mm}
\begin{table}[H]
\caption{Un ejemplo del dilema del prisionero (extraído de Roemer, \citeyear{Roemer_2019}).}
\centering
\begin{tabular}{cccc}
\hline
           & \textbf{A} & \textbf{} & \textbf{B} \\ \hline
\textbf{A} & (1,1)      &           & (-1, 2)    \\
\textbf{}  &            &           &            \\
\textbf{B} & (2, -1)    &           & (0,0)      \\ \hline
\end{tabular}
\end{table}
\vspace{3mm}


En este caso contamos con dos jugadores: un jugador fila y un jugador columna. Ambos jugadores tienen el mismo espacio de estrategias, es decir, pueden jugar A o B. Los pagos para el jugador fila corresponden al primer número dentro de las intersecciones de las estrategias. Por ejemplo, si el jugador fila juega A y el jugador columna juega B, entonces los pagos son de 2 para el jugador fila y -1 para el jugador columna. Un juego juego simétrico es aquel en donde los jugadores están situados de manera idéntica: poseen el mismo espacio de estrategias y los pagos son los mismos. En este caso las estrategias posibles son: jugar $A$, jugar $B$ o jugar una estrategia mixta que le asigna una probabilidad positiva $\alpha$ a jugar A y una probabilidad $1-\alpha$ a jugar B. Los pagos de la diagonal principal de la matriz son los mismos para ambos $(1,1)$ y $(0,0)$ y en los restantes casos los pagos se invierten.

Desde la óptica de un agente que emplea el protocolo de Nash, la estrategia $B$ es una estrategia estrictamente dominante ya que los pagos que recibe al jugar dicha estrategia siempre son mayores a los podría recibir si jugara $A$. Como cada jugador hace el mismo razonamiento, el equilibrio de Nash de este juego es el perfil de estrategias $(B,B)$. Por otro lado, un agente que optimizara de manera kantiana, se preguntaría: ¿Cuál es la estrategia que quisiera que ambos juguemos? La respuesta es la estrategia $A$ ya que ambos jugadores van a estar mejor que si los dos jugaran $B$. Llegamos así, a una definición: \say{En un juego simétrico, la estrategia que cada uno preferiría que todos jugaran es un equilibrio kantiano simple}\footnote{\say{In a symmetric game, the strategy that each would prefer all to play is a simple Kantian equilibrium} \citep[p. 13]{Roemer_2019}.} \citep[p. 13]{Roemer_2019}. Denominado SKE (\textit{Simple Kantian Equilibrium)} por sus siglas en inglés. Es posible brindarle más contenido al razonamiento que emplean los jugadores para arribar a los equilibrios kantianos:

\vspace{3mm}
\begin{quote}
   [...] debido a la simetría del juego, asumo que cualquier estrategia que yo decida también la decidirá mi oponente. De ello se deduce que solo debo considerar los perfiles de estrategia $(x,x)$ como aquellos que pueden ocurrir, donde $x \in \{A,B\}$. Por lo tanto, debo elegir la estrategia $x$ que maximice mi pago, si mi oponente y yo jugamos $(x,x)$. Esa es la estrategia $A$. Mi oponente elegirá la misma acción, porque también razonará de esta manera, y de esto estoy seguro, debido al supuesto de conocimiento común y nuestros poderes de razonamiento iguales\footnote{\say{[...] due to symmetry of the game, I assume that whatever strategy I decide upon will also be decided upon by my opponent. It follows that I must only consider strategy profiles $(x,x)$ as ones that migth occur, where $x \in \{A,B\}$. I therefore should choose the strategy $x$ that maximizes my payoff, if $(x,x)$ is played by my opponent and me. That is strategy $A$. My opponent will choose the same action, because he will reason this way as well, and of this I am confident, because of the common-knowledge assumption and out equal reasoning powers} \citep[p. 19]{Roemer_2019}.} \citep[p. 19]{Roemer_2019}.
\end{quote}
\vspace{3mm}

En este punto, cabe destacar la importancia del concepto de simetría entre los jugadores con los ideales de comunidad e igualdad planteados por \citet{Cohen_2014c}. En particular, la idea de que, para que dichos ideales socialistas se mantengan, tiene que fomentarse una preocupación de los individuos por los demás. Resulta poco plausible que surja este tipo de preocupación en un contexto de amplias desigualdades. Concretamente, ya que a los individuos les resultaría más difícil comprender las privaciones que experimentan los demás. Una situación de simetría permite superar dicha dificultad.

Ahora bien, un defensor de la teoría de juegos y de la noción del equilibrio de Nash, argumentaría que el razonamiento presentado anteriormente es irracional. \citet{Roemer_2019} alega que no es el caso: desde la teoría de juegos se ha tendido a desarrollar el concepto de racionalidad de una manera muy estrecha como la maximización individual de utilidad. Emplear un razonamiento que busque lograr la coordinación entre las personas no sería irracional e incluso podría llevar a resultados que son mejoras de Pareto en comparación a los resultados que se llegan empleando la noción de equilibrio de Nash\footnote{En este punto existe cierta similaridad con los conceptos desarrollados por \citeauthor{Rawls_1971} (\citeyear{Rawls_1971}, \citeyear{Rawls_2002}) respecto de la razonabilidad y la racionalidad (Véase Sección \ref{sec2.3})}.




Es posible evidenciar algunos aspectos deseables de emplear la optimización kantiana si nos centramos en dos casos concretos: \textit{la tragedia de los comunes} y la contribución a los bienes públicos o problema del \textit{free-rider}. El caso de \textit{la tragedia de los comunes}, presentado por \citet{Hardin_1968}, puede ser visto como un juego en el cual la acción de los demás jugadores genera una externalidad negativa que afecta a todos. En este contexto, si pensamos desde una optimización según el protocolo de Nash, se llega a una situación en donde se sobreexplota un recurso al cual una comunidad tiene acceso. En cambio, desde la optimización kantiana, al incorporar al resto de los individuos a la hora de tomar una decisión, se logra llegar a una situación se explota en menor medida el recurso y se alcanza la eficiencia de Pareto.

Por otro lado, los problemas de \textit{free-rider} se pueden originar a la hora de querer financiar bienes públicos. En este caso, el problema puede presentarse de la siguiente manera: dado que nadie puede excluirme del uso del bien público una vez que se produce, entonces no tengo ninguna razón para aportar a la producción de dicho bien. Si todos piensan de esa manera o un número importante de personas, los bienes públicos pueden no producirse o producirse en una magnitud menor. Esto es lo que predice la teoría de juegos estándar como equilibrio de Nash en esta situación. Ahora bien, desde la optimización kantiana, se tiene en cuenta la externalidad positiva que generaría la contribución al bien público. Lo que se espera es que cada persona contribuya hasta llegar a un nivel del bien público que sea Pareto eficiente.

Habiendo presentando este esquema de análisis, cabe preguntarse cómo podría influir en los resultados concebir a las personas siendo altruistas. \citet{Roemer_2019} implementa una preocupación altruista en los individuos a partir de incorporar un argumento adicional en las preferencias. Dicho argumento es una ponderación positiva de una función de bienestar social, es decir, los individuos se preocupan tanto por su utilidad personal como la utilidad del conjunto de la sociedad. La idea de incorporar una preocupación altruista en los individuos podría ser una forma adecuada de mostrar un interés por parte de Roemer por las cuestiones esbozadas por \citet{Cohen_2014c}. En particular, el altruismo podría ser una forma en la cual se fortalezcan los ideales de comunidad e igualdad. 

Volviendo a Roemer, el autor enuncia: \say{los equilibrios kantianos para una economía con un grado positivo de altruismo, con respecto a una regla de asignación, son idénticos a los equilibrios kantianos para la economía asociada con preferencias puramente egoístas}\footnote{\say{[...] the kantian equilibria for an economy with a positive degree of altruism, with respect to an allocation rule, are identical to the kantian equilibria for the associated economy with purely self-regarding preferences} \citep[p. 85]{Roemer_2019}.} \citep[p. 85]{Roemer_2019}. Lo que quiere decir el autor con esto es que no podemos distinguir observacionalmente entre dos situaciones en donde en un caso las personas posean un grado de altruismo y en otro se encuentren optimizando de manera kantiana auto-interesada: \say{Si una comunidad ha aprendido a cooperar en el sentido de emplear la optimización kantiana, no podemos decir observando el equilibrio si sus miembros tienen preferencias altruistas o no -al menos, con el altruismo modelado de esta manera}\footnote{\say{If a community has learned to cooperate in the sense of employing kantian optimizacion, we cannot tell by observing the equilibrium whether its members hold altruistic preferences of not -at least, with altruism modeled in this way} \citep[p. 86]{Roemer_2019}.} \citep[p. 86]{Roemer_2019}.

Otro punto de controversia dentro de esta teoría es la posibilidad de concebir a la optimización kantiana como meros equilibrios de Nash en donde las personas toman en cuenta la asignación total de los pagos y no exclusivamente sus pagos individuales. Básicamente, la pregunta es si a partir de un equilibrio kantiano con unas preferencias auto-interesadas se puede llegar al mismo resultado a traves de un equilibrio de Nash con otro tipo de preferencias. La respuesta puede parece un insatisfactoria ya que es afirmativa. El punto de Roemer es que resulta poco creíble que los agentes realicen dicha transformación ya que la misma puede llegar a ser muy compleja. En este caso, Roemer parece estar apelando a un principio de parsimonia a la hora de adoptar el protocolo kantiano como explicación del comportamiento cooperativo.

Habiendo presentado el esquema conceptual detrás de la optimización kantiana y su aplicación a ciertos contextos, en gran parte juegos, a continuación se desarrolla un modelo de socialismo de mercado que incorpora este protocolo a la hora de resolver el problema de la asignación de recursos. En particular, se toma como referencia el trabajo de \citet{Roemer_2021a}. 

\subsection{Socialismo de mercado como una economía cooperativa}

Según \citet[p. 572]{Roemer_2021a}, cada sistema socio económico tiene tres pilares: \say{un ethos de comportamiento económico, una ética de justicia distributiva y un conjunto de relaciones de propiedad que (se espera) implementarán la ética si se sigue el ethos conductual}\footnote{\say{[...] an ethos of economic behavior, an ethic of distributive justice, and a set of property relation that will (it is hoped) implement the ethic if the behavioral ethos is followed} \citep[p. 572]{Roemer_2021a}.}. A la hora de describir la forma que tienen estos pilares dentro de un sistema socialista, el autor plantea que el \textit{ethos} correspondiente es el de la cooperación. Respecto de la manera de concebir la ética distributiva, Roemer invoca la noción de \textit{igualdad de oportunidades socialista} desarrollada por \citet{Cohen_2014c}. En la misma linea argumental que en trabajos previos, como \citet{Roemer_1994}, las relaciones de propiedad o derechos de propiedad se diseñan en función de los objetivos perseguidos. Dichos objetivos son: \say{implementar la igualdad de oportunidades socialista, en la medida en que esto sea posible en una economía de mercado, y a reflejar el espíritu cooperativo del comportamiento económico}\footnote{\say{[...] implement socialist equality of opportunity, so far as this is possible in a market economy, and to reflect the cooperative ethos of economic behavior} \citep[p. 572-573]{Roemer_2021a}.} \citep[p. 572-573]{Roemer_2021a}.

En contraposición al socialismo, se caracterizan los tres pilares del capitalismo. En primer lugar, el \textit{ethos} comportamental es individualista. Desde la noción de equilibrio de Nash, las interacciones dentro de los juegos o mercados conciben al resto de las personas como meros componentes fijos dentro de su entorno. En segundo lugar, la ética distributiva es la del \textit{laissez-faire}. Cada uno por su cuenta y según su capacidad debe conseguir los bienes que desea. No se cuestionan los resultados desiguales que se pueden seguir de una lotería natural arbitraria de los talentos. Finalmente, los derechos de propiedad, en gran medida, son asignados de manera privada.

El objetivo de \citet{Roemer_2021a} es presentar dos modelos de socialismo que incorporan la idea del \textit{ethos} cooperativo bajo la forma del protocolo de optimización kantiana. El primer modelo, denominado \textit{Socialismo 1}, consiste en una socialdemocracia; mientras que el segundo modelo, denominado \textit{Socialismo 2} es una \textit{sharing economy}. La principal diferencia entre estos modelos es que en el segundo caso los beneficios producidos por las empresas se distribuyen entre los agentes que contribuyen en la producción, es decir, aquellos agentes que suministran trabajo o capital a la empresa. En el primer modelo, los beneficios pertenecen a los dueños de las firmas.


Para dar cuenta de algunas características de los modelos de socialismo, se comienza a partir un modelo de equilibrio general clásico. Nos encontramos en una economía en donde se produce un único bien. Existe una firma con una función de producción \say{bien comportada}\footnote{Bajo esta expresión se denota que la función de producción es creciente, diferenciable y cóncava.} que emplea trabajo y capital. La sociedad está conformada por $n$ individuos. Las personas poseen unas preferencias que se pueden representar por una función de utilidad $u$ que tiene como argumentos: el consumo del bien $x_j$ que afecta positivamente a la utilidad y la cantidad de trabajo que ofrece $L_j$ que afecta negativamente a la utilidad. Algunos individuos poseen dotaciones de capital positivas y son dueños de las acciones de las empresas. Los precios relevantes para esta economía son: el precio del bien de consumo $(p)$, el salario $(w)$ y la remuneración al capital $(r)$. 

Dicho todo esto, es posible establecer que un equilibrio en esta economía consiste en 1) un vector de los precios $(p,w,r)$, 2) unos valores de la demanda de trabajo y capital, y 3) unos valores de la oferta de bienes de consumo, trabajo y capital; tales que: la empresa existente se encuentra maximizando sus beneficios, los individuos ofertan la cantidad de trabajo que maximiza su utilidad y los mercados se vacían, es decir, la oferta es igual a la demanda en todos los mercados. Las condiciones de primer orden de este problema de maximización implican que los factores productivos son remunerados exactamente en la misma proporción que su contribución marginal al producto.

De este equilibrio, \citet{Roemer_2021a} destaca dos características positivas: a) por el primer teorema de la economía del bienestar sabemos que este equilibrio es Pareto eficiente; b) el sistema de precios descentraliza la asignación competitiva, es decir, las empresas solamente necesitan conocer los precios y su función de producción mientras que los consumidores solo necesitan conocer los precios y sus preferencias. Como se menciona anteriormente (Véase Sección \ref{sec2.2}) esta distribución es cuestionable desde una perpesctiva igualitaria. En particular, debido a  que los trabajadores e inversionistas reciben una remuneracion acorde a su contribución marginal y los beneficios son apropiados enteramente por los dueños de la empresa.

Supongamos ahora que se quiere implementar un impuesto que busque redistribuir el excedente de manera más equitativa. El resultado que predice la teoría es que el nuevo equilibrio no va a resultar Pareto eficiente, es decir, se va a llegar a una situación en la cual, entre otras cosas, el nivel de bienes que dispone la economía va a ser menor. El punto que Roemer quiere destacar es que esta pérdida de eficiencia se debe a que los individuos maximizan empleando el protocolo de Nash. La contribución del autor es la idea de que la optimización kantiana permite superar el \textit{trade-off} entre eficiencia y equidad.

Para evidenciar esto, es necesario introducir dos conceptos nuevos: a) equilibrio kantiano aditivo y b) equilibrio kantiano multiplicativo (\citeauthor{Roemer_2019}, \citeyear{Roemer_2019}, \citeyear{Roemer_2021a}). Nos situamos en el contexto de los juegos en donde las estrategias de los jugadores consisten en su nivel de \say{contribución} o \say{esfuerzo}: $E^i$. A su vez, el autor se enfoca en aquellos juegos en donde la función de pagos $V$, es estrictamente creciente o decreciente en función de $E^i$.  La noción de equilibrio kantiano aditivo da cuenta de una situación en donde ningún agente tiene incentivos a modificar su contribución mediante la suma de un escalar positivo. Específicamente, si su contribución es $E^i$, entonces no tiene incentivos a jugar $E^i + j$ con $j \neq 0$\footnote{De manera formal: un perfil de estrategias $(E^1,..., E^n)$, con cada $E^i \in I$, siendo $I$ el espacio de estrategias, es un equilibrio kantiano aditivo si $\forall_i (0 = \arg_{\{p|(E^i + p) \in I\}} \max V^i (E^1 + p, E^2 + p, ... E^n + p))$. Véase \citet[p. 578]{Roemer_2021a}). Básicamente, cero es el valor del argumento que maximiza la función de pagos, es decir, nadie tiene incentivos a cambiar su contribución.}. Por otro lado, la noción de equilibrio kantiano multiplicativo es similar a la aditiva solo que en este caso nadie tiene incentivos a modificar su contribución mediante el factor de un escalar, es decir, si su contribución es $E^i$, entonces no tiene incentivos a jugar $(E^i \times j)$ con $j \neq 1$\footnote{De manera formal: un perfil de estrategias $(E^1,..., E^n)$, con cada $E^i \in I$, siendo $I$ el espacio de estrategias, es un equilibrio kantiano multiplicativo si $\forall_i (1 = \arg_{\{p|(p E^i) \in I\}} \max V^i (p E^1, p E^2, ..., p E^n))$. Véase \citet[p. 578]{Roemer_2021a}. Básicamente, 1 es el valor del argumento que maximiza la función de pagos, es decir, nadie tiene incentivos a cambiar su contribución.}.

Con esto en mente podemos comenzar a analizar el modelo de \textit{Socialismo 1} o socialdemócrata. La descripción de la economía es igual al modelo del equilibrio general tradicional. Una diferencia radica en que en este caso existe un impuesto sobre los ingresos de las personas que persigue un fin redistributivo. A su vez, los trabajadores toman la decisión de cuánto trabajo ofertar, es decir, su contribución o esfuerzo, a través de un proceso de optimización kantiana. La idea es que buscan maximizar su utilidad que depende de los ingresos provenientes de todas las fuentes: salario, transferencias a través de lo que se recauda con el impuesto, ganancias por el capital y excedente en caso de ser propietarios de acciones de la firma. Al tomar la decisión sobre la cantidad de trabajo que ofertan, mediante el protocolo kantiano, logran tener en cuenta cómo cada una de sus decisiones individuales afecta el monto que reciben por las transferencias. 

A partir de lo anterior, \citet{Roemer_2021a} plantea que un equilibrio socialdemócrata consiste en: 1) un vector de precios $(p,w,r)$, 2) unos valores de la demanda de trabajo y capital, y 3) unos valores de la oferta de bienes de consumo, trabajo y capital; tales que: a) la empresa existente se encuentra maximizando sus beneficios; b) la decisión por parte de los trabajadores, sobre cuánto trabajar, conforma un \textit{equilibrio kantiano aditivo}; y c) los mercados se vacían. 

Lo más relevante de este tipo de equilibrio es el resultado que \citet[p. 579]{Roemer_2021a} enuncia y demuestra: si tenemos una asignación que es un equilibrio socialdemócrata, resultante de cualquier tasa impositiva $t \in [0,1]$, entonces es Pareto eficiente. El autor denomina a este resultado como \say{primer teorema de la economía del bienestar en la socialdemocracia}. Este teorema es el que le permite afirmar que se logra superar el \textit{trade-off} entre eficiencia y equidad. Concretamente, debido a que cualquier asignación que sea un equilibrio con un nivel arbitrario de la tasa impositiva va a resultar un equilibrio Pareto eficiente. Esto sucede debido a que, como las personas emplean la optimización kantiana, todas las personas ajustan su oferta laboral de acuerdo al impuesto teniendo en cuenta cómo su decisión va a afectar al resto de las personas a través de las transferencias.

A continuación, se analiza el modelo de \textit{Socialismo 2} o \textit{sharing economy}. Nos situamos en una economía que funciona de manera similar a los dos modelos anteriores. En este caso, no hay un impuesto al ingreso. La principal diferencia de este modelo es que existe un parámetro exógeno $\lambda \in [0,1]$ que representa cómo se divide el excedente entre trabajadores y los dueños de capital. Cuando $\lambda = 1$, los trabajadores se apropian de todo excedente; mientras que si $\lambda = 0$, los inversionistas, es decir, aquellos que aportar el capital para producir, se apropian de todo el excedente. En este modelo, a los dueños de la empresa que produce el bien de consumo no les corresponde ninguna parte del excedente ya que no aportan ningun insumo productivo.

Bajo este esquema, un equilibrio de la economía $\lambda$-\textit{sharing} consiste en: 1) un vector de precios $(p,w,r)$, 2) unos valores de la demanda de trabajo y capital, y 3) unos valores de la oferta de bienes de consumo, trabajo y capital; tales que: a) la empresa existente se encuentra maximizando sus beneficios; b) la decisión por parte de los trabajadores, sobre cuánto trabajar, conforma un \textit{equilibrio kantiano multiplicativo}; y c) los mercados se vacían. Respecto de la eficiencia de este equilibrio, \citet[p. 584]{Roemer_2021a} enuncia y demuestra que cualquier equilibrio $\lambda$-\textit{sharing}, en el cual las personas ofrecen una cantidad positiva de trabajo, es Pareto eficiente.

A través de estos ejemplos, se puede evidenciar cómo a través de la optimización kantiana se logran resultados deseables en términos de eficiencia y equidad. En el modelo de \textit{Socialismo 1}, se logra redistribuir a través de los impuestos; mientras que en el modelo de \textit{Socialismo 2} se redistribuye el excedente entre las personas que aportan a la producción. Ahora bien, el autor discute respecto de si este tipo de comportamiento puede llegar a ser creible o es meramente una curiosidad matemática:

\vspace{3mm}
\begin{quote}
    Los tres requisitos previos necesarios para que un grupo de individuos optimice de manera Kantiana son la comprensión, el deseo y la confianza. La gente debe \textit{entender} que la optimización Kantiana puede conducir a buenas soluciones (eficientes) al problema económico. Deben \textit{desear} cooperar, porque ven su situación como solidaria, es decir, enfrentan un problema económico común (la lucha contra la Naturaleza) cuya solución requerirá de la cooperación. Tercero, cada uno debe \textit{confiar} en que los demás optimizarán de la manera Kantiana si él/ella lo hace, para que los optimizadores a la Nash no se aprovechen de los Kantianos, quienes casi siempre pueden beneficiarse como individuos, al menos a corto plazo, jugando Nash contra la multitud Kantiana\footnote{\say{The three prerequisites necessary for a group of individuals to optimize in the kantian manner are understanding, desire, and trust. People must understand that kantian optimization can lead to good (efficiente) solutions to the economic problem. They must desire to cooperate, because they see their situation as one of solidarity, meaning that they face a common economic problem (the struggle against Nature) whose solution will requiere cooperation. Third, each must trust that others will optimize in the kantian manner if he/she does, so that the Kantians will not be taken advantage of by Nash optimizers, who can almost always benefit as individuals, at least in the short run, by playing Nash against the Kantian crowd} \citep[p. 591]{Roemer_2021a}.} \citep[p. 591]{Roemer_2021a}.
\end{quote}
\vspace{3mm}

En la misma línea, Roemer comparte la opinión de \citet{Cohen_2014c} de que las grandes desigualdades pueden dificultar que surja el deseo de cooperar\footnote{Aunque el punto de \citet{Cohen_2014c} se encuentra más vinculado al surgimiento y mantenimiento del ideal de comunidad, también se podría plantear que el ideal comunitario involucra la cooperación.}. En efecto, \citet[p. 594]{Roemer_2021a} plantea que quizás el modelo de \textit{Socialismo 2} posee \say{la ventaja de promover una estabilidad del ethos en comparación con la socialdemocracia}\footnote{\say{[...] the advantage of promoting ethos stability compared to social democracy} \citet[p. 594]{Roemer_2021a}.}. Esta ventaja surge por el hecho de que en dicho modelo los trabajadores e inversores comparten el excedente productivo. No existe un agente que se apropie del excedente \say{sin aportar nada a la producción}. 

La ventaja de los modelos analizados en este apartado radica en que se vuelve explícita la idea de que para lograr un ideal igualitario hay que tomar en cuenta el comportamiento individual. Se parten de modelos de socialismo de mercado en donde el énfasis estaba centrado exclusivamente en los derechos de propiedad hacia modelos en donde, adicionalmente, se incorpora un \textit{ethos} igualitarista. De todas formas, cabe preguntarse si el modelo esbozado por \citet{Roemer_2021a} logra superar el desafío puesto por Cohen.





\section{Reflexiones finales} \label{sec5}





En el presente trabajo se parte desde un análisis en el marco de la teoría económica neoclásica. En dicho contexto, la presencia de incentivos diferenciales no es problemática sino que es deseable. En particular, las personas más productivas o más talentosas reciben estos beneficios adicionales. A su vez, los incentivos diferenciales conforman uno de los componentes que vuelve eficiente a la asignación de recursos resultante.

Al analizar esta situación, se llega a que dichos resultados son insatisfactorios desde el punto de vista de la justicia distributiva. En particular, como argumentan \citeauthor{Rawls_1971} (\citeyear{Rawls_1971}, \citeyear{Rawls_2002}) y \citet{Roemer_1998}, una distribución de remuneraciones bajo esta lógica ignora cómo las personas llegan a desarrollar sus capacidades. Concretamente, la asignación de recursos resultante es injusta en la medida que se encuentra influenciada en buena parte por resultados arbitrarios consecuencia de la lotería natural.

Por lo cual, el principio de la diferencia (\citeauthor{Rawls_1971}, \citeyear{Rawls_1971}, \citeyear{Rawls_2002}) se propuso como una alternativa. En este caso, los incentivos diferenciales son considerados justos en la medida de que se cumplan ciertas condiciones previas. Específicamente, se debe cumplir el primer principio de justicia el cual garantiza un mismo nivel de libertades básicas para los ciudadados. Adicionalmente, se debe cumplir el segundo principio de justicia que establece una igualdad de oportunidades en la posibilidad de acceder a cargos y posiciones. Si se cumplen estos principios, es posible habilitar al principio de la diferencia. Este principio permite la existencia de desigualdades sociales y económicas siempre y cuando las mismas favorezcan al grupo menos aventajado de la sociedad.

Ahora bien, como argumenta \citet{Cohen_2001}, existe una tensión entre el ideal igualitario y los incentivos diferenciales. El autor sostiene que si las personas se rigen por el principio de la diferencia, no deberían de requerir incentivos diferenciales para realizar un trabajo que beneficie a los menos aventajados. Desde la perspectiva rawlsiana, si consideramos que los principios de justicia han de ser aplicados exclusivamente a la estructura básica de la sociedad, entonces hay un espacio de elecciones individuales que quedan por fuera del ámbito de la justicia. Estas elecciones podrían socavar las ambiciones equitativas de las instituciones que conforman a la estructura básica. Por lo cual, \citet{Cohen_2001} sostiene que es necesario contar un \textit{ethos} igualitario que sea reduzca estos comportamientos contrarios a la igualdad.

En la misma linea, \citet{Cohen_2014c} argumenta que el socialismo cuenta con ideales que apuntan hacia la igualdad. En particular, el autor plantea que los ideales normativos principales del socialismo son la igualdad y la comunidad. Estos ideales se materializan en el concepto de igualdad de oportunidades socialista. Este concepto implica la corrección de las desventajas que no son elegidas por parte de los individuos, es decir, aquellas donde la persona no puede ser considerada racionalmente responsable de ellas.

Con este planteo en mente, Roemer busca desarrollar un modelo de socialismo de mercado que pueda plasmar los ideales de igualdad y comunidad. Para dar cuenta de estos ideales, \citeauthor{Roemer_2019} (\citeyear{Roemer_2019}, \citeyear{Roemer_2021a}) se concentra en el comportamiento cooperativo. En particular, el autor desarrolla el concepto de optimización kantiana, el cual pretende ser una alternativa al modo de razonamiento que se emplea en los equilibrios de Nash. Al emplear este nuevo criterio, se pretende explicar cómo pueden surgir los comportamientos cooperativos. Mediante la optimización kantiana es posible construir modelos de socialismo de mercado en los cuales se supera el \textit{trade-off} entre eficiencia y redistribución.

Una idea que se rescata del análisis previo es que resulta poco razonable confiar en que los derechos de propiedad o los arreglos institucionales formales hagan todo el trabajo a la hora de materializar ciertos ideales de justicia e igualdad. Resulta necesario establecer un \textit{ethos} igualitario que sea parte vital de la construcción de sociedades justas. Por lo cual, esto genera un problema de diseño. En particular, ¿cómo han de diseñarse los arreglos institucionales formales e informales para que se establezca y se mantenga este \textit{ethos}? El desarrollo de Roemer es un aporte invaluable a la pregunta sobre el diseño institucional. En particular, centrarse en la cooperación a través del concepto de \textit{optimización kantiana} es una estrategia prometedora. 

De todas formas, respecto de la \textit{optimización kantiana} cabe hacer unos comentarios. En primer lugar, parece insatisfactorio el resultado de que observacionalmente sería lo mismo que las personas posean o no altruismo desde esta perspectiva. Esto puede ser cierto al menos desde un ámbito enfocado en los resultados dentro del mercado. Por fuera de las esferas de la economía, el altruismo puede ser deseable para sostener el \textit{ethos} igualitario y la idea de comunidad. El ideal de comunidad puede ir mas allá de las interacciones dentro de los mercados. Por lo tanto, una sociedad en la cual las personas muestren cierto grado de altruismo debería reflejar el ideal comunitario del socialismo de mejor manera que una sociedad en la cual las personas poseen preferencias meramente auto-interesadas.

En segundo lugar, quedan sin resolver los problemas que surgen en contextos de información imperfecta. En particular, como se esbozó en la introducción, los problemas de principal-agente como el riesgo moral. En los modelos analizados, dicho problema podría no ocurrir ya que las personas son concebidas de manera diferente a lo que plantea la teoría neoclásica. Concretamente, las personas optimizan empleando un protocolo diferente. Este punto es recogido por \citet{Roemer_2021b}, el autor argumenta que este problema no ha sido tratado con rigurosidad en su trabajo. Sin embargo, una posible respuesta es que los problemas de este tipo no serían tan salientes. En especial, si se toma en cuenta que uno de los requisitos para que las personas empleen el criterio de la \textit{optimización kantiana} es que haya confianza entre las personas.

Por otro lado, queda abierta la pregunta de si los nuevos modelos de socialismo de mercado propuestos por \citet{Roemer_2021a} logran superar la crítica de \citet{Cohen_2014c}. Se puede evidenciar un mayor refinamiento teórico entre el modelo de \citet{Roemer_1994} y los modelos que incorporan la \textit{optimización kantiana}. No cabe duda que el denominado \say{primer teorema de la economía del bienestar en la socialdemocracia} es un resultado interesante ya que permite superar las pérdidas de eficiencias asociadas a la redistribución de ingresos. De todas formas, en estos modelos no se definen las instituciones que permitirían el desarrollo de los ideales de igualdad y comunidad. El rasgo distintivo de estas sociedades es el logro de los objetivos de eficiencia y redistribución. 

El argumento de Roemer podría ser que lo sustancial de este resultado es la forma en que se llega a la eficiencia y la redistribución. En particular, a través de los requisitos necesarios para la \textit{optimización kantiana} (comprensión, deseo y confianza) es que se lograrían establecer los ideales de igualdad y comunidad. Sin embargo, este tipo de comportamiento se desarrolla en las interacciones económicas. Por lo cual, cabe preguntarse cómo actuarían los individuos en otros contextos. 

Finalmente, si bien los modelos de \citeauthor{Roemer_2019} (\citeyear{Roemer_2019}, \citeyear{Roemer_2021a}) resultan sólidos en términos teóricos, cabe preguntarse cuáles serían los pasos a seguir para buscar su implementación. En términos prácticos, se podría argumentar que el primer objetivo sería el de fomentar las precondiciones para la \textit{optimización kantiana} (comprensión, deseo y confianza). Ahora bien, la pregunta es si sería posible llegar a dicha situación partiendo de una sociedad en la cual predomina un \textit{ethos} individualista. En mi opinión creo que dicha transición es posible y conocer cómo realizarla constituye una línea de investigación que se podría desarrollar a futuro.




\newpage




\cleardoublepage
\phantomsection
\addcontentsline{toc}{section}{Referencias}
\bibliography{Bibliography}





\end{document}

