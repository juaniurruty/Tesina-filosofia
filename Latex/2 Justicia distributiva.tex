
El concepto de justicia distributiva es central para el desarrollo de este trabajo ya que es la idea normativa sobre la cual se evaluan las diversas propuestas que se presentan como respuestas al problema de si es posible conciliar incentivos e igualdad. La discusión sobre qué implica la justicia se encuentra presente a lo largo de toda la historia de la filosofía. Sin embargo, en el presente trabajo, se discute a partir de los desarrollos contemporáneos.


\subsection{Nociones básicas} \label{sec2.1}

A la hora de abordar cuestiones de justicia distributiva, existe una concepción que se encuentra ampliamente generalizada. La misma sostiene que los arreglos institucionales que conforman las sociedades implican una distribución de cargas y beneficios entre los miembros de dicha sociedad \citep{Rawls_1971}. La forma que tenga esta distribución es relevante ya que tiene un impacto sobre la vida de las personas. En este sentido, indagar sobre cuestiones de justicia distributiva puede implicar la búsqueda de los principios más adecuados para regular las instituciones encargadas de la distribución de las cargas y beneficios. 

Existen numerables visiones sobre cuáles han de ser los principios rectores. Es posible evidenciar principios que surgen de las siguientes corrientes filosóficas: un igualitarismo estricto, el principio de la diferencia (y el resto de los principios de la teoría rawlsiana), la igualdad de oportunidades, el igualitarismo de la suerte, principios basados en el bienestar de las personas, principios basados en el mérito, principios libertarios, entre otros. A su vez, las cuestiones de justicia distributiva pueden plantearse en distintos ámbitos: a nivel de sociedades individuales o a nivel de relaciones entre diversos países, a nivel de las relación entre las distintas generaciones, a nivel de aspectos asociados con las conductas individuales y entender a la justicia como una virtud, entre otros enfoques.

Este trabajo se centra en la discusión respecto de los principios de justicia que han de regir a una sociedad sin entrar en cómo las distintas sociedades interactúan entre sí. Como punto de partida, \citet[p. 81]{Rawls_2002} argumenta que el problema de la justicia distributiva consiste en responder a la pregunta de: \say{¿cómo han de regularse las instituciones de la estructura básica como un esquema unificado de instituciones para que pueda mantenerse a lo largo del tiempo, de una generación a otra, un sistema equitativo, eficiente y productivo de cooperación social?} Rawls enfatiza en la distinción entre los conceptos de \textit{justicia distributiva} y \textit{justicia asignativa}. La justicia asignativa se centra en la repartición de un conjunto de bienes entre diversas personas; mientras que la justicia distributiva implica más que la asignación debido a que tiene en cuenta, entre otras cosas, al proceso de cooperación social que produce los bienes materiales. 

Esta forma de concebir a la justicia ha sido denominada como el \say{paradigma distributivo de la justicia} \citep{Gosepath_2013}. Según Gosepath, desde el paradigma distributivo se entiende que:

\vspace{3mm}
\begin{quote}
    [...] toda concepción de la justicia debe ser capaz de dar una respuesta fundada a la pregunta acerca de quién debe qué a quién, bajo qué circunstancias, de qué manera, por qué, desde qué perspectiva, según qué principio y con qué modo de aplicación. [...] Según el paradigma distributivo, la tarea material de la justicia es asegurar una distribución equitativa de aquellos bienes que proveen a todos los miembros de la sociedad del estatus de personas libres e iguales, y que dentro de este marco les permiten perseguir sus proyectos elegidos de manera autónoma, siempre y cuando estos proyectos no interfieran con la misma libertad de las demás personas \citep[p. 48]{Gosepath_2013}.
\end{quote}
\vspace{3mm}

Sin embargo, existen diversas críticas a esta visión de la justicia distributiva ya que se entiende que este enfoque deja por fuera cuestiones relevantes. En particular, \citet{Young_2000} argumenta que este paradigma distributivo ignora cómo los bienes son producidos y cómo se determinan las estructuras políticas que influyen sobre su distribución. Adicionalmente, \citet{Forst_2013} sostiene que existen dos imágenes de la justicia y que el paradigma distributivo se ha centrado principalmente en una de ellas. Según el autor, esta imagen no permite evidenciar que los reclamos sobre la posesión de bienes se producen discursivamente en un contexto adecuado de justificiación. En la misma línea, esta visión ignora cuestiones asociadas a la injusticia ya que no permite distinguir entre distintas situaciones de carencias de bienes y los fundamentos morales detrás de las razones por las cuales es relevante hacer algo al respecto de dicha privaciones. Por lo cual, se plantea que:

\vspace{3mm}
\begin{quote}
    Por estas razones, al tratar con cuestiones de justicia distributiva resulta especialmente necesario tener en cuenta el aspecto \textit{político} de la justicia y librarnos de la imagen falsa, centrada únicamente en cantidades de bienes por más importantes que estos sean. En contraste, en una segunda imagen, a mi entender más apropiada, la justicia debe orientarse a \textit{relaciones y estructuras intersubjetivas}, no a \textit{estados subjetivos o putativamente objetivos} de la provisión de bienes \citep[p. 31]{Forst_2013}.
\end{quote}
\vspace{3mm}

Según \citet[p. 49]{Gosepath_2013}, es posible resumir las críticas a esta forma de concebir a la justicia de la siguiente manera: 1) el paradigma distributivo deriva en una reificación de las relaciones sociales y de las normas sociales; 2) el paradigma distributivo concibe a la justicia exclusivamente en pautas de distribución no considerando seriamente a los procesos sociales subyacentes y 3) como consecuencia de lo anterior, esta visión terminaría ignorando los procesos de producción de los bienes a distribuir.

Sin embargo, para el autor es posible defender esta visión de las críticas presentadas. Para responder a estas críticas, \citet{Gosepath_2013} argumenta que es necesario evidenciar tres polémicas sobre: la concepción de los bienes, la concepción de las relaciones sociales y el estatus de la justicia política. En primer lugar, respecto de la concepción de los bienes, \citet{Gosepath_2013} sostiene que es posible distinguir entre bienes estructurales e individuales. Los bienes estructurales son los prerrequisitos institucionales de la vida humana, tales como un marco constitucional en el cual desarrollar los individuos pueden desarrollar sus planes vitales. Mientras que los bienes individuales pueden ser internos (talentos, aptitudes y salud) o externos (alimento, vestimenta y vivienda). Esta conceptualización de los bienes permite evidenciar que los derechos, libertades y oportunidades pueden ser concebidos como bienes en un sentido amplio. Por lo cual,

\vspace{3mm}
\begin{quote}
    Es verdad que uno no \say{posee} derechos y oportunidades en exactamente el mismo sentido en que uno es propietario de bienes materiales; sin embargo, incluso la propiedad, o mejor dicho la posesión, debería ser (también) entendida como un paquete de derechos morales. Esto significa que tanto en el caso de bienes materiales como cuando se trata de derechos y oportunidades, regulamos las pretensiones morales, es decir, hablamos de asignaciones que detallan qué bienes, derechos y oportunidades podemos esperar legítimamente y cómo podemos hacer uso de ellos \citep[p. 51]{Gosepath_2013}.
\end{quote}
\vspace{3mm}

En segundo lugar, respecto de la polémica asociada a la forma de concebir a las relacionales sociales, se argumenta que al expandir el concepto de justicia distributiva hacia bienes no materiales, estos objetos serían tratados como objetos estáticos más que relaciones y procesos sociales. Ante esto, \citet{Gosepath_2013} plantea que dicha objeción sería plausible solamente cuando las concepciones de justicia se centren exclusivamente en el resultado final de la distribución. El autor sostiene que existen razones morales para rechazar dicho tipo de enfoque. En particular, si se consideran las acciones individuales, además del resultado final, también es relevante tener en cuenta el proceso según el cual se logran dichos resultados.

Finalmente, respecto del status de la justicia, \citet{Gosepath_2013} propone que es necesario preguntarse por la justicia de los procesos sociales en sí mismos. Este tipo de cuestionamiento se encuentra vinculado con la tradición del contrato social, en especial, en la propuesta de \citeauthor{Rawls_1971} (\citeyear{Rawls_1971}, \citeyear{Rawls_2002}). Bajo la teoría rawlsiana, ser miembro de una sociedad consiste en un conjunto de derechos y deberes que los ciudadanos libremente se conceden unos a otros. En este sentido, \citet{Gosepath_2013} argumenta que el establecimiento de una sociedad justa es una cuestión de justicia distributiva. Cuando los individuos libres e iguales reflexionan sobre cuál es la mejor forma que debe poseer la estructura básica de la sociedad, se negocian sus cargas y beneficios de la cooperación social. Esta negociación alude a una \say{distribución original} o \say{estado original}.

Como propuesta alternativa a este tipo de teoría procedimentalista, es posible presentar el planteo de \citet{Honneth_2013}. Según este autor, una propuesta reconstructivista de la justicia captura de mejor manera las relaciones intersubjetivas y las relaciones de reconocimiento que se desarrollan a lo largo de la historia. Sin embargo, \citet{Gosepath_2013} sostiene que una teoría reconstructivista no puede establecer un punto de vista imparcial sobre el cual fundamentar los principios de justicia. En particular,

\vspace{3mm}
\begin{quote}
    [...] una teoría \say{reconstructiva} de la justicia de este tipo carece de cualquier tipo de criterio independiente que nos permita evaluar los estándares normativos inmanentes de una práctica histórica contingente. Para establecer tales criterios, necesariamente debemos recurrir a una teoría abstracta e ideal de justicia que funcione con supuestos hipotéticos, incluso cuando una teoría política y crítica de justicia comprensiblemente considere tales abstracciones, idealizaciones e hipostatizaciones como insatisfactorias \citep[p. 55]{Gosepath_2013}.
\end{quote}
\vspace{3mm}

A modo de síntesis, el punto central de \citet{Gosepath_2013} es que la distribución justa no se limita a los bienes sino que puede incluir cosas de vital importancia como pueden ser los derechos, deberes y oportunidades. En esencia, la intuición básica de la idea de justicia es asegurar que ningún individuo se vea excesivamente desfavorecido.

Habiendo definido algunas cuestiones básicas acerca del concepto de justicia distributiva, a continuación se desarrolla una forma de concebir el vínculo entre la filosofía normativa y la teoría económica.

\subsection{Justicia y economía} \label{sec2.2}

Siguiendo a \citet{Sen_1991}, desde sus orígenes la economía se ha vinculado con cuestiones normativas. Concretamente, el autor evidencia dos orígenes de la disciplina con enfoques diferenciados. En primer lugar, el autor argumenta que en los trabajos de \citeauthor{Aristoteles_2005} (\citeyear{Aristoteles_2005}, \citeyear{Aristoteles_2014}) se puede evidenciar una relación estrecha entre la política y la economía. En esta relación se destaca la faceta normativa de la economía. Por otro lado, \citet{Sen_1991} plantea que existe otro enfoque que denomina que asocia a la economía con la ingeniería. En este caso, la economía toma a los fines como algo dado y se cuestiona respecto de la adecuación de los medios materiales para obtener dichos fines. Según \citet{Sen_1991}, si bien conceptualizar a la economía de esta manera ha sido muy productivo en términos de resultados, este enfoque ha sido predominante en detrimento del enfoque normativo. En este sentido, el autor argumenta que la disciplina económica se enriquecería si su faceta normativa es puesta en un primer plano.

Existe un vínculo especial entre los economistas y filósofos a la hora de abordar cuestiones sobre justicia distributiva. Concretamente, los practicantes de las disciplinas mencionadas, proceden de distintas maneras. Según \citet{Roemer_1996a}, muchos economistas abandonan incursionar en cuestiones de filosofía política al encontrarla muy informal o carente de rigor para poder ser manejada. La práctica de la disciplina ha llevado a los economistas a tomar preguntas interesantes y buscar su formalización a través de modelos. Por otro lado, es cierto que los filósofos apreciarían la contribución de los modelos formales. Sin embargo, su interés giraría en torno al proceso intelectual que ha llevado a formularlo. Roemer sostiene que un filósofo se enfocaría en cuestionar los supuestos empleados cuando se encuentra ante algún modelo económico.  

Ahora bien, cuando estos dos enfoques se encuentran al abordar la temática de la justicia distributiva, Roemer argumenta que la relación y el aporte que puede hacer el economista al filósofo tiene la particularidad de que:
\vspace{3mm}

\begin{quote}
    [...] la economía es la sirvienta en esta relación. La forma de pensar del economista puede comprobar la consistencia de una teoría filosófica o proporcionar una formulación concreta (un modelo) para hacer precisa alguna de sus todavía vagas aseveraciones. A menudo puede traducir una visión filosófica sobre la justicia distributiva en una política social concreta [...] Sin embargo, no creo que la forma de pensar del economista haya producido, o producirá alguna vez, nuevos conocimientos importantes sobre lo que es la justicia distributiva\footnote{A lo largo de este trabajo, las traducciones de citas en inglés son propias: \say{[...] economics is the handmaiden in this relationship. The economist's way of thinking can check the consistency of a philosophical theory or provide a concrete formulation (a model) to make precise some of its still vague assertions. It can often translate a philosophical view about distributive justice into a concrete social policy [...] I do not, however, believe that the economist's way of thinking has produced or will ever produce, important new insights into what distributive justice is} \citep[p. 3]{Roemer_1996a}.} \citep[p. 3]{Roemer_1996a}.
\end{quote}
\vspace{3mm}


A la hora de evaluar los postulados de la economía, en este trabajo se hace énfasis en la teoría neoclásica. Cabe destacar que dicha teoría está compuesta por un conjunto de modelos. Existe un aforismo dentro de la disciplina económica y la estadística que dice que \say{todos los modelos están equivocados pero algunos son útiles}\footnote{Aforismo atribuido al estadístico Box, es posible encontrarlo en varios de sus trabajos. En particular, en \citet[p. 202]{Box_1979} una sección de este trabajo se titula: \say{All models are wrong, but some are useful.}}, es decir, los modelos siempre implican una simplificación de la realidad, pero los mismos pueden ser útiles para predecir algunos fenómenos. Esta aclaración sugiere que esta teoría no pretende determinar exhaustivamente los factores que explican las remuneraciones sino establecer un esquema general sobre el cual hacer predicciones. 

Un posible argumento que podría esbozar un partidario de la teoría neoclásica es que la misma no discute la distribución de las remuneraciones ya que, en la situación ideal que plantea el modelo, todos los factores reciben una remuneración en función de su aporte a la producción total. Esto es consecuencia de un postulado principal de dicha teoría. Concretamente, en una situación ideal en donde hay competencia e información perfecta, las remuneraciones de las personas se explican a través de la productividad de las mismas. En particular, \citet{Arrow_1954} emplean una demostración matemática de la existencia de una asignación de recursos que conforma un equilibrio general. En dicha asignación se cumple que: 1) las firmas maximizan sus beneficios, 2) los consumidores maximizan su utilidad y 3) todos los mercados se vacían. A su vez, en este equilibrio cada factor productivo es remunerado en función de su producción marginal.

Ahora bien, este argumento no cierra completamente la posibilidad de analizar las implicancias de esta teoría en términos de justicia distributiva. Es posible presentar dos críticas a este tipo de defensa. En primer lugar, \citet{Rawls_1971} plantea que una sociedad organizada de esta forma puede ser enmarcada en un sistema de libertad natural. Concretamente, bajo este esquema, se plantea que \say{una estructura básica que satisfaga el principio de eficiencia y en la que los puestos estén abiertos a quienes puedan y estén dispuestos a disputar por ellos conducirá a una distribución justa}\footnote{\say{[...] a basic structure satisfying the principle of efficiency and in which positions are open to those able and willing to strive for them will lead to a just distribution} \citep[p. 66]{Rawls_1971}.} \citep[p. 66]{Rawls_1971}.

De todas formas, el sistema de libertad natural posibilita que la distribución de ingreso y riqueza sea influenciada en buena medida por los resultados de la lotería natural. En particular, cuestiones que escapan al control de los individuos como pueden ser su dotación de talentos y el contexto social de nacimiento. En este sentido, \say{la injusticia más evidente del sistema de libertad natural es que permite que las partes que se distribuyen se vean indebidamente influidas por estos factores tan arbitrarios desde el punto de vista moral}\footnote{\say{[...] the most obvious injustice of the system of natural liberty is that it permits distributive shares to be improperly influenced by these factors so arbitrary from a moral point of view} \citep[p. 72]{Rawls_1971}.} \citep[p. 72]{Rawls_1971}. 

De manera similar, \citet{Roemer_1998} argumenta desde la perspectiva de la igualdad de oportunidades que existen factores que influyen sobre las capacidades de las personas que se encuentran por fuera del control de los individuos. Si no se tienen en cuenta las circunstancias en las que se encuentran situados los individuos, resulta evidente que si se aplicara el criterio distributivo que propone la teoría neoclásica se ignorarían cuestiones relevantes para la justicia distributiva. Si los salarios son iguales a la productividad, pero no se compensa a las personas por sus circunstancias arbitrarias, lo que ocurre es que las personas que salieron más beneficiadas de la lotería natural terminarían con remuneraciones elevadas por cuestiones que estaban fuera de su control, es decir, dichos niveles no tendrían una justificación en el esfuerzo o merecimiento.


En segundo lugar, aunque en el modelo de neoclásico se sostenga que cada factor productivo recibe una remuneración acorde a su contribución en el proceso productivo, por lo general existe un excedente que es apropiado por los dueños de las empresas. \citet{Roemer_2021a} argumenta que este ingreso no se explica a través aporte de un factor productivo, sino que se debe a que los empresarios poseen los derechos de propiedad sobre lo que se produce. Por lo cual, es posible establecer que:

\vspace{3mm}
\begin{quote}
    La equidad de esta asignación es cuestionable. ¿No es discutible que los trabajadores y los inversores deberían compartir el excedente que emerge en la producción? La estructura legal del capitalismo asigna los beneficios a los propietarios, pero esto no es necesariamente justo o ético. Es una tradición en la teoría neoclásica decir que los trabajadores no son explotados si reciben salarios iguales a sus productos (valores) marginales. Los marxistas, sin embargo, dicen que los trabajadores que reciben salarios iguales al producto marginal \textit{son} explotados porque no comparten el excedente de la producción\footnote{\say{The fairness of this allocation is questionable. For it is not arguable that workers and investors should share in the surplus that emerges in production? The legal structure of capitalism allocates profits to owners, but that is not necessarily fair or ethical. It is a tradition in neoclassical theory to say that workers are not exploited if they receive wages equal to their marginal (value) products. Marxists, however, say that workers who receive marginal-product wages \textit{are} exploited because they do not share in the surplus from production} \citep[p. 266]{Roemer_2021a}.} \citep[p. 266]{Roemer_2021a}.
\end{quote}
\vspace{3mm}


El esquema desarrollado por la teoría neoclásica se centra primordialmente en el aspecto de la competencia entre las personas y predice que las personas más productivas o talentosas serán las que recibirán las mayores recompensas. El hecho de que algunas formulaciones de la teoría neoclásica no problematizan cómo los individuos llegan a desarrollar sus capacidades y cómo se distribuye el excedente del proceso productivo, es la razón por la cual se vuelve necesario presentar un esquema conceptual alternativo. A continuación, se presenta un esbozo de la teoría rawlsiana como alternativa que busca articular las nociones de eficiencia, incentivos y justicia distributiva. En particular, se esboza un concepción en donde la idea de cooperación social tiene un papel más relevante que la idea de competencia o de simple coordinación entre agentes.


\subsection{Una teoría de la justicia} \label{sec2.3}

El trabajo de \citeauthor{Rawls_1971} (\citeyear{Rawls_1971}, \citeyear{Rawls_2002}) ha sido influyente tanto en el campo de la filosofía normativa como en el campo de las ciencias sociales y dicho impacto ha continuado hasta la actualidad\footnote{Para desarrollar la propuesta del autor se sigue la obra de \textit{A Theory of Justice} (\citeyear{Rawls_1971})  y \textit{La justicia como equidad} (\citeyear{Rawls_2002}).}. Siguiendo a \citeauthor{Rawls_1971} (\citeyear{Rawls_1971}, \citeyear{Rawls_2002}), la justicia es la virtud principal de las instituciones sociales. Para el autor, una sociedad es justa cuando los derechos de las personas no se encuentran sujetos a una negociación política o sujetos a una maximización de utilidad social. En este sentido, el propósito de Rawls es elaborar una teoría de la justicia que logre incorporar las convicciones de las personas respecto de la primacía de la justicia. Para este objetivo, se requieren de principios que determinen cuáles instituciones sociales se prefieren por sobre otras a la hora de determinar la distribución de los recursos. Concretamente, \say{estos principios son los principios de la justicia social: proporcionan una forma de asignar derechos y deberes en las instituciones básicas de la sociedad y definen la distribución adecuada de los beneficios y cargas de la cooperación social}\footnote{\say{These principles are the principles of social justice: they provide a way of assigning rights and duties in the basic institutions of society and they define the appropiate distribution of the benefits and burdens of social cooperation} \citep[p. 4]{Rawls_1971}.} \citep[p. 4]{Rawls_1971}.

Estos principios se enmarcan en sociedades que son concebidas como \say{un sistema equitativo de cooperación social a lo largo del tiempo de una generación a la siguiente} \citep[p. 28]{Rawls_2002}. Esta idea de cooperación social tiene tres rasgos esenciales: a) está guiada por reglas y procedimientos públicamente reconocidos, b) incluye términos equitativos de la cooperación asociados a una idea de reciprocidad y c) los participantes que cooperan promueven su propio bien. Este esquema de cooperación se conforma por participantes que son considerados razonables y racionales. Por un lado, según \cite{Rawls_2002}, las personas razonables están dispuestas a cumplir con los principios necesarios para la cooperación equitativa siempre que los demás estén igualmente dispuestos a honrarlos. Por otro lado, las personas racionales podrían tener motivos basados en el interés personal para no cumplir con dichos principios y obtener una ventaja. 

Una de las razones por las cuales se conciben a las sociedades ideales de esta forma se debe a la pluralidad de concepciones del bien que poseen los ciudadanos. Dado que no es posible llegar a un acuerdo sobre una única visión comprehensiva del mundo, es decir, un acuerdo sobre una idea del bien, los ciudadanos deben llegar a un acuerdo sobre una concepción política de la justicia que regirá la interacción social (\citeauthor{Rawls_1971}, \citeyear{Rawls_1971}, \citeyear{Rawls_2002}). En este acuerdo, se llega a un pluralismo razonable en donde las concepciones del bien que se mantienen en la sociedad no vulneran los derechos de ninguna persona. 

Al concebir a la sociedad como un sistema de cooperación, se presenta la idea de las sociedades bien ordenadas, es decir, sociedades que se encuentran efectivamente reguladas por una concepción pública de la justicia. Este concepto implica una idealización que presenta tres grandes rasgos \citep{Rawls_2002}. En primer lugar, cada persona acepta y sabe que los demás aceptan la misma concepción política de la justicia. En segundo lugar, los ciudadanos tienen buenas razones para creer que las principales instituciones políticas y sociales satisfacen dicha concepción de justicia. Finalmente, en una sociedad bien ordenada, los ciudadanos poseen un sentido de la justicia que \say{los capacita para entender y aplicar los principios públicamente reconocidos de justicia y, en su mayor parte, para actuar según lo exige su posición en la sociedad, con sus deberes y obligaciones} \citep[p. 32]{Rawls_2002}.


Ahora bien, cabe precisar el ámbito para el cual estan diseñados los principios de justicia. En particular, \citeauthor{Rawls_1971} (\citeyear{Rawls_1971}, \citeyear{Rawls_2002}) argumenta que el sujeto primario de la justicia es la estructura básica de la sociedad, es decir, aquellas instituciones de mayor relevancia social que definen la manera en que se distribuyen las cargas y beneficios de la cooperación social. La teoría rawlsiana se enfoca principalmente en este ámbito porque la estructura básica tiene efectos de vital importancia sobre los objetivos, aspiraciones y oportunidades de los ciudadanos. En la misma línea, \citet{Rawls_1971} sostiene que esta discusión sobre los principios que han de regir sobre la estructura básica se vincula con la tradición política del contrato social que se evidencia en autores como Locke, Rousseau y Kant\footnote{En particular, es posible destacar el trabajo de \citet{Locke_1960}, \citet{Rousseau_2007} y \citet{Kant_2012}.}. En este contexto, no se trata de definir un estado de naturaleza sobre el cual se genera un acuerdo original para establecer un gobierno, sino que:

\vspace{3mm}
\begin{quote}
    [...] los principios de justicia para la estructura básica de la sociedad son el objeto del acuerdo original. Son los principios que personas libres y racionales preocupadas por sus propios intereses aceptarían en una posición inicial de igualdad como definidores de los términos fundamentales de su asociación\footnote{\say{[...] the principles of justice for the basic structure of society are the object of the original agreement. They are the principles that free and rational persons concerned to further their own interests would accept in an initial position of equality as defining the fundamental terms of their association} \citep[p. 11]{Rawls_1971}.} \citep[p. 11]{Rawls_1971}.
\end{quote}
\vspace{3mm}

Para dar cuenta de este acuerdo original, \citeauthor{Rawls_1971} (\citeyear{Rawls_1971}, \citeyear{Rawls_2002}) desarrolla el concepto de la posición original. Bajo este concepto se busca construir una forma de deliberación para acordar los principios de justicia que van a regir a la estructura básica de la sociedad. Esta deliberación se realiza bajo condiciones que  \say{deben situar equitativamente a las personas libres e iguales y no deben permitir que algunos puedan negociar con los demás desde posiciones no equitativas de ventaja} \citep[p. 39]{Rawls_2002}. La posición original es una experimento mental, es decir, una situación imaginaria en donde todos los ciudadanos reales poseen un representante ideal y estos representantes llegan a un acuerdo sobre los principios de justicia que han de regir a los ciudadanos reales. Para mantener las condiciones equitativas, los representantes se encuentran bajo un velo de ignorancia que no les permite conocer características arbitrarias de las persona a las cuales se encuentran representando.


La teoría rawlsiana concibe a los ciudadanos como personas que poseen dos \say{facultades morales}:

\vspace{3mm}
\begin{quote}
    i) Una de esas facultades es la capacidad de poseer un sentido de la justicia: es la capacidad de entender, aplicar y obrar según (y no sólo de conformidad con) los principios de la justicia política que definen los términos equitativos de la cooperación social.
    
    ii) La otra facultad moral es la capacidad de poseer una concepción del bien: es la capacidad de poseer, revisar y perseguir racionalmente una concepción del bien \citep[p. 43]{Rawls_2002}.
\end{quote}
\vspace{3mm}

A través de estas facultades morales es posible establecer cómo es que los ciudadanos se conciben como personas libres e iguales. Por un lado, los ciudadanos se conciben como iguales debido a que todos se entienden poseedores de las facultades morales necesarias para participar en la cooperación social durante toda una vida. Por otro lado, los ciudadanos se conciben como libres por dos razones: 1) ya que reconocen que los demás tienen la facultad moral de poseer una concepción del bien y 2) se conciben como fuentes autoautentificatorias de exigencias válidas, es decir, \say{se ven a sí mismos con derecho a presentar exigencias a sus instituciones con ánimo de promover sus concepciones del bien} \citep[p. 48]{Rawls_2002}.

Mediante la deliberación en la posición original por representantes de ciudadanos libres e iguales, \citeauthor{Rawls_1971} (\citeyear{Rawls_1971}, \citeyear{Rawls_2002}) presenta dos principios de justicia que serían acordados como resultado del proceso. Se define a los dos principios de la siguiente manera:

\vspace{3mm}
\begin{quote}
   [...] a) cada persona tiene el mismo derecho irrevocable a un esquema plenamente adecuado de libertades básicas iguales que sea compatible con un esquema similar de libertades para todos; y
   
   b) las desigualdades sociales y económicas tienen que satisfacer dos condiciones: en primer lugar, tienen que estar vinculadas a cargos y posiciones abiertos a todos en condiciones de igualdad equitativa de oportunidades; y, en segundo lugar, las desigualdades deben redundar en un mayor beneficio de los miembros menos aventajados de la sociedad (el principio de diferencia) \citep[p. 73]{Rawls_2002}.
\end{quote}
\vspace{3mm}

Respecto de esta forma de concebir a los principios de justicia, en \textit{La justicia como equidad}, \citet{Rawls_2002} sostiene que se trata de una formulación revisada a la que se encuentra en \textit{Teoría de la justicia} §11. De todas formas, esta formulación revisada es similar a la que se encuentra en \textit{Teoría} §46. Asimismo, \citeauthor{Rawls_1971} (\citeyear{Rawls_1971}, \citeyear{Rawls_2002}) argumenta que el primer principio tiene una prioridad sobre el segundo principio. A su vez, dentro de las partes que componen al segundo principio de justicia, la igualdad equitativa de oportunidades (primera parte) tiene prioridad sobre el principio de la diferencia (segunda parte).

Estos principios tienen como objetivo distintas instancias de la estructura básica de la sociedad. Por un lado, el primer principio se enfoca en determinar la constitución política de las sociedades. Específicamente, este principio plantea que los ciudadanos tienen derecho a un conjunto de libertades básicas. En particular: 

\vspace{3mm}
\begin{quote}
[...] libertad de pensamiento y libertad de conciencia; libertades políticas (por ejemplo, el derecho de voto y el derecho a participar en política) y libertad de asociación, así como los derechos y libertades determinados por la libertad y la integridad (física y psicológica) de la persona; y finalmente, los derechos y libertades amparados por el imperio de la ley \citep[p. 75]{Rawls_2002}.
\end{quote}
\vspace{3mm}

Por otro lado, el ámbito del segundo principio es el diseño de las leyes y el diseño de las instituciones asociadas a la economía. Este principio de dos partes establece que, una vez que se haya cumplido con la igualdad equitativa de oportunidades para los ciudadanos, es posible establecer desigualdades económicas siempre y cuando las mismas favorezcan al grupo menos aventajado. A la hora de definir a este grupo, es necesario introducir el concepto de bienes primarios. Concretamente, 

\vspace{3mm}
\begin{quote}
    Estos bienes son las diversas condiciones sociales y los medios de uso universal que son por lo general necesarios para que los ciudadanos puedan desarrollarse adecuadamente y ejercer plenamente sus dos facultades morales, y para que puedan promover sus concepciones específicas del bien \citep[p. 90]{Rawls_2002}.
\end{quote}
\vspace{3mm}

Específicamente, el autor distingue cinco clases de bienes primarios: a) los derechos y libertades básicas, b) la libertad de movimiento y la libre elección de empleo, c) los poderes y prerrogativas que acompañan a cargos y posiciones, d) ingresos y riquezas y e) las bases sociales del autorrespeto. Bajo este último ítem, se enmarcan las instituciones básicas \say{esenciales si los ciudadanos han de tener clara conciencia de su valor como personas y han de ser capaces de promover sus fines con autoconfianza} \citep[p. 92]{Rawls_2002}. Según \citeauthor{Rawls_1971} (\citeyear{Rawls_1971}, \citeyear{Rawls_2002}), si nos situamos en una sociedad bien ordenada en donde se garantizan derechos y libertades a los ciudadanos, los menos aventajados son aquellos que integran el grupo de ingresos con las expectativas más bajas.

Volviendo sobre el principio de la diferencia, cabe recordar que el mismo debe ser entendido como un principio de justicia distributiva. En particular, este principio regula cómo han de diseñarse instituciones económicas que permiten la cooperación social para la producción de bienes materiales. La idea de permitir ciertas desigualdades es que las personas más productivas o talentosas consideran estas desigualdades como:

\vspace{3mm}
\begin{quote}
   [...] incentivos para que el proceso económico sea más eficiente, la innovación avance a un ritmo más rápido, etc. Eventualmente, los beneficios materiales resultantes se extienden por todo el sistema y hacia los menos aventajados. No consideraré hasta qué punto estas cosas son ciertas. El punto es que se debe argumentar algo de este tipo si estas desigualdades van a ser solo por el principio de diferencia\footnote{\say{[...] incentives so that the economic process is more efficient, innovation proceeds at a faster pace, and so on. Eventually the resulting material benefits spread throughout the system and to the least advantaged. I shall not consider how far these things are true. The point is that something of this kind must be argued if these inequalities are to be just by the difference principle} \citep[p. 71]{Rawls_1971}.} \citep[p. 71]{Rawls_1971}.
\end{quote}
\vspace{3mm}

En este sentido, \citet{Rawls_2002} argumenta que el principio de la diferencia invoca una idea de reciprocidad. Concretamente, a las personas más aventajadas se las incentiva a buscar beneficios mayores a través del cultivo de sus talentos y este desarrollo termina favoreciendo a los menos aventajados. La idea de reciprocidad que se encuentra en el principio de la diferencia brinda un argumento para alejarse de la distribución igualitaria estricta ya que \say{selecciona un punto focal natural entre las demandas de la eficiencia y la igualdad} \citep[p. 169]{Rawls_2002} En la misma línea, la noción de reciprocidad \say{está implícita en la idea de que ha de considerarse como un activo común la distribución de las dotaciones innatas} \citep[p. 170]{Rawls_2002}. 

A la hora de evidenciar arreglos institucionales que podrían cumplir con los dos principios de justicia, \citet{Rawls_2002} presenta la idea de la democracia de propietarios. Estos arreglos institucionales son contrastados con un capitalismo de Estado de bienestar, destacando que:

\vspace{3mm}
\begin{quote}
    [...] las instituciones de trasfondo de la democracia de propietarios contribuyen a dispersar la propiedad de la riqueza y el capital, con lo que impiden que una pequeña parte de la sociedad controle la economía y asimismo, indirectamente, la vida política. Por el contrario, el capitalismo del Estado de bienestar permite que una pequeña clase tenga un cuasimonopolio de los medios de producción \citep[p. 189]{Rawls_2002}.
\end{quote}
\vspace{3mm}

Por esta razón, un capitalismo de Estado de bienestar no logra satisfacer los principios de justicia. Específicamente, \citet{Rawls_2002} sostiene que en un capitalismo de Estado de bienestar se busca que ningún ciudadano caiga por debajo de un mínimo nivel de vida decente. Para lograr este objetivo, se establecen protecciones al empleo y a la salud. Adicionalmente, se emplea la redistribución de los ingresos a través de esquemas impositivos que pueden ser progresivos. En contraposición, en una democracia de propietarios el objetivo es que las instituciones logren operacionalizar la idea de la sociedad como un sistema equitativo de cooperación. Con este objetivo en mente, los ciudadanos deben ser provistos de los medios necesarios, tanto el capital humano y capital productivo, para ser miembros cooperantes de la sociedad durante toda una vida. En este contexto, los menos aventajados:

\vspace{3mm}
\begin{quote}
    [...] no son los desafortunados y desventurados —objetos de nuestra caridad y compasión, aunque mucho menos de nuestra piedad— sino aquellos a los que se debe reciprocidad en nombre de la justicia política, justicia política para una totalidad de ciudadanos libres e iguales. Aunque controlen menos recursos, cumplen su parte plenamente de un modo que es reconocido por todos como mutuamente ventajoso y consistente con el autorrespeto de cada cual \citep[p. 190]{Rawls_2002}.
\end{quote}
\vspace{3mm}

En síntesis, el principio de la diferencia brinda una respuesta al problema de los incentivos diferenciales: estos incentivos pueden ser compatibles con un esquema de justicia distributiva siempre y cuando se enmarquen en un contexto de igualdad de oportunidades y terminen siendo favorables para los menos aventajados. A su vez, la democracia de propietarios involucra unos arreglos institucionales diseñados para distribuir los bienes y mantener la igualdad entre los ciudadanos.

Ahora bien, aunque Rawls tiene presente la idea de cooperación tanto a la hora de conceptualizar a las sociedades como a la hora conceptualizar a los ciudadanos como racionales y razonables, parece persistir un cierto punto de la teoría neoclásica. En particular, si bien el principio de la diferencia se encuentra limitado por los demás principios de justicia, el mismo parece permitir ciertos aspectos de la competencia o características de las personas concebidas como egoístas racionales. Este punto será retomado \citet{Cohen_2001} ya que cuando se consideran las acciones de los ciudadanos, los principios de justicia no rigen en dicho ámbito generando una tensión en esta idea de cooperación.
