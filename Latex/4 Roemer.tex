Un autor que recoge el guante ante el desafío de Cohen es el economista John E. Roemer, especialmente si consideramos sus más recientes trabajos. Antes de entrar en el desarrollo de las propuestas más recientes, es necesario presentar algunos desarrollos teóricos previos para dar cuenta de la evolución en su pensamiento.

\subsection{Socialismo de mercado}

A lo largo de su trabajo, Roemer se ha presentado como un académico que logra vincular los aspectos normativos propios de la filosofía con aspectos operativos propios de la economía, es decir, el autor logra operacionalizar ciertas propuestas filosóficas a través de modelos económicos. Este autor comparte una trayectoria similar a la de Cohen, en el sentido de que ambos comenzaron analizando problemas desde una perspectiva marxista y luego, a través de varias influencias, se embarcan en cuestiones de justicia distributiva.

El trabajo de \citet{Roemer_1982} respecto de cuestiones normativas comienza con un análisis de la explotación en la tradición marxista aplicando los desarrollos de la teoría de juegos\footnote{En esta parte, para presentar un recorrido temporal en el trabajo del autor, sigo la introducción de \citet{Roemer_1996b}.}. Según el autor, la filosofía política o los conceptos normativos que estaban detrás del concepto marxista de explotación no eran claros. En este sentido, Roemer argumenta que lo que motivaba el diagnóstico de la presencia de explotación es una filosofía política igualitaria. En particular, se entendía que la justicia implica la distribución igualitaria de ciertos tipos de propiedad entre las personas. Habiendo llegado a esta noción es que el autor plantea que el enfoque de la explotación basado en la plusvalía estaba mal direccionado. Adicionalmente, Roemer sostiene que lo que Marx realmente quiso decir se captura mejor bajo un enfoque de la explotación basado en las relaciones de propiedad. 



Del análisis del concepto de explotación, Roemer concluye que la filosofía política que justificaba la crítica marxista del capitalismo es una especie de igualitarismo de recursos. Esto llevó al autor a indagar los trabajos de Dworkin (\citeyear{Dworkin_1981a}, \citeyear{Dworkin_1981b}) respecto de cuestiones igualitaristas. En la discusión del trabajo de Dworkin, Roemer comparte con \citet{Cohen_1989} que el \say{punto de corte} respecto de qué cosas son consideradas como responsabilidad del individuo no deberían ser sus preferencias. La propuesta tanto de Cohen como Roemer es considerar un conjunto de características llamadas \say{circunstancias}, las cuales están fuera del control del individuo y compensar a los individuos en función de cuan favorables o desfavorables resultan dichas circunstancias. Finalmente, según el autor, un igualitarista que defendiera estas ideas debería: \say{en el mundo actual, abogar por un mecanismo económico que puede describirse como socialismo de mercado}\footnote{\say{[...] in our present world, advocate an economic mechanism that can be described as market-socialism} \citep[p. 4]{Roemer_1996b}.} \citep[p. 4]{Roemer_1996b}. Por lo tanto, a continuación, se analiza una propuesta de Roemer: su modelo de socialismo de mercado.


Respecto del modelo de socialismo de mercado, es posible tomar como referencia el trabajo de \textit{A Future for Socialism} \citep{Roemer_1994}. En el contexto de la caída de la Unión Soviética, se había reforzado la creencia de que el socialismo no podría existir tanto en el presente como en una situación ideal. Ante esta situación, Roemer argumenta que es necesario revisar las experiencias del socialismo real y sugiere que se debe defender un nuevo modelo denominado como socialismo de mercado. En dicho modelo se busca combinar las fortalezas del sistema del mercado con las fortalezas del socialismo. A través de esta combinación, el modelo resultante estaría preocupado por lograr tanto la eficiencia productiva como también cumplir con ciertos ideales igualitarios. 

Según Roemer, esta conjunción se sustenta sobre la evidencia que muestra que los mercados no logran por si solos todos sus buenos resultados sino que necesitan ayuda. Los mercados se encuentran apoyados por muchas instituciones que han ido evolucionando a lo largo del tiempo para dar respuesta a diversos problemas. El autor plantea que: 

\vspace{3mm}
\begin{quote}
En contraste con la visión neoclásica \say{delgada}, que ve a los mercados como una estructura mínima que organiza la competencia entre individuos talentosos, la visión \say{gruesa} moderna ve a los mercados como una parte de una red compleja de instituciones creadas por el hombre, a través de las cuales todas las contribuciones de los individuos se pasteurizan y refinan\footnote{\say{In contrast to the «thin» neoclassical view, which sees markets as a minimal structure organizing competition among talented individuals, the modern \say{thick} view sees markets as part of a complex network of man-made institutions, through which all individuals contributions become pasteurized and refined} \citep[p. 5-6]{Roemer_1994}.} \citep[p. 5-6]{Roemer_1994}. 
\end{quote}
\vspace{3mm}

La visión \say{gruesa} de los mercados sería la apropiada y la más dócil para la coexistencia de los mercados y el socialismo. Ahora bien, ¿cuál sería la razón para proponer este tipo de modelo? Para dar cuenta de esto, \citet{Roemer_1994} presenta su visión de los objetivos que persiguen los socialistas. Dichos objetivos consisten en alcanzar una igualdad de oportunidades para: 1) la autorrealización y el bienestar, 2) la influencia política, y 3) el estatus social\footnote{En este punto se destaca una similitud con la propuesta rawlsiana. En particular, \citet[p. 177-178]{Rawls_2002} distingue tres razones para regular las desigualdades económicas: a) resulta erróneo que unos pocos tengan mucho y otros sufran privaciones, b) impedir que una parte de la sociedad domine al resto y c) impedir desigualdades de estatus social.}. Bajo la idea de autorrealización se entiende un proceso de desarrollo y aplicación de los talentos de un individuo de manera tal que le de sentido a su vida. Para Roemer, lo que distingue a las propuestas socialistas es que para ellos no basta con eliminar la discriminación en la contratación y garantizar el acceso a la educación. El ideal de igualdad de oportunidades requiere que las personas sean compensadas por desventajas que son inducidas por factores sobres los cuales no tienen ningún control\footnote{Esta visión se encuentra desarrollada extensamente en \citet{Roemer_1998}, en donde se presenta un algoritmo capaz de traducir cualquier visión, respecto de la proporción de las circunstancias y esfuerzo de las personas, en una política social que garantice el cumplimiento de un ideal de igualdad de oportunidades.}. La situación ideal sería encontrar unos arreglos institucionales o derechos de propiedad que logren obtener lo máximo posible de los tres objetivos. Sin embargo, el autor reconoce que no es posible maximizar los tres objetivos a la vez, lo que se debe hacer es buscar un \textit{trade-off} entre los mismos. 

Ahora bien, \citet{Roemer_1994} plantea que sería posible argumentar, por parte de algunos socialistas, que los objetivos defendidos desde su posición no son más que los postulados por el credo del liberalismo igualitario. En particular, debido a que los objetivos perseguidos por el socialismo es la eliminación del sistema capitalista en el cual una pequeña clase vive de la plusvalía que extraen de los trabajadores a través de la explotación. Ante esta crítica, el autor argumenta que es necesario desarrollar la crítica al capitalismo desde la distribución desigual de la propiedad de algunos bienes en contraposición al enfoque de la plusvalía. Al tomar esta nueva perspectiva, es necesario esclarecer cuestiones de filosofía política normativa, cuestiones que no eran tomadas en cuenta desde el marxismo clásico. En este sentido, pueden existir ciertas coincidencias con los postulados del liberalismo igualitario pero eso no implica que la corriente socialista colapse en un liberalismo igualitario. 

Partiendo desde los objetivos y el énfasis en los derechos de propiedad, es necesario mencionar a la noción de propiedad pública como principal alternativa a la propiedad privada. Según Roemer, como resultado de las experiencias del socialismo real se ha tendido a asociar a la propiedad pública con la propiedad estatal cuando, en rigor, estos conceptos no son lo mismo. En la misma linea, el autor sostiene que \say{los socialistas han convertido a la propiedad pública en un fetiche: la propiedad pública ha sido vista como la condición \textit{sine qua non} del socialismo, pero este juicio se basa en una inferencia falsa}\footnote{\say{[...] socialists have made a fetish of public ownership: public ownership has been viewed as the \textit{sine qua non} of socialism, but this judgement is based on a false inference} \citep[p. 20]{Roemer_1994}.} \citep[p. 20]{Roemer_1994}. El punto de Roemer es que la forma que tengan los derechos de propiedad va a depender exclusivamente de su idoneidad a la hora de cumplir con los objetivos establecidos anteriormente. Concretamente, el autor sostiene que: 

\vspace{3mm}
\begin{quote}
[...] considero que la elección de los derechos de propiedad sobre las empresas y otros recursos es un asunto enteramente instrumental; los socialistas deben evaluar las posibilidades de organizar tales derechos de acuerdo con la probabilidad de que induzcan las tres igualdades con las cuales los socialistas están preocupados\footnote{\say{[...] I view the choice of property rights over firms and other resources to be an entirely instrumental matter; possibilities for organizing such rights should be evaluated by socialists according to the likelihood that they will induce the three equalities with which socialists are concerned} \citep[p. 23]{Roemer_1994}.} \citep[p. 23]{Roemer_1994}.    
\end{quote}
\vspace{3mm}

Por otro lado, dentro de la discusión de la idea de socialismo de mercado, Roemer distingue cinco etapas. Dentro de este debate, es posible distinguir a \citet{Lange_1956} como uno de los principales defensores de la idea del socialismo de mercado y a \citeauthor{Hayek_1935} (\citeyear{Hayek_1935}, \citeyear{Hayek_1940}) como uno de los principales críticos. En primer lugar, el debate se centró en la necesidad de usar un sistemas de precios para realizar el cálculo económico dentro de las economías socialistas, ya que utilizar una unidad natural como la cantidad de energía o trabajo que se requiere para producir un bien tendría serios problemas. En segundo lugar, la discusión derivó en la visión de que dentro de una economía socialista sería posible calcular unos precios en donde se lograría alcanzar el equilibrio general resolviendo un sistema complejo de ecuaciones simultáneas. La tercera etapa se caracterizó por la idea de que resultarían necesarios los mercados en una economía socialista para poder hallar los equilibrios. En particular, debido a que los planificadores centrales nunca podrían obtener toda la inmensa información necesaria para realizar los cálculos sobre qué y cuánto producir.

En cuarto lugar, los aportes provenientes desde la teoría de la compatibilidad de incentivos, la cual no estaba interesada directamente en este debate, contribuyó de manera importante. Esto se asocia con el funcionamiento de las economías socialistas ya que en dicho contexto se genera un problema de principal-agente entre el planificador central y los administradores de las empresas. La teoría de la compatibilidad de incentivos se centra en el estudio de mecanismos que logren que los agentes (en este caso los administradores) encuentren que actuar de acuerdo a la regla propuesta por el principal (en este caso los planificadores) es su mejor curso de acción. Finalmente, la quinta etapa se caracterizó por reconocer que no era necesario insistir exclusivamente en la propiedad pública, entendida como propiedad estatal, de las empresas o, de manera más general, de los medios de producción.

Las fortalezas de los modelos resultantes de este debate residen en que incorporan las lecciones del socialismo de la Unión Soviética. Según Roemer, la caida del modelo soviético se debió a una conjunción de tres de sus características: 


\vspace{3mm}
\begin{quote}
(1) la asignación de la mayoría de los bienes por un aparato administrativo bajo el cual los productores no estaban obligados a competir entre sí, (2) el control directo de las empresas por parte de unidades políticas, y (3) la política no democrática y no competitiva\footnote{\say{(1) the allocation of most goods by an administrative apparatus under which producers were not forced to compete with each other, (2) direct control of firms by political units, and (3) noncompetitive, nondemocratic politics} \citep[p. 37]{Roemer_1994}.} \citep[p. 37]{Roemer_1994}.
    
\end{quote}
\vspace{3mm}


A la par de esto, el autor plantea que existieron tres instancias de problemas de principal-agente: a) entre los administradores de las empresas y los trabajadores, b) entre los planificadores del gobierno y los administradores de las empresas, y c) entre la población en general y los planificadores. En conjunto, estas características y problemas, ocasionaron que no existiera un proceso de innovación que impulsara el crecimiento económico luego de que la Unión Soviética alcanzara cierto nivel de desarrollo productivo.



A continuación se presenta un resumen de las ideas generales del modelo de socialismo de mercado propuesto por \citet[§8]{Roemer_1994}. En esta exposición, el autor propone comparar los posibles resultados, en términos de bienestar de la población, de su modelo con los resultados de una economía capitalista. Para dar cuenta de esto, Roemer propone una descripción general que es igual tanto para el caso capitalista como para el caso del socialismo de mercado. Lo que pretende hacer es analizar cómo impacta el cambio del mecanismo económico sobre los resultados.

Como es habitual en los modelos económicos, primero es necesario explicitar los supuestos simplificadores subyacentes. En el modelo se tiene una economía que produce un solo bien que todas las personas desean consumir. Existe un \say{mal público}\footnote{Este concepto se encuentra estrechamente asociado a la idea de bien público. Un bien público es aquel cuyo consumo es indivisible y no excluible a ningún miembro de la comunidad. El concepto de \say{mal público} habla de algún fenómeno que nadie desea experimentar o consumir pero resulta inevitable.}, que puede ser visto como la contaminación, que se produce conjuntamente con el bien. En esta economía, hay un número de ciudadanos de los cuales una pequeña fracción son inicialmente ricos mientras que una gran proporción son pobres. Todos los ciudadanos poseen las mismas preferencias a lo largo del tiempo respecto del bien de consumo y del mal público. Se dice que el bienestar de los ciudadanos es creciente con el nivel de consumo del bien y decreciente en el nivel de consumo del mal público. El bienestar de los ciudadanos considerados de manera individual, se captura a través de una función de utilidad $u(x_0,x_2,z)$, en donde $x_0$ es el consumo en $t=0$, $x_2$ es el consumo en $t=2$ y $z$ es el \say{consumo} del mal público en $t=2$.

De manera muy esquemática y resumida, dado que el objetivo es captar las intuiciones generales, la puesta en marcha del modelo funcionaría de la siguiente manera\footnote{En el siguiente párrafo sigo a \citet[p. 62]{Roemer_1994}.}. Existen tres momentos temporales, $t=\{0,1,2\}$. En el momento $t=0$, se supone que los ciudadanos conocen la probabilidad de los posibles estados de la naturaleza que pueden ocurrir en $t=2$. Esta incertidumbre afecta a la producción del bien por parte de las empresas. A su vez, en $t=0$, los ciudadanos poseen una cantidad del bien (razón por la cual algunos son ricos y otros pobres) y una parte o participación per cápita igual de cada empresa. Esto quiere decir que los ciudadanos son \say{dueños} de una parte, que es igual para todos, de todas las empresas de la economía. En el momento inicial, es decir, en $t=0$, las personas realizan sus decisiones de consumo e inversión. Luego, en $t=1$, los ciudadanos votan para determinar el nivel del mal público que será permitido (por ejemplo, el nivel de contaminación permitido para las empresas). Finalmente, en $t=2$, ocurre uno de los estados de la naturaleza, se realiza la producción y el producto de las empresas es consumido por los ciudadanos en función de las decisiones de inversión tomadas en $t=0$. En este modelo se supone que cada agente realiza sus planes con el objetivo de maximizar su utilidad esperada.

Como se menciona anteriormente, estos supuestos y la puesta en marcha descrita son iguales en el caso capitalista como en el socialismo de mercado. El siguiente paso es analizar lo que sucede en los distintos escenarios. En primer lugar, en el caso capitalista, se supone la existencia de un mercado de las acciones de las empresas que están en mano de los ciudadanos. Los precios de las acciones se expresan en términos del bien de consumo. En este contexto, una persona puede elaborar un portafolio de acciones utilizando las acciones que posee al comenzar como también los bienes que tiene al principio, es decir, uno puede comerciar acciones por otras acciones o bienes por acciones. Ante esta situación, lo que se espera es que: 

\vspace{3mm}
\begin{quote}
   [...] los pobres venderán buena parte de su dotación inicial de acciones de las empresas a los ricos, quienes las pagarán con el bien, que los pobres consumirán en el momento 0. Esto concentrará la propiedad de las acciones en manos de los ricos, con dos efectos: los ricos serán el grupo que tiene el control en la mayoría de las empresas y, por lo tanto, las elecciones de inversión de las empresas serán de su interés, y los ricos tendrán un mayor interés que los pobres en permitir un alto nivel del mal público\footnote{\say{[...] the poor will sell a good deal of their initial endowment of firm shares to the rich, who shall pay for them with the good, which the poor shall consume at date 0. This will concentrate the ownership of stock in the hands of the rich, with two effects: the rich will be the controlling group in most firms, and hence the firms' investment choices will be in their interest, and the rich will have a greater interest than the poor in permitting a high level of the public bad} \citep[p. 68]{Roemer_1994}.} \citep[p. 68]{Roemer_1994}.
\end{quote}
\vspace{3mm}

Por otro lado, en el caso del socialismo de mercado existe una diferencia en el mercado de las acciones: no es posible comprar acciones con el bien de consumo, solamente es posible comprar acciones utilizando cupones: \say{Cada ciudadano comienza con una dotación del bien, como antes, y, digamos, 1000 cupones, impresos por el gobierno}\footnote{\say{Each citizen begins with an endowment of the good, as before, and, say, 1000 coupons, printed by the goverment} \citep[p. 66]{Roemer_1994}.} \citep[p. 66]{Roemer_1994}. En este mercado de acciones modificado, los precios de las acciones se expresan exclusivamente en términos de los cupones y es ilegal comerciar cupones por bienes. En este escenario, se espera que los pobres controlen las empresas ya que poseen la mayoría de los cupones. Debido a esto, las decisiones de inversión van a estar en función de los intereses de los pobres y el nivel del mal público producido va a ser menor que en el caso anterior. Esto ocurre debido a que, como la propiedad de las empresas se encuentra distribuida entre muchas personas, no hay tantos incentivos a permitir altos niveles del mal público. En el caso de la economía capitalista, como una pequeña fracción de la población controlaba a las empresas, estos se veían beneficiados en mayor medida por aumentar la producción y permitir altos niveles de contaminación.

En conclusión, bajo el esquema del socialismo de mercado, se previene \say{el problema de \textit{free-rider} que afecta a los pobres bajo el capitalismo}\footnote{\say{[...] prevents the free rider problem that afflicts the poor under capitalism} \citep[p. 73]{Roemer_1994}.} \citep[p. 73]{Roemer_1994}. En la versión capitalista del modelo, los pobres encontraban que lo mejor para ellos, individualmente, consistía en vender sus acciones, lo cual concentraba el control de las empresas en los ciudadanos ricos. Por lo tanto, dado un conjunto de parametrizaciones del modelo, existen casos en donde el bienestar de los pobres es mayor y mantienen el control de la mayoría de las empresas. 

En este modelo, lo importante resulta la asignación de los derechos de propiedad sobre algunos recursos importantes para el desarrollo de las personas. Este tipo de modelo es lo que \citet{Cohen_2014c} tenía en mente a la hora de presentar su crítica. Se puede decir que son modelos \textit{segundos-mejores} ya que logran una mejora en comparación con los arreglos institucionales de las economías capitalistas. Sin embargo, nada se dice en los supuestos del modelo sobre el tipo de comportamiento que deban realizar las personas. En efecto, todo parece indicar que nos encontramos en una situación tradicional de los modelos económicos: personas egoístas racionales. Por esta razón, se podría argumentar que este modelo se queda corto a la hora de plasmar los ideales de comunidad e igualdad defendidos desde la visión de Cohen (\citeyear{Cohen_2014b}, \citeyear{Cohen_2014c}). Por lo tanto, a continuación se presenta un refinamiento a la hora de formular este tipo de modelos. En una palabra, se trata de operacionalizar el concepto de \textit{ethos} propuesto por Cohen.


\subsection{La noción de Optimización \textit{Kantiana}}

En sus trabajos más recientes, por ejemplo en \citet{Roemer_2017}, se sugiere que en sus desarrollos teóricos previos se carece de un enfoque basado en el \textit{ethos}. Tal es el caso del modelo de socialismo de mercado analizado anteriormente. En dicho modelo se hace especial énfasis en las cuestiones de los derechos de propiedad y se asumía que las personas se comportaban de manera egoísta. 

El autor reconoce el punto de \citet{Cohen_2014c} de que el problema principal de llevar adelante los ideales socialistas, es un problema de diseño. En este sentido, Roemer considera otros tipos de conductas que pueden tener los agentes como forma de idear nuevos modelos de socialismo de mercado. Con este objetivo en mente, es que se logra incoporar el concepto de \textit{ethos}. Los trabajos de Roemer más recientes giran en torno a la idea de modelizar la manera en que las personas cooperan en contextos económicos. La idea del \textit{ethos} es vinculada estrechamente con el concepto de cooperación. 

La forma de modelizar la cooperación que propone Roemer (\citeyear{Roemer_2019}, \citeyear{Roemer_2021a}), se presenta como una alternativa a la idea de equilibrio de Nash en la teoría de juegos. En particular, \citet{Roemer_2019} sostiene que la teoría económica se ha centrado en analizar cómo los agentes compiten entre si en los mercados y en los juegos. Sin embargo, la competencia no es el único comportamiento que las personas exhiben en estas situaciones. En diversos contextos económicos, las personas logran cooperar. 

Desde la teoría de juegos se ha intentado dar una explicación al comportamiento cooperativo a través de \textit{multistage games}\footnote{Estos juegos consisten en varias etapas. Supongamos que tenemos dos períodos de tiempo: $t \in \{1,2\}$, en $t=1$ se juega el juego $G_1$ mientras que en $t=2$ se juega $G_2$. En este contexto pueden surgir estrategias que sean cooperativas ya que los agentes pueden jugar de una manera que se puede resumir de la siguiente forma: \say{En el primer juego, juego la acción que implica cooperar. Si los demás jugadores cooperan, entonces coopero en el segundo juego, en otro caso juego la acción de no cooperar}. Este tipo de estrategias pueden llegar a ser equilibrios de Nash perfectos por subjuegos (Véase \citet[cap. 9]{Tadelis_2013}).} o juegos repetidos infinitas veces\footnote{Estos juegos consisten en jugar de manera infinita el mismo juego. En este caso, estrategias que impliquen cooperar siempre y cuando la historia del juego sea exclusivamente de cooperación, pueden llegar a ser equilibrios de Nash perfectos por subjuegos (Véase \citet[cap. 10]{Tadelis_2013}).}. Desde la economía comportamental se ha rechazado este tipo de explicación y se propone que la manera de explicar el comportamiento cooperativo es abandonando la idea de que las personas tienen preferencias exclusivamente auto-interesadas. Ahora bien, Roemer plantea que ambas explicaciones resultan insatisfactorias ya que las mismas comparten el uso de la noción de equilibrio de Nash. Para alcanzar un equilibrio de Nash, un jugador razona considerando a todas las acciones que pueden tomar los demás jugadores como parte de su entorno. Cuando un agente optimiza utilizando este protocolo se pregunta: si considero a las acciones de los demás como algo fijo, ¿cuál es la acción o estrategia que maximiza mi utilidad? 

El autor sostiene que un modelo de cooperación debería mostrar la forma en que cada individuo contempla cómo va a coordinar con los demás, es decir, los demás jugadores no deben ser vistos como parte de su entorno sino como \textit{parte de su acción}. En este sentido, Roemer plantea que se podría explicar el comportamiento cooperativo de una manera más intuitiva si se asume que las personas optimizan de otra forma en algunos contextos: \say{En juegos simples (simétricos), un jugador en una situación cooperativa se pregunta, «¿Cuál es la estrategia que me gustaría que todos juguemos?»}\footnote{\say{In simple games (symmetric ones), a player in a cooperative situation asks himself, «What is the strategy I would like all of us to play?»} \citep[p. viii]{Roemer_2019}} \citep[p. viii]{Roemer_2019}. Por este motivo es que el autor denomina a este proceso como \textit{Optimización Kantiana} ya que el tipo de razonamiento que se emplea puede considerarse como una interpretación del imperativo categórico o hipotético de Kant: \say{realiza aquellas acciones que querrías que fueran universalizables}\footnote{Esta es la interpretación que realiza Roemer de los imperativos kantianos. Sin embargo, en el caso de Kant, estos imperativos se encuentran dirigidos hacia máximas que puedan conformar leyes universalizables. En este sentido, es posible argumentar que Roemer se suma a la corriente de economistas que extraen las nociones kantianas hacia el campo de la economía de una manera no tan rigurosa.}. 

Con esto en mente, es posible elaborar una tipología de los modelos a tener en cuenta:

%\vspace{3mm}


\begin{table}[H]

\caption{Una tipología de los modelos}
%\resizebox{14cm}{!}{
\begin{tabular}{cccccc}
                      &                      & \multicolumn{4}{c}{\textbf{Preferencias}}                                                                                                                                                                                                                        \\ \cline{3-6} 
                      &                      &                      & \textbf{Auto-interesadas}                                                                                        &                      & \textbf{Altruistas - Complejas}                                                                 \\ \cline{3-6} 
                      &                      & \textbf{Nash}        & Modelo clásico                                                                                                   &                      & \begin{tabular}[c]{@{}c@{}}Economía \\ comportamental\end{tabular}                              \\
\textbf{Optimización} &                      &                      &                                                                                                                  &                      &                                                                                                 \\
                      &                      & \textbf{Kantiana}    & \begin{tabular}[c]{@{}c@{}}En la mayoría de\\ los capítulos de\\ How We Cooperate \\ (Roemer, 2019)\end{tabular} &                      & \begin{tabular}[c]{@{}c@{}}En el capítulo 5 de\\ How We Cooperate\\ (Roemer, 2019)\end{tabular} \\ \cline{3-6} 
\multicolumn{1}{l}{}  & \multicolumn{1}{l}{} & \multicolumn{1}{l}{} & \multicolumn{1}{l}{}                                                                                             & \multicolumn{1}{l}{} & \multicolumn{1}{l}{}                                                                            \\
\multicolumn{6}{l}{Fuente: adaptado de \citet[p.39]{Roemer_2019}}                                                                                                                                                                                                                       
\end{tabular}
\end{table}


Como punto de partida, \citet{Roemer_2019} alude a trabajos del campo de la psicología evolutiva, como los trabajos de \citeauthor{Tomasello_2014a} (\citeyear{Tomasello_2014a}, \citeyear{Tomasello_2014b}, \citeyear{Tomasello_2016}). Desde estos trabajos de la psicología evolutiva se argumenta que los seres humanos son una especie cooperativa ya que poseen tres características distintivas respecto de otros primates: a) los humanos son los únicos con esclera, b) los humanos pueden señalar e imitar y c) los humanos cuentan con el lenguaje.

Ahora bien, resulta necesario esclarecer el concepto de cooperación que se maneja en este contexto. \citet{Roemer_2019} entiende a la cooperación como el trabajo conjunto de personas para conseguir un fin en común. Bajo esta definición, las personas pueden no tener un interés por los demás, los une el fin en común. Estrechamente relacionado al concepto de cooperación, se encuentra el de solidaridad. La solidaridad puede ser entendida como la unión de propósitos, simpatías o intereses entre miembros de un grupo. Este concepto es una caracterización de la situación objetiva de las personas, todas se encuentran en una situación similar. El punto del autor es que \say{la capacidad de cooperar por motivos de interés personal es menos exigente que la prescripción de preocuparse por los demás}\footnote{\say{[...] is that the ability to cooperate for reason of self-interest is less demanding than the prescription to care about others} \citep[p. 5]{Roemer_2019}.} \citep[p. 5]{Roemer_2019}. De todas formas, las condiciones necesarias para la cooperación son la solidaridad y la confianza entre las personas involucradas.

Dicho esto, para dar cuenta del concepto de Optimización Kantiana partimos del análisis de un ejemplo sencillo, un juego simétrico, un ejemplo del dilema del prisionero:


\vspace{3mm}
\begin{table}[H]
\caption{Un ejemplo del dilema del prisionero (extraído de Roemer, \citeyear{Roemer_2019}).}
\centering
\begin{tabular}{cccc}
\hline
           & \textbf{A} & \textbf{} & \textbf{B} \\ \hline
\textbf{A} & (1,1)      &           & (-1, 2)    \\
\textbf{}  &            &           &            \\
\textbf{B} & (2, -1)    &           & (0,0)      \\ \hline
\end{tabular}
\end{table}
\vspace{3mm}


En este caso contamos con dos jugadores: un jugador fila y un jugador columna. Ambos jugadores tienen el mismo espacio de estrategias, es decir, pueden jugar A o B. Los pagos para el jugador fila corresponden al primer número dentro de las intersecciones de las estrategias. Por ejemplo, si el jugador fila juega A y el jugador columna juega B, entonces los pagos son de 2 para el jugador fila y -1 para el jugador columna. Un juego juego simétrico es aquel en donde los jugadores están situados de manera idéntica: poseen el mismo espacio de estrategias y los pagos son los mismos. En este caso las estrategias posibles son: jugar $A$, jugar $B$ o jugar una estrategia mixta que le asigna una probabilidad positiva $\alpha$ a jugar A y una probabilidad $1-\alpha$ a jugar B. Los pagos de la diagonal principal de la matriz son los mismos para ambos $(1,1)$ y $(0,0)$ y en los restantes casos los pagos se invierten.

Desde la óptica de un agente que emplea el protocolo de Nash, la estrategia $B$ es una estrategia estrictamente dominante ya que los pagos que recibe al jugar dicha estrategia siempre son mayores a los podría recibir si jugara $A$. Como cada jugador hace el mismo razonamiento, el equilibrio de Nash de este juego es el perfil de estrategias $(B,B)$. Por otro lado, un agente que optimizara de manera kantiana, se preguntaría: ¿Cuál es la estrategia que quisiera que ambos juguemos? La respuesta es la estrategia $A$ ya que ambos jugadores van a estar mejor que si los dos jugaran $B$. Llegamos así, a una definición: \say{En un juego simétrico, la estrategia que cada uno preferiría que todos jugaran es un equilibrio kantiano simple}\footnote{\say{In a symmetric game, the strategy that each would prefer all to play is a simple Kantian equilibrium} \citep[p. 13]{Roemer_2019}.} \citep[p. 13]{Roemer_2019}. Denominado SKE (\textit{Simple Kantian Equilibrium)} por sus siglas en inglés. Es posible brindarle más contenido al razonamiento que emplean los jugadores para arribar a los equilibrios kantianos:

\vspace{3mm}
\begin{quote}
   [...] debido a la simetría del juego, asumo que cualquier estrategia que yo decida también la decidirá mi oponente. De ello se deduce que solo debo considerar los perfiles de estrategia $(x,x)$ como aquellos que pueden ocurrir, donde $x \in \{A,B\}$. Por lo tanto, debo elegir la estrategia $x$ que maximice mi pago, si mi oponente y yo jugamos $(x,x)$. Esa es la estrategia $A$. Mi oponente elegirá la misma acción, porque también razonará de esta manera, y de esto estoy seguro, debido al supuesto de conocimiento común y nuestros poderes de razonamiento iguales\footnote{\say{[...] due to symmetry of the game, I assume that whatever strategy I decide upon will also be decided upon by my opponent. It follows that I must only consider strategy profiles $(x,x)$ as ones that migth occur, where $x \in \{A,B\}$. I therefore should choose the strategy $x$ that maximizes my payoff, if $(x,x)$ is played by my opponent and me. That is strategy $A$. My opponent will choose the same action, because he will reason this way as well, and of this I am confident, because of the common-knowledge assumption and out equal reasoning powers} \citep[p. 19]{Roemer_2019}.} \citep[p. 19]{Roemer_2019}.
\end{quote}
\vspace{3mm}

En este punto, cabe destacar la importancia del concepto de simetría entre los jugadores con los ideales de comunidad e igualdad planteados por \citet{Cohen_2014c}. En particular, la idea de que, para que dichos ideales socialistas se mantengan, tiene que fomentarse una preocupación de los individuos por los demás. Resulta poco plausible que surja este tipo de preocupación en un contexto de amplias desigualdades. Concretamente, ya que a los individuos les resultaría más difícil comprender las privaciones que experimentan los demás. Una situación de simetría permite superar dicha dificultad.

Ahora bien, un defensor de la teoría de juegos y de la noción del equilibrio de Nash, argumentaría que el razonamiento presentado anteriormente es irracional. \citet{Roemer_2019} alega que no es el caso: desde la teoría de juegos se ha tendido a desarrollar el concepto de racionalidad de una manera muy estrecha como la maximización individual de utilidad. Emplear un razonamiento que busque lograr la coordinación entre las personas no sería irracional e incluso podría llevar a resultados que son mejoras de Pareto en comparación a los resultados que se llegan empleando la noción de equilibrio de Nash\footnote{En este punto existe cierta similaridad con los conceptos desarrollados por \citeauthor{Rawls_1971} (\citeyear{Rawls_1971}, \citeyear{Rawls_2002}) respecto de la razonabilidad y la racionalidad (Véase Sección \ref{sec2.3})}.




Es posible evidenciar algunos aspectos deseables de emplear la optimización kantiana si nos centramos en dos casos concretos: \textit{la tragedia de los comunes} y la contribución a los bienes públicos o problema del \textit{free-rider}. El caso de \textit{la tragedia de los comunes}, presentado por \citet{Hardin_1968}, puede ser visto como un juego en el cual la acción de los demás jugadores genera una externalidad negativa que afecta a todos. En este contexto, si pensamos desde una optimización según el protocolo de Nash, se llega a una situación en donde se sobreexplota un recurso al cual una comunidad tiene acceso. En cambio, desde la optimización kantiana, al incorporar al resto de los individuos a la hora de tomar una decisión, se logra llegar a una situación se explota en menor medida el recurso y se alcanza la eficiencia de Pareto.

Por otro lado, los problemas de \textit{free-rider} se pueden originar a la hora de querer financiar bienes públicos. En este caso, el problema puede presentarse de la siguiente manera: dado que nadie puede excluirme del uso del bien público una vez que se produce, entonces no tengo ninguna razón para aportar a la producción de dicho bien. Si todos piensan de esa manera o un número importante de personas, los bienes públicos pueden no producirse o producirse en una magnitud menor. Esto es lo que predice la teoría de juegos estándar como equilibrio de Nash en esta situación. Ahora bien, desde la optimización kantiana, se tiene en cuenta la externalidad positiva que generaría la contribución al bien público. Lo que se espera es que cada persona contribuya hasta llegar a un nivel del bien público que sea Pareto eficiente.

Habiendo presentando este esquema de análisis, cabe preguntarse cómo podría influir en los resultados concebir a las personas siendo altruistas. \citet{Roemer_2019} implementa una preocupación altruista en los individuos a partir de incorporar un argumento adicional en las preferencias. Dicho argumento es una ponderación positiva de una función de bienestar social, es decir, los individuos se preocupan tanto por su utilidad personal como la utilidad del conjunto de la sociedad. La idea de incorporar una preocupación altruista en los individuos podría ser una forma adecuada de mostrar un interés por parte de Roemer por las cuestiones esbozadas por \citet{Cohen_2014c}. En particular, el altruismo podría ser una forma en la cual se fortalezcan los ideales de comunidad e igualdad. 

Volviendo a Roemer, el autor enuncia: \say{los equilibrios kantianos para una economía con un grado positivo de altruismo, con respecto a una regla de asignación, son idénticos a los equilibrios kantianos para la economía asociada con preferencias puramente egoístas}\footnote{\say{[...] the kantian equilibria for an economy with a positive degree of altruism, with respect to an allocation rule, are identical to the kantian equilibria for the associated economy with purely self-regarding preferences} \citep[p. 85]{Roemer_2019}.} \citep[p. 85]{Roemer_2019}. Lo que quiere decir el autor con esto es que no podemos distinguir observacionalmente entre dos situaciones en donde en un caso las personas posean un grado de altruismo y en otro se encuentren optimizando de manera kantiana auto-interesada: \say{Si una comunidad ha aprendido a cooperar en el sentido de emplear la optimización kantiana, no podemos decir observando el equilibrio si sus miembros tienen preferencias altruistas o no -al menos, con el altruismo modelado de esta manera}\footnote{\say{If a community has learned to cooperate in the sense of employing kantian optimizacion, we cannot tell by observing the equilibrium whether its members hold altruistic preferences of not -at least, with altruism modeled in this way} \citep[p. 86]{Roemer_2019}.} \citep[p. 86]{Roemer_2019}.

Otro punto de controversia dentro de esta teoría es la posibilidad de concebir a la optimización kantiana como meros equilibrios de Nash en donde las personas toman en cuenta la asignación total de los pagos y no exclusivamente sus pagos individuales. Básicamente, la pregunta es si a partir de un equilibrio kantiano con unas preferencias auto-interesadas se puede llegar al mismo resultado a traves de un equilibrio de Nash con otro tipo de preferencias. La respuesta puede parece un insatisfactoria ya que es afirmativa. El punto de Roemer es que resulta poco creíble que los agentes realicen dicha transformación ya que la misma puede llegar a ser muy compleja. En este caso, Roemer parece estar apelando a un principio de parsimonia a la hora de adoptar el protocolo kantiano como explicación del comportamiento cooperativo.

Habiendo presentado el esquema conceptual detrás de la optimización kantiana y su aplicación a ciertos contextos, en gran parte juegos, a continuación se desarrolla un modelo de socialismo de mercado que incorpora este protocolo a la hora de resolver el problema de la asignación de recursos. En particular, se toma como referencia el trabajo de \citet{Roemer_2021a}. 

\subsection{Socialismo de mercado como una economía cooperativa}

Según \citet[p. 572]{Roemer_2021a}, cada sistema socio económico tiene tres pilares: \say{un ethos de comportamiento económico, una ética de justicia distributiva y un conjunto de relaciones de propiedad que (se espera) implementarán la ética si se sigue el ethos conductual}\footnote{\say{[...] an ethos of economic behavior, an ethic of distributive justice, and a set of property relation that will (it is hoped) implement the ethic if the behavioral ethos is followed} \citep[p. 572]{Roemer_2021a}.}. A la hora de describir la forma que tienen estos pilares dentro de un sistema socialista, el autor plantea que el \textit{ethos} correspondiente es el de la cooperación. Respecto de la manera de concebir la ética distributiva, Roemer invoca la noción de \textit{igualdad de oportunidades socialista} desarrollada por \citet{Cohen_2014c}. En la misma linea argumental que en trabajos previos, como \citet{Roemer_1994}, las relaciones de propiedad o derechos de propiedad se diseñan en función de los objetivos perseguidos. Dichos objetivos son: \say{implementar la igualdad de oportunidades socialista, en la medida en que esto sea posible en una economía de mercado, y a reflejar el espíritu cooperativo del comportamiento económico}\footnote{\say{[...] implement socialist equality of opportunity, so far as this is possible in a market economy, and to reflect the cooperative ethos of economic behavior} \citep[p. 572-573]{Roemer_2021a}.} \citep[p. 572-573]{Roemer_2021a}.

En contraposición al socialismo, se caracterizan los tres pilares del capitalismo. En primer lugar, el \textit{ethos} comportamental es individualista. Desde la noción de equilibrio de Nash, las interacciones dentro de los juegos o mercados conciben al resto de las personas como meros componentes fijos dentro de su entorno. En segundo lugar, la ética distributiva es la del \textit{laissez-faire}. Cada uno por su cuenta y según su capacidad debe conseguir los bienes que desea. No se cuestionan los resultados desiguales que se pueden seguir de una lotería natural arbitraria de los talentos. Finalmente, los derechos de propiedad, en gran medida, son asignados de manera privada.

El objetivo de \citet{Roemer_2021a} es presentar dos modelos de socialismo que incorporan la idea del \textit{ethos} cooperativo bajo la forma del protocolo de optimización kantiana. El primer modelo, denominado \textit{Socialismo 1}, consiste en una socialdemocracia; mientras que el segundo modelo, denominado \textit{Socialismo 2} es una \textit{sharing economy}. La principal diferencia entre estos modelos es que en el segundo caso los beneficios producidos por las empresas se distribuyen entre los agentes que contribuyen en la producción, es decir, aquellos agentes que suministran trabajo o capital a la empresa. En el primer modelo, los beneficios pertenecen a los dueños de las firmas.


Para dar cuenta de algunas características de los modelos de socialismo, se comienza a partir un modelo de equilibrio general clásico. Nos encontramos en una economía en donde se produce un único bien. Existe una firma con una función de producción \say{bien comportada}\footnote{Bajo esta expresión se denota que la función de producción es creciente, diferenciable y cóncava.} que emplea trabajo y capital. La sociedad está conformada por $n$ individuos. Las personas poseen unas preferencias que se pueden representar por una función de utilidad $u$ que tiene como argumentos: el consumo del bien $x_j$ que afecta positivamente a la utilidad y la cantidad de trabajo que ofrece $L_j$ que afecta negativamente a la utilidad. Algunos individuos poseen dotaciones de capital positivas y son dueños de las acciones de las empresas. Los precios relevantes para esta economía son: el precio del bien de consumo $(p)$, el salario $(w)$ y la remuneración al capital $(r)$. 

Dicho todo esto, es posible establecer que un equilibrio en esta economía consiste en 1) un vector de los precios $(p,w,r)$, 2) unos valores de la demanda de trabajo y capital, y 3) unos valores de la oferta de bienes de consumo, trabajo y capital; tales que: la empresa existente se encuentra maximizando sus beneficios, los individuos ofertan la cantidad de trabajo que maximiza su utilidad y los mercados se vacían, es decir, la oferta es igual a la demanda en todos los mercados. Las condiciones de primer orden de este problema de maximización implican que los factores productivos son remunerados exactamente en la misma proporción que su contribución marginal al producto.

De este equilibrio, \citet{Roemer_2021a} destaca dos características positivas: a) por el primer teorema de la economía del bienestar sabemos que este equilibrio es Pareto eficiente; b) el sistema de precios descentraliza la asignación competitiva, es decir, las empresas solamente necesitan conocer los precios y su función de producción mientras que los consumidores solo necesitan conocer los precios y sus preferencias. Como se menciona anteriormente (Véase Sección \ref{sec2.2}) esta distribución es cuestionable desde una perpesctiva igualitaria. En particular, debido a  que los trabajadores e inversionistas reciben una remuneracion acorde a su contribución marginal y los beneficios son apropiados enteramente por los dueños de la empresa.

Supongamos ahora que se quiere implementar un impuesto que busque redistribuir el excedente de manera más equitativa. El resultado que predice la teoría es que el nuevo equilibrio no va a resultar Pareto eficiente, es decir, se va a llegar a una situación en la cual, entre otras cosas, el nivel de bienes que dispone la economía va a ser menor. El punto que Roemer quiere destacar es que esta pérdida de eficiencia se debe a que los individuos maximizan empleando el protocolo de Nash. La contribución del autor es la idea de que la optimización kantiana permite superar el \textit{trade-off} entre eficiencia y equidad.

Para evidenciar esto, es necesario introducir dos conceptos nuevos: a) equilibrio kantiano aditivo y b) equilibrio kantiano multiplicativo (\citeauthor{Roemer_2019}, \citeyear{Roemer_2019}, \citeyear{Roemer_2021a}). Nos situamos en el contexto de los juegos en donde las estrategias de los jugadores consisten en su nivel de \say{contribución} o \say{esfuerzo}: $E^i$. A su vez, el autor se enfoca en aquellos juegos en donde la función de pagos $V$, es estrictamente creciente o decreciente en función de $E^i$.  La noción de equilibrio kantiano aditivo da cuenta de una situación en donde ningún agente tiene incentivos a modificar su contribución mediante la suma de un escalar positivo. Específicamente, si su contribución es $E^i$, entonces no tiene incentivos a jugar $E^i + j$ con $j \neq 0$\footnote{De manera formal: un perfil de estrategias $(E^1,..., E^n)$, con cada $E^i \in I$, siendo $I$ el espacio de estrategias, es un equilibrio kantiano aditivo si $\forall_i (0 = \arg_{\{p|(E^i + p) \in I\}} \max V^i (E^1 + p, E^2 + p, ... E^n + p))$. Véase \citet[p. 578]{Roemer_2021a}). Básicamente, cero es el valor del argumento que maximiza la función de pagos, es decir, nadie tiene incentivos a cambiar su contribución.}. Por otro lado, la noción de equilibrio kantiano multiplicativo es similar a la aditiva solo que en este caso nadie tiene incentivos a modificar su contribución mediante el factor de un escalar, es decir, si su contribución es $E^i$, entonces no tiene incentivos a jugar $(E^i \times j)$ con $j \neq 1$\footnote{De manera formal: un perfil de estrategias $(E^1,..., E^n)$, con cada $E^i \in I$, siendo $I$ el espacio de estrategias, es un equilibrio kantiano multiplicativo si $\forall_i (1 = \arg_{\{p|(p E^i) \in I\}} \max V^i (p E^1, p E^2, ..., p E^n))$. Véase \citet[p. 578]{Roemer_2021a}. Básicamente, 1 es el valor del argumento que maximiza la función de pagos, es decir, nadie tiene incentivos a cambiar su contribución.}.

Con esto en mente podemos comenzar a analizar el modelo de \textit{Socialismo 1} o socialdemócrata. La descripción de la economía es igual al modelo del equilibrio general tradicional. Una diferencia radica en que en este caso existe un impuesto sobre los ingresos de las personas que persigue un fin redistributivo. A su vez, los trabajadores toman la decisión de cuánto trabajo ofertar, es decir, su contribución o esfuerzo, a través de un proceso de optimización kantiana. La idea es que buscan maximizar su utilidad que depende de los ingresos provenientes de todas las fuentes: salario, transferencias a través de lo que se recauda con el impuesto, ganancias por el capital y excedente en caso de ser propietarios de acciones de la firma. Al tomar la decisión sobre la cantidad de trabajo que ofertan, mediante el protocolo kantiano, logran tener en cuenta cómo cada una de sus decisiones individuales afecta el monto que reciben por las transferencias. 

A partir de lo anterior, \citet{Roemer_2021a} plantea que un equilibrio socialdemócrata consiste en: 1) un vector de precios $(p,w,r)$, 2) unos valores de la demanda de trabajo y capital, y 3) unos valores de la oferta de bienes de consumo, trabajo y capital; tales que: a) la empresa existente se encuentra maximizando sus beneficios; b) la decisión por parte de los trabajadores, sobre cuánto trabajar, conforma un \textit{equilibrio kantiano aditivo}; y c) los mercados se vacían. 

Lo más relevante de este tipo de equilibrio es el resultado que \citet[p. 579]{Roemer_2021a} enuncia y demuestra: si tenemos una asignación que es un equilibrio socialdemócrata, resultante de cualquier tasa impositiva $t \in [0,1]$, entonces es Pareto eficiente. El autor denomina a este resultado como \say{primer teorema de la economía del bienestar en la socialdemocracia}. Este teorema es el que le permite afirmar que se logra superar el \textit{trade-off} entre eficiencia y equidad. Concretamente, debido a que cualquier asignación que sea un equilibrio con un nivel arbitrario de la tasa impositiva va a resultar un equilibrio Pareto eficiente. Esto sucede debido a que, como las personas emplean la optimización kantiana, todas las personas ajustan su oferta laboral de acuerdo al impuesto teniendo en cuenta cómo su decisión va a afectar al resto de las personas a través de las transferencias.

A continuación, se analiza el modelo de \textit{Socialismo 2} o \textit{sharing economy}. Nos situamos en una economía que funciona de manera similar a los dos modelos anteriores. En este caso, no hay un impuesto al ingreso. La principal diferencia de este modelo es que existe un parámetro exógeno $\lambda \in [0,1]$ que representa cómo se divide el excedente entre trabajadores y los dueños de capital. Cuando $\lambda = 1$, los trabajadores se apropian de todo excedente; mientras que si $\lambda = 0$, los inversionistas, es decir, aquellos que aportar el capital para producir, se apropian de todo el excedente. En este modelo, a los dueños de la empresa que produce el bien de consumo no les corresponde ninguna parte del excedente ya que no aportan ningun insumo productivo.

Bajo este esquema, un equilibrio de la economía $\lambda$-\textit{sharing} consiste en: 1) un vector de precios $(p,w,r)$, 2) unos valores de la demanda de trabajo y capital, y 3) unos valores de la oferta de bienes de consumo, trabajo y capital; tales que: a) la empresa existente se encuentra maximizando sus beneficios; b) la decisión por parte de los trabajadores, sobre cuánto trabajar, conforma un \textit{equilibrio kantiano multiplicativo}; y c) los mercados se vacían. Respecto de la eficiencia de este equilibrio, \citet[p. 584]{Roemer_2021a} enuncia y demuestra que cualquier equilibrio $\lambda$-\textit{sharing}, en el cual las personas ofrecen una cantidad positiva de trabajo, es Pareto eficiente.

A través de estos ejemplos, se puede evidenciar cómo a través de la optimización kantiana se logran resultados deseables en términos de eficiencia y equidad. En el modelo de \textit{Socialismo 1}, se logra redistribuir a través de los impuestos; mientras que en el modelo de \textit{Socialismo 2} se redistribuye el excedente entre las personas que aportan a la producción. Ahora bien, el autor discute respecto de si este tipo de comportamiento puede llegar a ser creible o es meramente una curiosidad matemática:

\vspace{3mm}
\begin{quote}
    Los tres requisitos previos necesarios para que un grupo de individuos optimice de manera Kantiana son la comprensión, el deseo y la confianza. La gente debe \textit{entender} que la optimización Kantiana puede conducir a buenas soluciones (eficientes) al problema económico. Deben \textit{desear} cooperar, porque ven su situación como solidaria, es decir, enfrentan un problema económico común (la lucha contra la Naturaleza) cuya solución requerirá de la cooperación. Tercero, cada uno debe \textit{confiar} en que los demás optimizarán de la manera Kantiana si él/ella lo hace, para que los optimizadores a la Nash no se aprovechen de los Kantianos, quienes casi siempre pueden beneficiarse como individuos, al menos a corto plazo, jugando Nash contra la multitud Kantiana\footnote{\say{The three prerequisites necessary for a group of individuals to optimize in the kantian manner are understanding, desire, and trust. People must understand that kantian optimization can lead to good (efficiente) solutions to the economic problem. They must desire to cooperate, because they see their situation as one of solidarity, meaning that they face a common economic problem (the struggle against Nature) whose solution will requiere cooperation. Third, each must trust that others will optimize in the kantian manner if he/she does, so that the Kantians will not be taken advantage of by Nash optimizers, who can almost always benefit as individuals, at least in the short run, by playing Nash against the Kantian crowd} \citep[p. 591]{Roemer_2021a}.} \citep[p. 591]{Roemer_2021a}.
\end{quote}
\vspace{3mm}

En la misma línea, Roemer comparte la opinión de \citet{Cohen_2014c} de que las grandes desigualdades pueden dificultar que surja el deseo de cooperar\footnote{Aunque el punto de \citet{Cohen_2014c} se encuentra más vinculado al surgimiento y mantenimiento del ideal de comunidad, también se podría plantear que el ideal comunitario involucra la cooperación.}. En efecto, \citet[p. 594]{Roemer_2021a} plantea que quizás el modelo de \textit{Socialismo 2} posee \say{la ventaja de promover una estabilidad del ethos en comparación con la socialdemocracia}\footnote{\say{[...] the advantage of promoting ethos stability compared to social democracy} \citet[p. 594]{Roemer_2021a}.}. Esta ventaja surge por el hecho de que en dicho modelo los trabajadores e inversores comparten el excedente productivo. No existe un agente que se apropie del excedente \say{sin aportar nada a la producción}. 

La ventaja de los modelos analizados en este apartado radica en que se vuelve explícita la idea de que para lograr un ideal igualitario hay que tomar en cuenta el comportamiento individual. Se parten de modelos de socialismo de mercado en donde el énfasis estaba centrado exclusivamente en los derechos de propiedad hacia modelos en donde, adicionalmente, se incorpora un \textit{ethos} igualitarista. De todas formas, cabe preguntarse si el modelo esbozado por \citet{Roemer_2021a} logra superar el desafío puesto por Cohen.



