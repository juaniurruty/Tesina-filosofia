
La idea de que los incentivos diferenciales son justos cuando terminan beneficiando a los peores situados, no está exenta de polémica. Un autor que discrepa con el argumento de Rawls analizado anteriormente es el filósofo Gerald Cohen. 

\subsection{¿Por qué la igualdad?}

Para enmarcar el pensamiento de Cohen, se distinguen dos etapas. En primer lugar, una etapa en donde su interés radica en analizar la teoría marxista desde una perspectiva diferente. Al emplear su formación en la tradición analítica, Cohen perteneció a una corriente denominada como \say{marxismo analítico}; la obra fundamental de esta etapa es \textit{Karl Marx's Theory of History: A Defence} (\citeyear{Cohen_1978}). En un segundo momento, el autor se embarca en el estudio de la filosofía política normativa haciendo énfasis en cuestiones de justicia distributiva. Este movimiento en el campo de interés podría ser considerado poco fructícero desde la corriente marxista tradicional.

Ahora bien, \citet{Cohen_2001} argumenta que una corriente socialista debería tener un interés sobre la filosofía normativa ya que resulta difícil sostener dos hipótesis defendidas desde el marxismo tradicional respecto del desarrollo histórico, a saber: 1) el ascenso de la clase trabajadora organizada, que garantizaría la igualdad y 2) el desarrollo de las fuerzas productivas que generaría una abundancia de bienes materiales. Estos postulados eran presentados como tendencias irreversibles del devenir histórico, por lo cual carecía de sentido un argumento desde el punto de vista de la igualdad en la distribución. 

Para el autor, la primera hipótesis resulta falsa ya que el proletariado se encuentra en un proceso de desintegración. Con el desarrollo del capitalismo, no se generó una situación en la que todos los trabajadores se encasillaron en la pobreza sino que la fuerza de trabajo se ha ido complejizando volviendo poco realista la idea de que todos los trabajadores se enfrentan a situaciones de escasez similares. Por otro lado, respecto de la segunda hipótesis, existe un límite impuesto por los recursos naturales existentes en el mundo. Este límite no permitiría un nivel de consumo de bienes para todas las personas similar al consumo de los países desarrollados.

En la misma línea, \citet{Cohen_2001} sostiene que para los marxistas ahondar en los fundamentos normativos detrás de su propuesta política no era relevante. En particular, ya que las personas que conformaban el proletariado contaban con las siguientes características: 1) constituían la mayoría de  la sociedad, 2) producían la riqueza de la sociedad, 3) eran los explotados de la sociedad y 4) eran los necesitados de la sociedad; dadas estas características, es posible establecer que 5) no tendrían nada que perder con la revolución y 6) podrían transformar y transformarían la sociedad. Teniendo en cuenta esta descripción, muchos principios normativos podrían alinearse con la lucha por mejores condiciones de vida para el proletariado, por lo cual no sería relevante indagar la temática.

Partiendo de la base de las dos hipótesis sobre las tendencias del devenir histórico y de la caracterización del proletariado, no se exigía la igualdad sino que se pensaba que era algo inevitable. Según Cohen, al abandonar este terreno del marxismo tradicional, surge un nuevo fundamento para exigir la igualdad vinculado con la crisis ecológica:

\vspace{3mm}
\begin{quote}
    [...] nuestro medio ambiente ya está severamente degradado y que, si hay alguna forma de salir de la crisis, esa forma ha de pasar por un menor consumo material del que ahora existe y, como resultado de ello, ha de pasar por cambios no deseados en el estilo de vida de cientos de millones de personas. [...] Es indudable que el consumo que realiza Occidente, \textit{medido en términos de uso de la energía combustible fósil y de recursos naturales}, en porcentaje debe reducirse drásticamente y que el consumo que realizan los países no occidentales, considerado en conjunto, nunca alcanzará los niveles actuales de Occidente, \textit{medidos de esta forma} \citep[p. 152-153]{Cohen_2001}.
\end{quote}
\vspace{3mm}

Al verse frustrada la posibilidad de una mejora ilimitada en la calidad de vida de las personas debido a las restricciones que imponen los recursos naturales, Cohen afirma que los niveles de desigualdad en términos de consumo y/o ingreso se vuelven mucho más intolerables desde un punto de vista moral. Ahora bien, cabe destacar la importancia del postulado de la abundancia que surgiría con el desarrollo de las fuerzas productivas. Según \citet{Cohen_2001}, respecto de esta proposición, existía un importante optimismo desde el marxismo tradicional. En particular, ya que al mismo tiempo se tenía una postura pesimista respecto del ordenamiento social en condiciones que no fueran de abundancia material. Concretamente, bajo las condiciones de escasez, se creía que la sociedad de clases era inevitable.

Para el autor, es necesario abandonar tanto el optimismo sobre la posibilidad de la abundancia material como también el pesimismo sobre el cambio social en condiciones de escasez. Esta doble renuncia implica que al aceptar la premisa de la escasez material hay pensar claramente respecto de \say{aquello que estamos buscando, qué razones tenemos para buscarlo y por qué medios institucionales puede realizarse} \citep[p. 156]{Cohen_2001}.

Pese al aparente desinterés del marxismo tradicional respecto de las cuestiones normativas, \citet{Cohen_2014a} cree que la justicia ocupa un lugar central en la creencia marxista revolucionaria. Para dar cuenta de esto, el autor establece una comparación entre la izquierda marxista y los socialdemócratas\footnote{Sobre este punto, en la Sección \ref{sec2.3}, Rawls desarrolla un contraste similar entre la democracia de pequeños propietarios y el capitalismo de Estado de bienestar. Los socialdemócratas que Cohen tiene en mente en este caso favorecían la consecución de un capitalismo de Estado de bienestar.}. A la hora de plantear objeciones morales a la economía de mercado, los socialdemócratas argumentan que en este tipo de sociedades, los más débiles padecen niveles de privación elevados que podrían ser mitigados mediante, por ejemplo, el establecimiento de un sistema de previsión social. Este tipo de crítica respecto de algunas consecuencias no deseadas de las economías de mercado no llega a ser una crítica respecto de la injusticia de las instituciones implicadas. Cohen argumenta que para que se invoquen cuestiones asociadas a la justicia, la crítica debería plantear que es por la operativa del mercado que algunas personas se ven privadas de \say{sus derechos sobre aquello que moralmente debería considerarse un bien común} (\citeyear[p. 40]{Cohen_2014a}). 

En la misma línea, para que los socialdemócratas esbocen una crítica a las consecuencias de la economía de mercado en el bienestar de las personas, se debería poner en tela de juicio a la causa de dichas consecuencias, es decir, a la economía de mercado en sí misma. Lo que sucede es que desde la perspectiva de la socialdemocracia, no se cuestiona dicha estructura institucional y se podría decir que se toman sus consecuencias como algo dado. Según Cohen, los defensores de esta corriente se abstienen de profundizar en este debate debido a que se entraría en un terreno más radical, el cual no quieren recorrer. El autor sugiere, de manera exagerada, que los socialdemócratas \say{parecen sensibles a los efectos de la explotación sobre las personas, pero no al fenómeno de la explotación en sí. Quieren socorrer a los que explotados y minimizar cualquier posible confrontación con aquellos que los explotan} (\citeyear[p. 43-44]{Cohen_2014a}). En contraposición a esta postura, Cohen sostiene que la crítica socialista al capitalismo se basa en la justicia de sus instituciones. En particular, dicha objeción plantea que la economía de mercado \say{permite la propiedad privada de medios de existencia que nadie tiene el derecho a poseer de manera privada, basándose por ende sobre un fundamento injusto} \citep[p. 41]{Cohen_2014a}.


Habiendo presentado el giro normativo de Cohen, a continuación se desarrolla la crítica realizada por el autor al principio de la diferencia desarrollado por Rawls. Esta crítica se centra en analizar el argumento por el cual se considerarían justos los incentivos diferenciales para motivar la contribución de las personas.


\subsection{La crítica al argumento de los incentivos}

\begin{quote}
-\textit{Mejor situado:} \say{Mire, conciudadano, trabajaré duro y mejoraré tanto a usted como a mí, siempre que obtenga una parte mayor que usted.}

-\textit{Peor situado:} \say{Pues eso es bastante bueno; pero ¿pensé que estabas de acuerdo en que la justicia requiere igualdad?}

-\textit{Mejor situado:} \say{Sí, pero eso es solo como punto de referencia, ya ves. Para que estemos aún mejor, los dos, usted entiende, se pueden requerir pagos de incentivos diferenciales a personas como yo.}

-\textit{Peor situado:} \say{Vaya. Bueno, ¿qué los hace necesarios?}

-\textit{Mejor situado:} \say{Lo que los hace necesarios es que no trabajaré tan duro si no consigo más que usted.}

-\textit{Peor situado:} \say{Bueno, ¿por qué no?}

-\textit{Mejor situado:} \say{No lo sé... Supongo que esa es la forma en la que estoy constituido.}

-\textit{Peor situado:} \say{Lo que significa que realmente no te importa mucho la justicia, ¿eh?}

-\textit{Mejor situado:} \say{Em, no, supongo que no.}\footnote{\textit{Rawls on Equal Distribution of Wealth} \citep[p. 287-288]{Narveson_1978}: «\textit{Well-off:} \say{Look here, fellow citizen, I'll work hard and make both you and me better off, provided I get a bigger share than you.} \textit{Worse-off:} \say{Well, that's rather good; but I thought you were agreeing that justice requires equality?} \textit{Well-off:} \say{Yes, but that's only as a benchmark, you see. To do still better, both of us, you understand, may require differential incentive payments to people like me.} \textit{Worse-off:} \say{Oh. Well, what makes them necessary?} \textit{Well-off:} \say{What makes them necessary is that I won't work as hard if I don't get more than you.} \textit{Worse-off:} \say{Well, why not?} \textit{Well-off:} \say{I dunno... I guess that's just the way I'm built.} \textit{Worse-off:} \say{Meaning, you don't really care all that much about justice, eh?} \textit{Well-off:} \say{Er, no, I guess not.}»}
\end{quote}
\vspace{3mm}

La cita con la que comienza esta sección\footnote{Esta misma cita se encuentra también en \citet{Cohen_2008}.} puede resumir en buena medida la crítica de Cohen al \textit{argumento de los incentivos} presentado por diversos autores dentro de los cuales cabe destacar a Rawls (\citeyear{Rawls_1971}, \citeyear{Rawls_2002}). La crítica de Cohen sostiene que resulta contradictorio que personas que dicen estar motivadas por una idea de justicia al mismo tiempo exijan el pago de incentivos diferenciales para esforzarse bajo el argumento de que, si se brindan incentivos, todas las personas se verían beneficiadas en mayor o menos medida. La contradicción se presenta cuando una persona está motivada por un egoísmo contrario a la idea de igualdad como también por una idea de justicia que involucra la igualdad.  

Antes de entrar en el detalle de la crítica de Cohen, es necesario esclarecer algunas cuestiones. El argumento de los incentivos puede presentarse de diversas formas, por ejemplo, como defensa del capitalismo. Cuando la lógica mercantil basada en la propiedad privada es preponderante, los empresarios pueden disponer de sus bienes de la manera que mejor les parezca. Esto genera condiciones productivas favorables, en contraposición a situaciones donde existen restricciones a la iniciativa lucrativa. A su vez, se sostiene que hasta los desposeídos de bienes se ven beneficiados por el funcionamiento del libre mercado. En este contexto, Cohen plantea que aparece la noción de los incentivos ya que \say{interferir con la tendencia natural a que las ganancias se acumulen en manos de aquellos que disfrutan de la riqueza y de posiciones elevadas frena su creatividad como inversores, empresarios y generentes, lo que perjudica a todos} (\citeyear[p. 29]{Cohen_2014a}).

Esta formulación del argumento, no involucra cuestiones de justicia sino que enuncia las bondades de ciertos arreglos institucionales de las economías de mercado que terminarían beneficiando a todas las personas. Otra forma de presentar dicho argumento se logra mediante el postulado de que, en algunas situaciones, los incentivos diferenciales son justos. En este caso, \citet[p. 19]{Cohen_2008} plantea que el argumento puede tomar la siguiente forma:

\vspace{3mm}
\begin{enumerate}
    \item Las desigualdades son injustas a no ser que sean necesarias para hacer que las personas peores situadas se encuentren mejor. En tal caso son justas.
    \item Brindar incentivos desiguales a las personas productivas es necesario para hacer que las personas peores situadas se encuentren mejor.
    \item Por lo tanto, el pago de incentivos desiguales es justo\footnote{\say{1. Inequalities are unjust unless they are necessary to make the worst off people better off, in which case they are just. 2. Unequalizing incentive payments to productive people \textit{are} necessay to make the worst off people better off. 3. Therefore, unequalizing incentive payments are just} \citep[p. 19]{Cohen_2008}.}.
\end{enumerate}
\vspace{3mm}

Es contra esta versión del argumento que Cohen desarrolla la mayor parte de sus críticas. Según el autor, el argumento de los incentivos puede emplearse a la hora de presentar algún tipo de defensa de las desigualdades materiales existentes entre las personas. A la hora de elaborar una defensa de la desigualdad, \citet{Cohen_2001} sostiene que existen dos tipos de argumentos: normativos o fácticos. Un argumento del tipo fáctico plantea que las desigualdades materiales entre las personas resultan inevitables ya que los individuos se comportan mayoritariamente de manera egoísta. Este tipo de comportamiento se vincula con una explicación que puede apelar a una naturaleza humana o a una explicación sociológica. 

\citet{Cohen_2001} rechaza la explicación basada en una naturaleza humana egoísta ya que comparte la creencia marxista de que son las estructuras sociales las que determinan la conciencia de las personas y no en el sentido contrario. A su vez, el autor rechaza la explicación sociológica ya que, aunque las personas sean egoístas, se podrían concebir otros arreglos institucionales que no tuvieran como resultado la desigualdad. De todas formas, Cohen reconoce que antes su postura respecto del rechazo de la explicación sociológica era más contundente; ya no se encuentra tan convencido de que solamente modificando los arreglos institucionales se pueda lograr la igualdad. El autor muestra una mayor simpatía respecto de la noción de que para superar la desigualdad es necesario \say{que haya una revolución en el sentimiento o en la motivación, en oposición a una (mera) revolución en la estructura económica}  \citep[p. 163]{Cohen_2001}.

Por otro lado, un argumento de tipo normativo sostiene que las desigualdades son justas. En este caso, el argumento de los incentivos presentado anteriormente podría ser considerado como un ejemplo de una defensa normativa de la desigualdad. En particular, a través de la idea del principio de la diferencia, Rawls (\citeyear{Rawls_1971}, \citeyear{Rawls_2002}) plantea que las desigualdades que mejoran las condiciones de vida de los peores situados, son justas.

A modo ilustrativo, supongamos que tenemos una sociedad conformada por dos personas en la cual se busca igualar respecto a único bien en particular, como puede ser el salario percibido, entonces podríamos establecer el siguiente escenario:


\vspace{3mm}
\begin{table}[H]
\caption{Ejemplo del argumento de los incentivos}
\centering
\begin{tabular}{ccc|c}
                     & \textbf{A} & \textbf{} & \textbf{B} \\ \hline
\textbf{Situación 1} & 50         &           & 50         \\
\textbf{}            &            &           &            \\
\textbf{Situación 2} & 50 + $j$     &           & 50 + $i$    
\end{tabular}
\end{table}
\vspace{3mm}





Supongamos que la única diferencia relevante entre A y B es que A es más talentosa que B en el sentido de que puede conseguir los mismos resultados que B empleando un menor esfuerzo. En la Situación 1, tanto A como B reciben 50 unidades monetarias; si en la Situación 2 tenemos que A busca realizar un esfuerzo mayor, el argumento de los incentivos nos diría que dicho esfuerzo se realizaría solamente cuando\footnote{En este caso se podría cuestionar que a la persona le interese meramente sus ingresos y que no importe la comparación con los demás; de todas formas, lo que motivaría la acción en este caso es la idea de recibir mayores ingresos.} $50 + j > 50$. Por otro lado, el principio de la diferencia nos diría que el escenario donde $50 + j \neq 50 + i$ y $j > i$, es justo sí y sólo sí $i > 0$. Es decir, la persona peor situada, que en este ejemplo es B, se beneficia de los pagos diferenciales que se le proporcionan a A. 

Un problema que puede surgir de este sencillo ejemplo es que la única condición que se explicita es que $i >0$, por lo que el valor de $j$ podría tomar valores elevados en comparación al valor de $i$. Este tipo de crítica puede encontrarse en el planteo de Cohen. De todas formas, este tipo de argumento, al menos a la hora de ser una crítica al planteo de Rawls, no considera que cuando el principio de la diferencia interactua con el principio de igualdad de oportunidades se conforma una especie de conjunto de situaciones en donde estos principios no entrarían en conflicto. En otras palabras, la situación en donde $j$ toma valores muy elevados en comparación con $i$ no sería posible; podríamos decir que existe un valor $k > 0$ tal que si $j - i \leq k$ no se violaría el principio de igualdad de oportunidades y se estaría cumpliendo con el principio de la diferencia.

Ahora bien, Cohen cuestionaría que, incluso en el intervalo delimitado por $k$, existe una contradicción de la teoría ya que sigue existiendo la conjunción de que las personas que dicen estar motivadas por una idea de igualdad exigen pagos diferenciales dando cuenta de motivaciones egoístas contrarias a la igualdad. Es por esta razón que el autor sostiene que el argumento de Rawls detrás del principio de la diferencia no debe ser concebido como una defensa normativa de la desigualdad sino como un argumento fáctico.



El punto de Cohen es que no solamente la estructura legal en la cual los individuos interactúan es lo que importa a la hora de buscar la igualdad, sino que hay que tener en cuenta las elecciones que las personas realizan dentro de dicha estructura. En este sentido es que el autor adhiere al \say{slogan} de \say{lo personal es político}. Según \citet{Cohen_2001}, es posible vincular este punto con la crítica que se realiza desde el feminismo a las posturas liberales a las que pertenece Rawls. Dicha crítica plantea que, desde una perspectiva liberal, podría darse el caso de que exista una división del trabajo y relaciones de poder que sean sexistas e injustas, incluso cuando el sistema legal que regula a la sociedad no muestre dicho sesgo sexista. Al abstraer las cuestiones asociadas al género de la crítica feminista, se obtiene la idea de que \say{las opciones no reguladas por la ley caen dentro de los límites básicos de la justicia} \citep[p. 167]{Cohen_2001}.

Cabe destacar que la crítica de Cohen se centra en la forma en que se aplica el principio de la diferencia ya que, para el autor, se estarían justificando desigualdades que perjudicarían a las personas peores situadas. Existe una discrepancia con Rawls \say{sobre el asunto de \textit{qué} desigualdades pasan el test que justifica la desigualdad según el principio y, por tanto, \textit{cuánta} desigualdad admite ese test} \citep[p. 169]{Cohen_2001}. Retomando la formulación del principio de la diferencia, dado que las remuneraciones diferenciales son necesarias como motivación para las personas más talentosas, es que Cohen plantea que dicha necesidad gira en torno a una \textit{opción} que las personas talentosas pueden tomar. 

Como en el ejemplo analizado, es posible concebir a una persona talentosa como una que cuente con la capacidad de conseguir ganancias significativas en el mercado y que pueda variar su productividad en función de la remuneración que percibe. Para el autor, en múltiples ocasiones el hecho de que una persona se encuentre en una situación tan favorable puede deberse a circunstancias que están fuera del control del individuo. Por lo tanto, Cohen argumenta que:

\vspace{3mm}
\begin{quote}
    Si una persona logra producir más que otras, esto se debe a que es más talentosa, hizo un mayor esfuerzo o tuvo suerte en las circunstancias de producción, lo que equivale a decir que tuvo suerte respecto de aquellos y aquello \textit{con} que produce. Esta última razón para una mayor productividad, la concurrencia de circunstancias afortunadas resulta moralmente (y no económicamente) ininteligible como para alzarse con una recompensa mayor. Y si bien recompensar una productividad ligada a un mayor despliegue de talento resulta de hecho moralmente inteligible desde cierta perspectiva ética, aún así implica una idea profundamente antisocialista, condenada con razón por J. S. Mill como una instancia de \say{dar a los que ya tienen} (Mill, 1848, p. 210) \footnote{Mill, J. S. (1848). \say{Principles of Political Economy} en J. M. Robson (ed.), \textit{The Collected Works of John Stuart Mill}, vols. II y III, Toronto, University of Toronto Press, Londres, Routledge \& Kegan Paul, 1965 [ed. cast.: \textit{Principios de Economía Política con algunas de sus aplicaciones a la Filosofía Social}, México, FCE, 1951].}, en la medida en que un talento mayor en sí mismo es una fortuna que no exige mayor recompensa \citep[p. 65]{Cohen_2014b}.
\end{quote}
\vspace{3mm}

A su vez, el autor reconoce que el argumento de los incentivos, puede presentar cierto atractivo cuando se desarrolla de manera impersonal. Ahora bien, \citet{Cohen_2008}\footnote{En esta parte sigo el trabajo de Cohen: \textit{Incentives, Inequality, and Community} (\citeyear{Cohen_1992}) que se encuentra en \textit{Rescuing Justice and Equality} (\citeyear{Cohen_2008})} argumenta que si nos situamos en un caso en donde una persona talentosa tiene que defender sus incentivos diferenciales ante una persona peor situada, resulta extraño para este último encontrar una razón satisfactoria del comportamiento del primero\footnote{Una instancia de esto es la cita con la que comienza esta subsección.}. Antes de examinar el argumento enunciado en primera persona podemos preguntar lo siguiente: ¿qué es lo que hace que los incentivos sean necesarios? El autor maneja dos posibilidades: o las personas talentosas no quieren trabajar de la misma manera si se le retiran los incentivos diferenciales o no podrían trabajar de la misma manera por más de que tengan intenciones de hacerlo. Una explicación podría ser que estas personas necesitan de bienes caros para lograr altos niveles de desempeño o quizás necesitan las recompensas elevadas como forma de motivarse. Aún así, Cohen sostiene que este tipo de argumento presenta a las personas talentosas como más débiles de lo que realmente podrían ser. En particular, ya que parecería ser el caso que no pueden modificar su comportamiento.  

Respecto de la idea de que las recompensas son necesarias como motivación, podría pensarse que es una razón en tanto que, si no existieran dichos incentivos, las personas talentosas sentirían que ciertas expectativas que tenían no se estarían cumpliendo. En este sentido, Cohen plantea que en las economías de mercado, puede ser que las personas talentosas, a través de su experiencia y mediante ejemplos de otras personas talentosas, hayan desarrollado la creencia de que los talentos deben ser altamente recompensados. Nuevamente, todo gira en torno a la elección que estas personas deciden realizar ante estas expectativas lo que hacen que los incentivos sean necesarios. Por lo tanto, es que \citet{Cohen_2008} concluye que el argumento de la inhabilidad de los más talentosos depende solamente de sus hábitos y creencias normativas; las cuales pueden modificarse.

Habiendo dicho esto, \citet{Cohen_2008} plantea que el argumento de los incentivos no logra conseguir una justificación comprehensiva. Según el autor, un argumento para apoyar una política cuenta con una justificación comprehensiva cuando logra superar un \textit{test interpersonal}. Mediante este test, se pone a prueba la robustez del argumento en cuestión cambiando a las personas que lo enuncian como también a las personas a las cuales se les enuncia dicho argumento. Lo que puede ocurrir es que:

\vspace{3mm}
\begin{quote}
    Si, \textit{debido} a quién lo está presentando, y/o a quién se lo presenta, el argumento no puede servir como una justificación de la política, entonces pase o no como tal bajo otras condiciones dialógicas, no logra (\textit{tout court}) proporcionar una justificación comprehensiva de la política\footnote{\say{If, \textit{because} of who is presenting it, and/or to whom it is presented, the argument, cannot serve as a justification of the policy, then whether or not it passes as such under other dialogical conditions, it fails (\textit{tout court}) to provide a comprehensive justification of the policy} \citep[p. 42]{Cohen_2008}.} \citep[p. 42]{Cohen_2008}.
\end{quote}
\vspace{3mm}

Detrás de la idea de justificación comprehensiva, subyace la idea de una comunidad justificatoria: \say{una comunidad justificatoria es un conjunto de personas entre las cuales prevalece una norma (que no siempre puede ser satisfecha) de justificación comprehensiva}\footnote{\say{A justificatory community is a set of people among whom there prevails a norm (which need not always be satisfied) of comprehensive justification} \citep[p. 43]{Cohen_2008}.} \citep[p. 43]{Cohen_2008}. El punto del autor es que el argumento de los incentivos no logra superar el test interpersonal cuando se enuncia desde las personas más talentosas hacia las personas peores situadas. Específicamente, esto se debe a que los primeros no pueden justificar razonablemente el motivo por el cual modifican su comportamiento. Según Cohen, cuando se emplea el argumento de los incentivos para defender ciertas desigualdades, se tiene en mente un modelo de sociedad que carece de los elementos que conforman una comunidad justificatoria. 

Para ejemplificar esto, Cohen propone que imaginemos una situación en la que ejecutivos y profesionales de alta jerarquía se encuentran dialogando con trabajadores que reciben un salario bajo y/o personas que, por una razón u otra, dependen de las ayudas del estado de bienestar. En este caso, es posible tomar un ejemplo que presenta \citet{Cohen_2008} respecto de una modificación al esquema impositivo. Para presentar una defensa a la reforma impositiva ante las personas desfavorecidas, los ejecutivos podrían argumentar de la siguiente manera:

\vspace{3mm}
\begin{quote}
    Las políticas públicas deberían hacer que las personas peores situadas (en este caso, como sucede, ustedes) se encuentren mejor.
        
    Si el impuesto máximo sube al 60 porciento, nosotros trabajaremos menos, y, como resultado, la posición de los pobres (su posición) será peor.
         
    Por lo tanto, el impuesto máximo sobre nuestros ingresos no debería aumentarse al 60 porciento\footnote{\say{Public policy should make the worst off people (in this case, as it happens, you) better off. If the top tax goes up to 60 percent, we shall work less hard, and, as a result, the position of the poor (your position) will be worse. So the top tax on our income should not be raised to 60 percent} \citep[p. 59]{Cohen_2008}.} \citep[p. 59]{Cohen_2008}.
    
\end{quote}
\vspace{3mm}

Ahora bien, ante esta presentación en primera persona del argumento, una persona pobre podría preguntarle al talentoso rico respecto de cuál sería su justificación para trabajar menos cuando el impuesto sube. Ante esta exigencia de justificación, las personas talentosas no pueden invocar la idea de que las desigualdades son necesarias para que los pobres estén mejor porque son ellos mismos los que las hacen necesarias. Al querer argumentar de esta manera, Cohen plantea que podría darse un caso de alienación respecto de los talentosos ricos sobre su capacidad de actuar de otra manera. Una posible respuesta del talentoso rico podría ser algo como: \say{Mira, simplemente no valdría la pena trabajar tan duro si la tasa de impuestos fuera más alta, y si estuvieran en nuestro lugar se sentirían de la misma manera}\footnote{\say{Look, it simply would not be worth our while to work that hard if the tax rate were any higher, and if you were in our shoes you would feel the same way} \citep[p. 60]{Cohen_2008}.} \citep[p. 60]{Cohen_2008}.

Por su parte, dentro de las personas peores situadas, alguien podría contestar que, si estuvieran en la misma situación que las personas talentosas, no necesariamente se comportarían de la misma manera. Al verse frustrado el argumento por parte de la persona talentosa, ésta podría alegar que todas las personas poseen un derecho a perseguir ciertos intereses, que pueden ser denominados como egoístas, dentro de un nivel razonable. Según Cohen, es cierto que las personas tienen derecho a perseguir sus intereses pero \say{un modesto derecho al interés propio parece insuficiente para justificar el rango de desigualdad, los extremos de riqueza y pobreza, que realmente prevalecen en la sociedad en cuestión}\footnote{\say{[...] a modest right of self-interest seems insufficient to justify the range of inequality, the extremes of wealth and poverty, that actually obtain in the society under discussion} \citep[p. 61]{Cohen_2008}.} \citep[p. 61]{Cohen_2008}. Entonces parece ser el caso de que no existe un argumento satisfactorio que pueda presentarse desde la perspectiva de las personas talentosas y que la \textit{necesidad} de los incentivos radica exclusivamente en la voluntad de estas personas de no esforzarse de la misma manera.

En la misma linea, \citet{Cohen_2001} sostiene que el argumento de los incentivos implica una aplicación distorsionada del principio de la diferencia. Esto se debe a que o bien las personas más talentosas aceptan el principio de la diferencia o no lo aceptan. En el caso que las personas más talentosas no aceptaran el principio de la diferencia, tendríamos que dicha sociedad no sería considerada justa desde la perspectiva rawlsiana ya que resulta necesario que las personas acepten los principios de justicia que rigen a la misma. Por otro lado, en el caso alternativo, a las personas más talentosas se les podría formular la pregunta de: por qué \say{exigen un pago mayor del que obtienen aquellos menos dotados por un trabajo que, de hecho puede requerir un talento especial, pero que no es especialmente desagradable} \citep[p. 172]{Cohen_2001}. Cohen argumenta que resultaría difícil para estas personas contestar dicha pregunta y que no podría invocarse al principio de la diferencia. En particular, ya que es la actitud de los más talentosos, de no esforzarse o trabajar de la misma manera, lo que hace que dichos incentivos sean necesarios. Por lo tanto, Cohen concluye que las remuneraciones diferenciales para los más talentosos son \say{necesarias sólo porque las opciones de los más dotados no están debidamente ajustadas al principio de la diferencia} \citep[p. 173]{Cohen_2001}.


Es entonces que Cohen afirma que, debido a los problemas que las motivaciones egoístas generan respecto de las posibilidades de alcanzar un ideal igualitario, una sociedad justa requiere además de las reglas formales un \textit{ethos} que fomente la igualdad. A su vez, Cohen argumenta que de no existir dicho \textit{ethos}, se producirían desigualdades que no ayudarían a los que están peor situados. Este \textit{ethos} \say{fomenta una distribución más justa de lo que las reglas del juego económico pueden asegurar por sí mismas} \citep[p. 174]{Cohen_2001}.



Sin embargo, es posible presentar una objeción al argumento de Cohen. Dicha objeción ha sido denominada por el autor como \say{la objeción de la estructura básica}. La misma plantea que los principios de justicia no deberían aplicarse a las elecciones personales, sino que su ámbito de aplicación es la estructura básica de la sociedad. Para responder a esta crítica, el autor elabora dos posibles respuestas. Respecto de la primera respuesta, es posible encontrar fragmentos del trabajo de Rawls en donde se pueden evidenciar ciertas contradicciones con la idea de que los principios de justicia se aplican exclusivamente en la estructura básica de la sociedad. Concretamente, \citet{Cohen_2001} presenta tres instancias donde se presenta una contradicción: 

\vspace{3mm}
\begin{enumerate}
    \item Cuando se cumple con el principio de la diferencia, la sociedad demuestra fraternidad (en un sentido fuerte: las personas no quieren tener grandes ventajas entre sí).
    \item Cuando rige el principio de la diferencia, los peores situados llevan su situación con dignidad, puesto que no sería posible una mejora material para ellos.
    \item Cuando en una sociedad justa se actúa con un sentido de justicia, las personas aplican en sus vidas propias los principios de justicia.
\end{enumerate}
\vspace{3mm}

Ante la situación que se presenta en 1, Cohen argumenta que la fraternidad que se enuncia no sería posible si los principios se centraran exclusivamente en la estructura básica. En especial, ya que no se censurarían las conductas egoístas que terminarían creando desigualdades significativas. Respecto de 2, Cohen menciona que resulta falsa ya que podría lograrse una mejora de la situación de los peores situados si las personas talentosas no fueran egoístas. Finalmente, sobre 3, el autor plantea la pregunta sobre por qué sería necesario que las personas aplicaran los principios de la justicia a sus elecciones diarias si lo que importa, desde la propuesta de Rawls, es que las mismas se encuentren dentro del marco que establece la estructura básica.

De todas formas, esta primera respuesta de Cohen no resulta tan contundente ya que en cada uno de estos casos, existe una salida para Rawls: o elimina la restricción respecto del ámbito de aplicación de los principios de justicia o elimina los comentarios 1, 2 y 3. Es en esta última dirección que procede Rawls respecto del comentario 1, ya que para éste el comentario implicaría ir en la dirección de postular una concepción comprehensiva de la justicia, es decir, una concepción moral completa no meramente política. No obstante, esta salida tiene un costo ya que \say{no se puede decir que los ideales de dignidad, fraternidad y la total realización de las naturalezas morales de la gente se expresan a través de la justicia rawlsiana} \citep[p. 183]{Cohen_2001}.

Debido a esto es que Cohen desarrolla una segunda respuesta a la objeción de la estructura básica. Para el autor esta respuesta lograría mostrar que la justicia requiere \say{un \textit{ethos} que gobierne las elecciones diarias, un \textit{ethos} que va más allá de la obediencia a las reglas justas} \citep[p. 184]{Cohen_2001}. Esta respuesta gira en torno a indagar respecto de qué es lo que conforma la estructura básica, es decir, cuáles instituciones son las que definen dicha estructura. Una primera posibilidad es que la estructura básica consista en las instituciones de carácter coercitivo-legal, es decir, aquellas que delimitan el comportamiento desde el punto de vista legal. 

Sin embargo, Cohen argumenta que esta no parece ser la caracterización correcta. En particular, ya que en el planteo de Rawls se dice que la estructura básica \say{consiste en las instituciones sociales \textit{más importantes} y no pone énfasis particular en la coerción cuando anuncia \textit{esa} especificación de la estructura básica} \citep[p. 185]{Cohen_2001}. Ahora bien, si se define a la estructura básica de esta manera, podrían incluirse instituciones \textit{importantes} que se apartan de las dependen exclusivamente de la ley. En especial, un caso paradigmático es el de la familia. Según Cohen, en el planteo rawlsiano a veces se incluye a la familia dentro de la estructura básica y a veces no. En el caso de incluir a instituciones que dependen más de la convención y la expectativa, como la familia, no se podría argumentar que hay que excluir del ámbito de la justicia a las elecciones personales no restringidas desde el punto de vista legal. Concretamente, ya que dichas instituciones se rigen a partir de las elecciones que las personas realizan en su vida cotidiana y algunas elecciones son censuradas dentro de las convenciones sociales de carácter informal.

Una posible salida desde la perspectiva rawlsiana, sería nuevamente volver a una caracterización meramente coercitiva. Sin embargo, cabe recordar que Rawls pone el foco en la estructura básica de la sociedad como el asunto principal de la justicia debido a sus efectos profundos sobre la configuración de las instituciones sociales. Por lo cual, Cohen argumenta que \say{es falso que sólo la estructura \textit{coercitiva} cause efectos profundos, como una vez más nos recuerda el ejemplo de la familia} (\citeyear[p. 187]{Cohen_2001}). Los efectos de la estructura informal respecto de las cuestiones de justicia se puede evidenciar a través de dos ejemplos: las elecciones dentro del mercado y la familia.

Por un lado, respecto de la familia, se tiene un ámbito en donde podría decirse que se distribuyen las cargas y beneficios de la asociación de las personas que conforman un hogar. Dentro de la familia existen prácticas que no están definidas por la ley que tienen un impacto importante sobre las posibilidades vitales de las personas involucradas. Sin embargo, si definimos a la estructura básica como exclusivamente coercitiva-legal, tendríamos que las prácticas que se reproducen dentro de la familia, que pueden tener un sesgo sexista, quedan por fuera del ámbito de la justicia. Por otro lado, en el caso de las elecciones dentro del mercado, podría concebirse una legislación que maximice la cuantía de los bienes primarios de la sociedad y lograra cumplir con el principio de la diferencia. Según Cohen, dicha situación es compatible con un \textit{ethos} maximizador que puede producir \say{grandes desigualdades y un escaso nivel de abastecimiento para los que peor están; no obstante Rawls tiene que declarar que esas dos cosas son justas si mantiene una concepción coercitiva de lo que la justicia juzga} \citep[p. 189]{Cohen_2001}. Una manera concisa de formular el punto de Cohen se evidencia en la siguiente pregunta: 

\vspace{3mm}
\begin{quote}
   ¿Por qué nos preocupa de manera tan desproporcionada la estructura básica coercitiva, cuando la principal razón para que nos preocupe, su impacto sobre las vidas de las personas, es también una razón para preocuparnos por la estructura informal y los criterios de elección personal? \citep[p. 190]{Cohen_2001}. 
\end{quote}
\vspace{3mm}

Entonces Cohen concluye que, debido al impacto que tiene la estructura informal sobre las posibilidades de realización del ideal de igualdad, resulta necesario un \textit{ethos} que influya sobre las elecciones personales además de una correcta formulación de la estructura formal de la sociedad. Resulta apropiado preguntarse en qué consiste este \textit{ethos}. El concepto de \textit{ethos} de una sociedad consiste en \say{un grupo de sentimientos y actitudes en virtud del cual su práctica normal y sus presiones informales son lo que son} \citep[p. 197]{Cohen_2001}. El autor reconoce que las presiones informales carecen de fuerza cuando no hay una práctica normal establecida a través de reglas que dichas presiones buscan hacer cumplir. 

Finalmente, Cohen presenta el ejemplo de las desigualdades económicas entre el Reino Unido y Estados Unidos como una instancia de cómo diferentes \textit{ethos} pueden generar ciertos comportamientos. En los primeros años posteriores a la Segunda Guerra Mundial, en los dos países considerados existía una economía de mercado. Además, se daba el caso de que la diferencia de los ingresos entre un ejecutivo de alta jerarquía y un obrero era mayor en Estados Unidos que en el Reino Unido. De todas formas, el autor estipula que:

\vspace{3mm}
\begin{quote}
    [...] muchos de los ejecutivos británicos \textit{no} habrían sentido la tentación de decir: nosotros deberíamos presionar para ganar más, puesto que había un \textit{ethos} de reconstrucción después de la guerra, un \textit{ethos} de proyecto común, que moderó el deseo de ganancia personal \citep[p. 195]{Cohen_2001}
\end{quote}
\vspace{3mm}

Habiendo presentado la faceta crítica de Cohen respecto de la posibilidad de conciliar incentivos y justicia, a continuación, se presenta el desarrollo de la propuesta del autor.


\subsection{La propuesta de Cohen}

A la hora de formular la propuesta de Cohen, se consideran dos trabajos en particular. En primer lugar, en el ensayo \textit{¿Por qué no el socialismo?} \citet{Cohen_2014c} presenta un modelo básico de un campamento para establecer un caso en donde se preferiría un tipo de organización socialista. A su vez, el autor desarrolla los principios normativos que se encuentran en el trasfondo de dicho modelo. En segundo lugar, en \textit{Un retorno a los fundamentos del socialismo} \citep{Cohen_2014b}, el autor analiza un documento elaborado en 1993 por un \textit{think-tank} cercano al Partido Laborista de Inglaterra. En dicho documento, se planteaba la necesidad de emular el ascenso que había experimentado la derecha en la década anterior. A su vez, se argumenta que para lograr dicho objetivo, no resultaba necesario entrar en una discusión filosófica respecto de los principios de la izquierda. Cohen discrepa con esta idea y sostiene que resulta de vital importancia que la izquierda se reapropie de sus valores fundamentales.

En el contexto del modelo del campamento \citep{Cohen_2014c}, los participantes no tendrían razones para organizarse según jerarquías. Todos comparten un objetivo común que consiste en pasarla bien cada uno haciendo las actividades que más les guste hacer ya sea individualmente o en grupo. Existen instrumentos para llevar a cabo dichas actividades y éstos son aprovechados colectivamente. A su vez, se distribuyen las tareas según los intereses de las personas:

\vspace{3mm}
\begin{quote}
    Uno pesca, otro prepara la comida y otro cocina. Aquellas personas que odian cocinar pero disfrutan lavar, lavan, y así sucesivamente. Somos muy diferentes, pero nuestros acuerdos mutuos y espíritu de nuestro emprendimiento aseguran que no haya desigualdades en las cuales alguien pudiera fundar una queja \citep[p. 180]{Cohen_2014c}.
\end{quote}
\vspace{3mm}

A continuación Cohen plantea la posibilidad de que sea otro el modo en que se organice el campamento. Por ejemplo, empleando una lógica más individualista. En esta nueva forma de organizar las tareas, Cohen presenta algunos escenarios de los cuales se destaca el siguiente: 

\vspace{3mm}
\begin{quote}
    A Harry le encanta pescar, y Harry es muy buen pescador. Por consiguiente, él aporta más pescado que los demás. Harry dice: \say{El modo en que estamos manejando las cosas es injusto. Yo debería comer el mejor pescado. Yo debería comer trucha\footnote{En esta traducción realizada por Luciana Sanchez, Roberto Gargarella, Félix Ovejero y Verónica Lifrieri, con el objetivo de facilitar la comprensión del ejemplo, se cambia la palabra \say{perch}, que es un tipo de comida regional estadounidense, por pescado.}, no la mezcla de trucha y bagre que todos comimos hasta ahora}. Pero sus compañeros le dirían: \say{Oh, Dios, Harry, no seas tan cretino. Te esfuerzas y transpiras tanto como nosotros. Claro que eres muy buen pescador. Nosotros no despreciamos este don especial que tienes, que en realidad constituye una fuente de satisfacción para ti; pero ¿por qué deberíamos recompensarte por esta habilidad preexistente?} (\citeyear[p. 181-182]{Cohen_2014c}).
\end{quote}
\vspace{3mm}

Este ejemplo busca mostrar que, en un campamento, nuestras intuiciones parecen entrar en conflicto con la forma de organizar las cosas de manera más individualista. A su vez, esta idea se vincula con el punto anterior de que los incentivos diferenciales van en contra de la idea de la igualdad.  

Luego de presentar este modelo del campamento, Cohen procede a presentar dos principios que se manifiestan en dicho modelo: un principio de igualdad y un principio comunitario. Según el autor, el principio asociado a una idea de comunidad \say{restringe la operatividad del principio de igualdad, al prohibir determinadas inequidades en los resultados que el principio de igualdad permite} \citep[p. 183]{Cohen_2014c}. En la misma línea, volviendo sobre los principios socialistas, se puede decir que \say{el principio de igualdad sostiene que en términos generales la cantidad de cargas y beneficios que tiene una persona en su vida debería ser comparable a la de cualquier otra} \citep[p. 62]{Cohen_2014b}.

En el contexto del modelo del campamento, Cohen tiene en mente un principio de igualdad de oportunidades. Para presentar dicho principio, el autor desarrolla unos pasos previos. En primer lugar existe una \say{igualdad de oportunidades burguesa}. Bajo este tipo de ideal, se quitarían las restricciones socialmente construidas respecto de las oportunidades de vida \say{causadas por una asignación específica de derechos y por una percepción social intolerante y perjudicial} \citep[p. 184]{Cohen_2014c}. En segundo lugar, tendríamos una \say{igualdad de oportunidades de la izquierda liberal}. En este caso se va más allá de la igualdad de oportunidades burguesa y se propone quitar desigualdades que corresponden a las circunstancias de \say{nacimiento  y crianza, cuya restricción obra no por medio de la asignación de sus víctimas de un estatus inferior, sino por su sometimiento a la pobreza y otros medios de privación} \citep[p. 185]{Cohen_2014c}. Finalmente tendríamos la \say{igualdad de oportunidades socialista}, la cual: 

\vspace{3mm}
\begin{quote}
    [...] busca corregir todas las desventajas no elegidas, desventajas que son tales porque el agente no puede ser considerado racionalmente responsable de ellas, ya sea que reflejen desgracias sociales o desgracias naturales. Cuando prevalece la igualdad de oportunidades socialista, las diferencias en el resultado no reflejan más que diferencias de gusto o elección, en vez de reflejar las debidas a capacidades y poderes naturales o sociales \citep[p. 186]{Cohen_2014c}.
\end{quote}
\vspace{3mm}

Cohen sostiene que el principio de igualdad de oportunidades socialista es compatible con tres tipos de desigualdades, aunque en algunos casos existe una tensión problemática. En primer lugar, se encuentran las desigualdades en el acceso a algunos tipos de bienes por partes de las personas. Estas diferencias se explican por la elección de distintos estilos de vida por sobre otros que realizan las personas. Este tipo de desigualdad no representa un problema ya que las personas realizarían estas elecciones en un contexto favorable a la elección de diversos caminos de acción. Los tipos de desigualdad restantes consisten en: 2) desigualdades debidas a diferencias en el esfuerzo que se escoge realizar y 3) desigualdades debidas a las diferencias en la fortuna en la elección.

El segundo tipo de desigualdad se justifica \say{en razón del esfuerzo y/o la preocupación diferente de las personas que se encuentran, al inicio, en perfecta igualdad de condiciones y que son iguales hasta en sus capacidades para emplear su esfuerzo y en su preocupación} \citep[p. 189]{Cohen_2014c}. Finalmente, el último tipo de desigualdad, la basada en la suerte de opción, resulta la más problemática\footnote{Debido a este énfasis en cómo la suerte puede afectar las perspectivas igualitaristas, es que a autores como Cohen y Roemer se los considera pertenecientes a una corriente denominada, por primera vez en \citet{Anderson_1999}, como \textit{igualitarismo de la suerte} o \textit{luck egalitarianism}.}. El ejemplo tradicional es el de una apuesta entre individuos. Según Cohen, este tipo de desigualdad puede ser similar a algunas situaciones que ocurren dentro de los mercados por el accionar de los individuos.

Ahora bien, aunque las desigualdades de tipo 2 y 3 no sean condenadas bajo el principio de igualdad de oportunidades, eso no las vuelve menos reprochables desde una visión socialista. Por lo tanto, Cohen propone un principio comunitario como forma de contrarrestar la magnitud que podrían tomar estos tipos de desigualdades. El principio de comunidad implica que a las personas \say{les importe y, cuando sea necesario y posible, se preocupen por la suerte de los demás. Y también que les importe preocuparse los unos por los otros} \citep[p. 191]{Cohen_2014c}. 

En la misma linea, este principio comunitario era concebido por \citet{Cohen_2014b} como un principio \say{antimercado}. La razón de esto es que las personas que se rigen por este principio cooperan entre sí debido a que reconocen que los demás necesitan de sus servicios y no lo hacen en búsqueda la recompensa que podrían obtener. Este principio resulta antimercado en la medida que el mercado \say{estimula la contribución productiva, no en función del compromiso con los demás y el deseo de servirles y de ser servido \textit{por} ellos a la vez, sino en función de una recompensa monetaria impersonal} \citep[p. 59]{Cohen_2014b}. Si bien es posible reconocer que algunas personas están motivadas por esta idea de servir a los demás, Cohen plantea que dicho ideal no es lo que hace funcionar al mercado como mecanismo de asignación.

Volviendo a su posterior ensayo, \citet{Cohen_2014c} profundiza sobre su forma de entender al principio de comunidad. En particular, el autor presenta dos formas de concebir este principio comunitario. En primer lugar, si las personas no enfrentan adversidades similares, debido a que sus experiencias vitales son muy distintas, es difícil que las personas puedan entender las problemáticas de los otros. Algo de esta índole sucede cuando las personas perciben ingresos muy dispares. Concretamente, esto ocurre cuando las personas peores situadas se enfrentan a ciertas vulnerabilidades sociales que podrían ser en parte remediadas con la ayuda de los mejores situados. En este ejemplo, el principio de comunidad podría ser satisfecho si las personas de mayores ingresos destinaran una porción de los mismos, dentro de unos niveles razonables que no impliquen un sacrificio elevado, para ayudar a las personas pobres.

La segunda forma de concebir al principio comunitario consiste en una forma comunitaria de reciprocidad. Para el autor, esta reciprocidad, en un contexto de igualdad de oportunidades, no se considera como un requisito de la igualdad sino que es un principio \say{para que las relaciones humanas adquieran una forma deseable} \citep[p. 193]{Cohen_2014c}. En la misma línea, Cohen plantea que:

\vspace{3mm}
\begin{quote}
    La reciprocidad comunitaria es un principio negatorio del mercado, un principio conforme al cual yo le sirvo a usted no por lo que pueda llegar a obtener a cambio, sino porque usted necesita mis servicios; por ese mismo motivo, usted me sirve a mí. La reciprocidad comunitaria no es lo mismo que la reciprocidad de mercado, en tanto el mercado incentiva las contribuciones productivas no sobre la base del compromiso con nuestros congéneres, y nuestro deseo de servirles en tanto somos servidos por ellos, sino sobre la base de una recompensa económica \citep[p. 193]{Cohen_2014c}.
\end{quote}
\vspace{3mm}

En términos generales, \citet{Cohen_2014c} argumenta que las motivaciones principales que guían las actividades productivas en una economía de mercado son una combinación de miedo y codicia. Desde la óptica de la codicia, las otras personas son vistas como posibles fuentes de enriquecimiento; mientras que desde el miedo, los demás son vistos como posibles amenazas. Por esta razón es que Cohen afirma que el principio de reciprocidad comunitario se encuentra en las antípodas de una lógica mercantilista.

Ahora bien, habiendo desarrollado estos principios de igualdad y comunidad, Cohen presenta dos preguntas respecto del modelo del campamento: 1) ¿es deseable? y 2) ¿es factible? Respecto de la primera pregunta, Cohen maneja una posible crítica en la cual este modelo sería indeseable. En particular, este modelo no sería deseable debido a que no cuenta con un ámbito en donde las personas puedan elegir ciertas acciones que tengan resultados desiguales o impliquen la instrumentalización de otra persona. Esta crítica no sería tan importante si se tiene en cuenta que siempre va a existir un lugar para la elección de las personas y que incluso en las sociedades de mercado, las elecciones de un individuo se ven limitadas por las elecciones de los demás.

Respecto de la factibilidad, Cohen se pregunta si el egoísmo humano o la tecnología social disponible pueden ser obstáculos a la realización de los ideales considerados. Ante esta pregunta, es posible distinguir dos obstáculos \citep{Cohen_2014c}. En primer lugar, las personas pueden no ser lo suficientemente generosas y cooperativas cuando nos alejamos del ámbito reducido del modelo del campamento. En segundo lugar, aunque las personas mostraran los niveles necesarios de generosidad y cooperación, podría ser el caso de que no conozcamos la forma de hacer que dichos valores fueran el \textit{motor} de la economía. El tipo de factibilidad que le interesa a Cohen no es meramente la de poder implementar un sistema inspirado en los ideales del campamento, sino también cómo lograr que dicho sistema pueda mantenerse en el tiempo. Respecto a este objetivo, el autor sostiene que el principal problema al que se enfrentan los partidarios del modelo del campamento es un problema de diseño institucional. En la misma línea, Cohen argumenta que:  

\vspace{3mm}
\begin{quote}
    Egoísmo y generosidad existen, a fin de cuentas, en (¿casi?) todos. Nuestro problema es que, aunque sabemos cómo hacer funcionar un sistema económico sobre la base del egoísmo, no sabemos cómo hacerlo funcionar sobre la base de la generosidad. Incluso si en el mundo real, en nuestra propia sociedad, muchas cosas dependen de la generosidad, o, para expresarlo de manera más general y más negativo, no dependen de incentivos de mercado \citep[p. 199]{Cohen_2014c}.
\end{quote}
\vspace{3mm}

Ante este tipo de dificultad es que surgen modelos de socialismo de mercado. Estos modelos buscan combinar ciertos aspectos positivos de las economías de mercado, como la eficiencia productiva, con criterios organizacionales o regulatorios del socialismo para poder mitigar los efectos no deseados de los dos sistemas. Si nos situamos dentro de los proyectos socialistas del siglo XIX, una idea primordial era erradicar la organización mercantilista de la economía. Para lograr esto, se proponía implementar una planificación centralizada de la vida económica. Ahora bien, Cohen propone que la experiencia que surgió de la implementación de regímenes socialistas en el siglo XX nos dice que la centralización, al menos en la manera en que fue llevada a cabo, no es capaz lograr el éxito económico una vez que se llega a ciertos niveles de avance en la estructura productiva. Es por esto que se buscó ir más allá de la centralización y se desarrolló un modelo de socialismo de mercado que:

\vspace{3mm}
\begin{quote}
    [...] es socialista porque vence la división entre capital y trabajo: en el socialismo de mercado no existe una clase capitalista que enfrenta a los trabajadores desprovistos de capital, ya que quienes poseen las empresas son los trabajadores. Sin embargo, el socialismo de mercado se diferencia del socialismo tradicional en que estas empresas en propiedad de los trabajadores compiten entre sí y por los consumidores, al estilo de la competencia de mercado \citep[p. 204]{Cohen_2014c}.
\end{quote}
\vspace{3mm}

Aunque estos modelos puedan resultar en una mejora con respecto de los arreglos institucionales vigentes, \citet{Cohen_2014c} plantea que los modelos de socialismo de mercado deben ser vistos como un \say{segundo mejor}. Esto se debe a que: en primer lugar, el ideal de igualdad se ve afectado por la competencia de mercado creando ganadores y perdedores; y en segundo lugar, el ideal de comunidad se ve perjudicado por la lógica mercantil que reduce el ámbito de la reciprocidad.

En este contexto, Cohen realiza una crítica de algunos fervientes defensores de los modelos de socialismo de mercado. El autor sugiere que diversos intelectuales socialistas desarrollaron \say{preferencias adaptativas}, es decir, sus preferencias se vieron distorsionadas por una concepción de lo que consideran factible. Debido a esto: \say{Muchos socialistas llegaron a la conclusión de que el socialismo de mercado es maravilloso simplemente porque creen que no pueden diseñar nada mejor} \citep[p. 205]{Cohen_2014c}.

Queda entonces desarrollada la propuesta de Cohen y su \say{desafío} de pensar formas superiores al socialismo de mercado para poder implementar los ideales socialistas de igualdad y comunidad. A continuación, se desarrolla la propuesta de Roemer en tanto implica un intento de operacionalizar las ideas desarrolladas por Cohen. Este desarrollo lleva a Roemer a presentar una superación del concepto de socialismo de mercado estándar.




