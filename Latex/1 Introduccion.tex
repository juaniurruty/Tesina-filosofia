
\setcounter{page}{1}

La disciplina económica puede ser entendida de diversas maneras. Una formulación ampliamente generalizada es que la misma se encarga de la asignación de recursos escasos. Existen distintos criterios que podrían guiar dicha asignación. En particular, en los análisis económicos, se le brinda un rol sustantivo al criterio de la eficiencia. De manera no tan rigurosa, una forma de presentar el concepto de eficiencia es en contraposición al de eficacia. La eficacia implica el cumplimiento de un objetivo concreto mientras que la eficiencia requiere que dicho cumplimiento sea empleando la menor cantidad de recursos posibles. 


Desde la óptica de la producción de bienes, se podría aceptar provisionalmente que una situación con una mayor cantidad de bienes es preferible respecto de otra en donde hay una cantidad menor. En este sentido, un objetivo por parte de las firmas podría ser el de producir la mayor cantidad de bienes con el menor costo posible. Conseguir esta asignación de recursos productivos sería una forma de garantizar la eficiencia técnica. A la hora de querer operacionalizar esta intuición, los modelos microeconómicos simples llegan a soluciones o equilibrios. En estos equilibrios se logra: por un lado, la maximización de beneficios o la minimización de costos desde la óptica de las empresas\footnote{Bajo los supuestos estándar de la teoría microeconómica básica, resolver el problema de la maximización de beneficios es equivalente a resolver la minimización de costos ya que estos problemas presentan una relación de dualidad \citep{MasColell_1995}.}; por otro lado, desde la óptica de los trabajadores, se logra la maximización de su utilidad tomando en cuenta las decisiones respecto del trabajo y del ocio.


Estos modelos emplean ciertos supuestos para tratar de reducir la complejidad del fenómeno social. En particular, uno de los supuestos más extendidos en la disciplina económica es la idea del egoísmo racional. Este supuesto plantea que los individuos buscan tomar la acción que les brinde mayor utilidad o bienestar\footnote{Si bien cuestionar este supuesto podría ser un tema de investigación en sí mismo, en este trabajo se discutirá este punto de manera secundaria.}. De la aplicación de estos modelos, se llega al resultado de que la maximización de beneficios implica que la productividad marginal de los factores productivos debe ser igual a su remuneración. 

De este resultado se sigue que los salarios son iguales a la productividad marginal del trabajo. Por lo cual, un modelo microeconómico básico plantea que las personas más productivas deben obtener una remuneración mayor. En la misma línea, es posible complejizar la situación introduciendo asimetrías de información entre los agentes. Concretamente, si las firmas desconocen cuál es la productividad de las personas que piensan contratar, alguien puede querer hacerse pasar por una persona que posee una alta productividad. Este ejemplo es denominado por la literatura como selección adversa \citep{Laffont_2002}. Una forma que tienen las firmas para lidiar con este problema es a través de la implementación de mecanismos que induzcan a las personas a revelar su información privada. Por lo general, estos mecanismos involucran el pago de incentivos económicos diferenciales a las personas de mayor productividad.



Con estos dos ejemplos se busca ilustrar el siguiente punto: en búsqueda de soluciones eficientes, la economía como disciplina propone el uso de incentivos económicos y remuneraciones diferenciales para personas con mayor productividad. Existe una corriente dentro de la economía que se deslinda de entrar en el terreno de los juicios normativos. Siguiendo con esta corriente, que se encuentra en autores como \citet{Friedman_1966}, la bondad de la teoría, es decir, de la parte \textit{positiva}, gira en torno a la capacidad de predecir fenómenos. Ocupa un lugar secundario preguntarse si los supuestos empleados resultan \textit{creíbles}. A su vez, se evita entrar en cuestionamientos normativos de los resultados que se derivan, solamente importa la faceta predictiva de la teoría.

Ahora bien, existe una corriente de pensadores dentro de la economía que defiende otro tipo de enfoque. En particular, \citet{Sen_1991} sostiene que la disciplina podría servirse de sus orígenes asociados con la política. Al hacer esto, se pondría en evidencia la naturaleza normativa de la economía y se la enriquece. A su vez, la propia división entre economía positiva y economía normativa ha sido ampliamente criticada. Siguiendo a \citet{Putnam_2004}, es posible argumentar que la distinción entre economía normativa y positiva se sustenta a partir de la dicotomía entre los conceptos de hecho y valor. El argumento central del autor es que, si bien es posible evidenciar una distinción entre los juicios normativos que pueden ser de carácter ético y otros tipos de juicios, esto no implica que exista una dicotomía a nivel metafísico entre los conceptos de hecho y valor. 

En contraposición a una dicotomía, \citet{Putnam_2004} plantea que existe un enredo (\textit{entanglement}) entre dichos conceptos. En el ámbito de la ciencia, es posible evidenciar este enredo si se reconoce que no todos los valores son valores éticos. En particular, existen valores epistémicos (coherencia, plausibilidad, simplicidad, entre otros) que son empleados a la hora de elegir entre hipótesis y teorías que compiten por ser las explicaciones predilectas. 

En la misma linea, \citet{Davis_2022} argumenta que la disciplina económica debe entenderse como normativa. A diferencia de lo que algunas corrientes de economistas creen, dentro de la teoría económica \textit{mainstream} se encuentra un ideal normativo que es central. Concretamente, dicho ideal consiste en la realización del individuo. A su vez, \citet{Davis_2022} sostiene que la teoría económica \textit{mainstream} entiende que la manera de realizar este ideal es mediante la satisfacción de las preferencias de los individuos. 

Habiendo evidenciado la faceta normativa de la economía como disciplina, es posible preguntarse: ¿cómo pueden ser vistas estas propuestas de remuneraciones diferenciales según la productividad de las personas desde una teoría de la justicia? ¿Existe una tensión entre la idea de justicia distributiva y los incentivos económicos diferenciales? A la hora de brindar una respuesta a estas preguntas, es necesario problematizar cómo es que las personas llegan a desarrollar estas diferentes productividades. En dicha discusión entran en juego: talentos, suerte, esfuerzo, entre otras variables. A su vez, se vuelve necesario cuestionar un supuesto empleado en algunas situaciones dentro de la ciencia económica, a saber, los salarios son iguales a la productividad. Una visión más realista sugeriría que la determinación de los salarios gira en torno a variables más allá de la productividad como puede ser: suerte, \textit{networking}, sexo, educación, antigüedad, entre otras variables. En la misma línea, existe numerable evidencia de que las instituciones que rigen el mercado laboral influyen sobre las remuneraciones \citep{Blau_1999}.


Respecto de la pregunta sobre la relación entre los incentivos diferenciales y la justicia distributiva, es posible encontrar una respuesta en el trabajo de John Rawls (\citeyear{Rawls_1971}, \citeyear{Rawls_2002}). En particular, el autor argumenta que algunos factores que influyen sobre los talentos o habilidades de los individuos están fuera de su control. Bajo el concepto de lotería natural, \citet{Rawls_1971} alude a cómo algunas dotaciones de talentos o de recursos de los individuos les son dadas por el contexto en el cual nacen. En este sentido, emplear un sistema que remunera de manera diferente a los más talentosos iría en contra de un ideal de justicia distributiva.

De todas formas, \citet{Rawls_1971} sostiene que existe una manera en la cual se pueden compatibilizar a los incentivos diferenciales con un ideal de justicia distributiva. Concretamente, en un contexto en el que se garanticen un conjunto de libertades y una igualdad de oportunidades para todos los individuos, \citet{Rawls_1971} argumenta que las desigualdades que terminen mejorando la situación de los menos aventajados pueden ser permitidas bajo el principio de la diferencia.

Sin embargo, el filósofo Gerald Cohen (\citeyear{Cohen_2001}) ha elaborado una crítica al argumento presentado por Rawls. En particular, \citet{Cohen_2001} sostiene que se evidencia una contradicción dentro de la teoría rawlsiana ya que existen individuos que dicen estar motivados por una preocupación igualitaria pero a la vez exigen una compensación diferencial para emplear sus talentos de manera productiva. En este sentido, el autor argumenta que, además de fijarse en el diseño de las instituciones formales para que plasmen los ideales de justicia distributiva, resulta necesaria la existencia de un \textit{ethos} igualitarista que fomente ciertas conductas que tienden hacia la equidad. En particular, este \textit{ethos} lograría influir las conductas de los individuos que no son capturadas por las instituciones formales. En la misma línea, a la hora de presentar su preocupación por cuestiones de justicia distributiva, \citeauthor{Cohen_2014b} (\citeyear{Cohen_2014b}, \citeyear{Cohen_2014c}) ha propuesto una vuelta a los ideales socialistas de igualdad y comunidad. 

Un punto del planteo de \citet{Cohen_2014c} que cabe destacar es que el principal problema al que se enfrentan los teóricos del socialismo es un problema de diseño. Desde la teoría económica se puede extraer información sobre cómo diseñar unos arreglos institucionales que funcionen en base a la competencia. Sin embargo, para desarrollar una economía que funcione bajo otros valores, no existe en términos relativos la misma cantidad de conocimiento teórico. En este sentido, la propuesta de \citet{Cohen_2014c} puede ser vista como un desafío de diseño institucional. 

Ante este problema de diseño, el economista John Roemer (\citeyear{Roemer_2019}, \citeyear{Roemer_2021a}) propone que una posible solución es trabajar sobre el concepto de cooperación. El autor comparte la intuición de \citeauthor{Cohen_2001} (\citeyear{Cohen_2001}, \citeyear{Cohen_2014c}) respecto de los ideales socialistas y la necesidad de un \textit{ethos} igualitarista. Con este objetivo en mente, \citet{Roemer_2019} desarrolla el concepto de \textit{optimización kantiana}. A través de este concepto, el autor busca modelizar de manera alternativa el comportamiento cooperativo y en base al mismo, construir modelos económicos denominados como socialismo de mercado.

Teniendo en cuenta los puntos esbozados anteriormente, el objetivo del siguiente trabajo es realizar un recorrido a través de esta línea de investigación respecto de la pregunta de los incentivos. Como forma de acotar el problema, la discusión se centra en los incentivos económicos. A su vez, se omite la discusión de los casos en donde existen asimetrías de información. La idea es hacer énfasis en cómo los talentos y habilidades de las personas llevan a distintos resultados en la esfera económica. 

El trabajo se divide de la siguiente manera. En la sección \ref{sec2} se introduce el concepto de justicia distributiva. Este concepto normativo es la pieza fundamental de este trabajo. A continuación, se presenta una evaluación o crítica normativa de la teoría neoclásica respecto de las consecuencias que se siguen de la misma. Concretamente, una crítica a las desigualdades que se podrían general al emplear los incentivos económicos. Dicha crítica busca evidenciar la necesidad de brindar una respuesta alternativa. Por lo cual, se presenta la idea del principio de la diferencia propuesto por el filósofo John Rawls. Si bien este concepto implica un tratamiento normativo de mayor alcance que la teoría neoclásica, no está exento de críticas. En este sentido, en la sección \ref{sec3}, se analiza la crítica realizada por el filósofo Gerald A. Cohen a ciertas implicancias del principio de la diferencia. En la misma línea, se presenta la propuesta de Cohen respecto de la igualdad de oportunidades socialista y la necesidad de un \textit{ethos} igualitarista. Dicha propuesta es recogida por el economista John E. Roemer, por lo cual, en la sección \ref{sec4} se presenta un intento de modelización económica de la misma. Finalmente, en la sección \ref{sec5} se presentan un par de reflexiones que buscan dar cuenta de las limitaciones y virtudes de la línea de investigación que se recorre en este trabajo.
