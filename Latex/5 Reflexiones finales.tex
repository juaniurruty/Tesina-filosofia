




En el presente trabajo se parte desde un análisis en el marco de la teoría económica neoclásica. En dicho contexto, la presencia de incentivos diferenciales no es problemática sino que es deseable. En particular, las personas más productivas o más talentosas reciben estos beneficios adicionales. A su vez, los incentivos diferenciales conforman uno de los componentes que vuelve eficiente a la asignación de recursos resultante.

Al analizar esta situación, se llega a que dichos resultados son insatisfactorios desde el punto de vista de la justicia distributiva. En particular, como argumentan \citeauthor{Rawls_1971} (\citeyear{Rawls_1971}, \citeyear{Rawls_2002}) y \citet{Roemer_1998}, una distribución de remuneraciones bajo esta lógica ignora cómo las personas llegan a desarrollar sus capacidades. Concretamente, la asignación de recursos resultante es injusta en la medida que se encuentra influenciada en buena parte por resultados arbitrarios consecuencia de la lotería natural.

Por lo cual, el principio de la diferencia (\citeauthor{Rawls_1971}, \citeyear{Rawls_1971}, \citeyear{Rawls_2002}) se propuso como una alternativa. En este caso, los incentivos diferenciales son considerados justos en la medida de que se cumplan ciertas condiciones previas. Específicamente, se debe cumplir el primer principio de justicia el cual garantiza un mismo nivel de libertades básicas para los ciudadados. Adicionalmente, se debe cumplir el segundo principio de justicia que establece una igualdad de oportunidades en la posibilidad de acceder a cargos y posiciones. Si se cumplen estos principios, es posible habilitar al principio de la diferencia. Este principio permite la existencia de desigualdades sociales y económicas siempre y cuando las mismas favorezcan al grupo menos aventajado de la sociedad.

Ahora bien, como argumenta \citet{Cohen_2001}, existe una tensión entre el ideal igualitario y los incentivos diferenciales. El autor sostiene que si las personas se rigen por el principio de la diferencia, no deberían de requerir incentivos diferenciales para realizar un trabajo que beneficie a los menos aventajados. Desde la perspectiva rawlsiana, si consideramos que los principios de justicia han de ser aplicados exclusivamente a la estructura básica de la sociedad, entonces hay un espacio de elecciones individuales que quedan por fuera del ámbito de la justicia. Estas elecciones podrían socavar las ambiciones equitativas de las instituciones que conforman a la estructura básica. Por lo cual, \citet{Cohen_2001} sostiene que es necesario contar un \textit{ethos} igualitario que sea reduzca estos comportamientos contrarios a la igualdad.

En la misma linea, \citet{Cohen_2014c} argumenta que el socialismo cuenta con ideales que apuntan hacia la igualdad. En particular, el autor plantea que los ideales normativos principales del socialismo son la igualdad y la comunidad. Estos ideales se materializan en el concepto de igualdad de oportunidades socialista. Este concepto implica la corrección de las desventajas que no son elegidas por parte de los individuos, es decir, aquellas donde la persona no puede ser considerada racionalmente responsable de ellas.

Con este planteo en mente, Roemer busca desarrollar un modelo de socialismo de mercado que pueda plasmar los ideales de igualdad y comunidad. Para dar cuenta de estos ideales, \citeauthor{Roemer_2019} (\citeyear{Roemer_2019}, \citeyear{Roemer_2021a}) se concentra en el comportamiento cooperativo. En particular, el autor desarrolla el concepto de optimización kantiana, el cual pretende ser una alternativa al modo de razonamiento que se emplea en los equilibrios de Nash. Al emplear este nuevo criterio, se pretende explicar cómo pueden surgir los comportamientos cooperativos. Mediante la optimización kantiana es posible construir modelos de socialismo de mercado en los cuales se supera el \textit{trade-off} entre eficiencia y redistribución.

Una idea que se rescata del análisis previo es que resulta poco razonable confiar en que los derechos de propiedad o los arreglos institucionales formales hagan todo el trabajo a la hora de materializar ciertos ideales de justicia e igualdad. Resulta necesario establecer un \textit{ethos} igualitario que sea parte vital de la construcción de sociedades justas. Por lo cual, esto genera un problema de diseño. En particular, ¿cómo han de diseñarse los arreglos institucionales formales e informales para que se establezca y se mantenga este \textit{ethos}? El desarrollo de Roemer es un aporte invaluable a la pregunta sobre el diseño institucional. En particular, centrarse en la cooperación a través del concepto de \textit{optimización kantiana} es una estrategia prometedora. 

De todas formas, respecto de la \textit{optimización kantiana} cabe hacer unos comentarios. En primer lugar, parece insatisfactorio el resultado de que observacionalmente sería lo mismo que las personas posean o no altruismo desde esta perspectiva. Esto puede ser cierto al menos desde un ámbito enfocado en los resultados dentro del mercado. Por fuera de las esferas de la economía, el altruismo puede ser deseable para sostener el \textit{ethos} igualitario y la idea de comunidad. El ideal de comunidad puede ir mas allá de las interacciones dentro de los mercados. Por lo tanto, una sociedad en la cual las personas muestren cierto grado de altruismo debería reflejar el ideal comunitario del socialismo de mejor manera que una sociedad en la cual las personas poseen preferencias meramente auto-interesadas.

En segundo lugar, quedan sin resolver los problemas que surgen en contextos de información imperfecta. En particular, como se esbozó en la introducción, los problemas de principal-agente como el riesgo moral. En los modelos analizados, dicho problema podría no ocurrir ya que las personas son concebidas de manera diferente a lo que plantea la teoría neoclásica. Concretamente, las personas optimizan empleando un protocolo diferente. Este punto es recogido por \citet{Roemer_2021b}, el autor argumenta que este problema no ha sido tratado con rigurosidad en su trabajo. Sin embargo, una posible respuesta es que los problemas de este tipo no serían tan salientes. En especial, si se toma en cuenta que uno de los requisitos para que las personas empleen el criterio de la \textit{optimización kantiana} es que haya confianza entre las personas.

Por otro lado, queda abierta la pregunta de si los nuevos modelos de socialismo de mercado propuestos por \citet{Roemer_2021a} logran superar la crítica de \citet{Cohen_2014c}. Se puede evidenciar un mayor refinamiento teórico entre el modelo de \citet{Roemer_1994} y los modelos que incorporan la \textit{optimización kantiana}. No cabe duda que el denominado \say{primer teorema de la economía del bienestar en la socialdemocracia} es un resultado interesante ya que permite superar las pérdidas de eficiencias asociadas a la redistribución de ingresos. De todas formas, en estos modelos no se definen las instituciones que permitirían el desarrollo de los ideales de igualdad y comunidad. El rasgo distintivo de estas sociedades es el logro de los objetivos de eficiencia y redistribución. 

El argumento de Roemer podría ser que lo sustancial de este resultado es la forma en que se llega a la eficiencia y la redistribución. En particular, a través de los requisitos necesarios para la \textit{optimización kantiana} (comprensión, deseo y confianza) es que se lograrían establecer los ideales de igualdad y comunidad. Sin embargo, este tipo de comportamiento se desarrolla en las interacciones económicas. Por lo cual, cabe preguntarse cómo actuarían los individuos en otros contextos. 

Finalmente, si bien los modelos de \citeauthor{Roemer_2019} (\citeyear{Roemer_2019}, \citeyear{Roemer_2021a}) resultan sólidos en términos teóricos, cabe preguntarse cuáles serían los pasos a seguir para buscar su implementación. En términos prácticos, se podría argumentar que el primer objetivo sería el de fomentar las precondiciones para la \textit{optimización kantiana} (comprensión, deseo y confianza). Ahora bien, la pregunta es si sería posible llegar a dicha situación partiendo de una sociedad en la cual predomina un \textit{ethos} individualista. En mi opinión creo que dicha transición es posible y conocer cómo realizarla constituye una línea de investigación que se podría desarrollar a futuro.




\newpage

